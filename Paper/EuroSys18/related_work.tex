Works on performance bug diagnosis usually fall into the following categories.
The most naive way is profiling.
The souce code should be accesible and it is of high ovehead and less practical especially when the performance bug that appears intermittentely.
The second category makes us of the statistic technique or machine learning.
It is useful in identify the possible performance bug.
However, it is not hard to tell the root cause of the performance bugs.
The second category is on the analysis of the dynamically collected data and examine the dependancy graph.
The limitation is the method may work well in the system that are fully understanded by the tool developers.
Our work complements this category of works in unfamiliar systems.
