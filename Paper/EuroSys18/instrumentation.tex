%Instumentation
Considering the trade off between userfulness and overhead, we first choose the well known events that previous work chosen.
We instrument the Cocoa Unbrella Framework for user input and display update.
Ueser input: user inputs from the I/O kit to the target application will refer the Umbrealla Frameworks HIToolbox and AppKit.
User inputs like key stroke and mouse move may get collapse in queues.
We trace the user inputs in the AppKit framework as it was done in previous work.
Display update: we not only record the updates but also trace the code locations where the need\_display flags are set. With this method, we track both the display updates with View class and CALayer class.
Asynchrouns call: asynchrous calls in MacOS can be implemented via RunLoop object, Grand Central Dispatcher, timer and wait queue.
IPC: Mach message is the main mechanism for IPC.
Thread synchronization: We also trace the thread synchronization by tracking the thread scheduling event, including to wait and to be made runnable, which has better coverage.
To provide hints for developers to exploring programming paradigms, we also add backtrace after every mach message and dispatched function execution.
We also trace certain events to exclude the obvious background noise introduced by hearbeat threads. 
The heartbeat threads in the system are identified with the interruption the running thread itermittently.
We address the interrupts and time share maintanence by isolate the event and the thread activites that it triggers from the current running thread activities.
Other potential heartbeat theread activities may relate to programing paradigms are left later for inspections.
