Confusing behaviors are everywhere in GUI Applications, which usually make the user feel annoying.
Spinning beachball is one of them in macOS.
However, even unveiling the mystery of its design is not easy due to the closed source.
Consequently, it is difficult for users to begin the debugging task in the wild.
A tool to understand the complex behavior of GUI Apps in MacOS is essential, especially for a user who is neither a system expert nor the developer of the applications, to aid debugging, compile concise bug report, or even come up a temporary binary patch.

Argus is the framework where system-wide activity is recorded until the mystery occurs in macOS, and analysis tools are built to explain the behavior.
It monitors the system with low-overhead instrumentation running 24X7, from kernel and libraries, and constructs the relationship graph with the schema defined in the framework.
The schema identifies execution boundaries in threads and the causality among them.
The graph usually contains not only the problematic case but also the normal execution.
Upon the graph, analysis tool compares them to manifest the root cause of the anomaly.
Results are represented with the suspicous control flows in anomaly case.
Hence users can precisely point out the path leading to the performance anomaly.
In this paper, we describe and study the usage of our Argus to reveal root causes for the impenetrable behaviors from the real world applications in macOS.
