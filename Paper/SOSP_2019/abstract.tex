Prior systems used causal tracing, a powerful technique that traces low-level
events and builds dependency graphs, to diagnose performance issues.  However,
they all assume that precise dependencies can be inferred from low-level
tracing by either limiting applications to using only a few supported
communication patterns or relying on developers to manually provide dependency
schemas upfront for all involved components.  Unfortunately, based on our own
study and experience of building a causal tracing system for MacOS, we found
that it is extremely difficult, if not impossible, to build precise dependency
graphs.  We report patterns such as data dependency, batch processing, and
custom communication primitives that introduce imprecision.  We present \xxx, a
practical system for effectively debugging performance issues in modern desktop
applications despite the imprecision of causal tracing.  \xxx lets a user
easily inspect current diagnostics and interactively provide more domain
knowledge on demand to counter the inherent imprecision of causal tracing.  We
implemented \xxx in MacOS and evaluated it on \nbug real-world, open
spinning-cursor issues in widely used applications.  The root causes of these
issues were largely previously unknown.  Our results show that \xxx effectively
helped us locate all root causes of the issues and incurred only 1\% CPU
overhead in its system-wide tracing.


%Confusing behaviors are everywhere in GUI Applications, which usually make the
%user feel annoying.  Spinning beachball is one of them in macOS.  However, even
%unveiling the mystery of its design is not easy due to the closed source.
%Consequently, it is difficult for users to begin the debugging task in the
%wild.  A tool to understand the complex behavior of GUI Apps in MacOS is
%essential, especially for a user who is neither a system expert nor the
%developer of the applications, to aid debugging, compile concise bug report, or
%even come up a temporary binary patch.
%
%Argus is the framework where system-wide activity is recorded until the mystery
%occurs in macOS, and analysis tools are built to explain the behavior.  It
%monitors the system with low-overhead instrumentation running 24X7, from kernel
%and libraries, and constructs the relationship graph with the schema defined in
%the framework.  The schema identifies execution boundaries in threads and the
%causality among them.  The graph usually contains not only the problematic case
%but also the normal execution.  With our tool, the problematic range can be
%easily narrowed down with with comparison.  Our interactive debugging tools are
%provided to debugging on refind rang to manifest the root cause of the anomaly.
%In this paper, we describe and study the usage of our Argus to reveal root
%causes for the impenetrable behaviors from the real world applications in
%macOS.
