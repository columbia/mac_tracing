\section{Dependency Semantics}
\label{sec:dependency-semantics}
\subsection{Tracing Events}
The log that \xxx collects consits of a squence of tracing events, falling into three categories:
semantics tracing events, dependency tracing events and boundary tracing events.
Semanctics events include system calls, backtraces, and instruction logs.  These events help diagnosis and are also used by \xxx to find similar events in its analysis.
Boundary events are recorded within a single thread where we may eventually place an execution segment boundary.  These events tend to be the calls to and returns from wait operations.
Dependency events are recorded whenever one thread comunnicates with another,such that we will eventually create causual links---for example, a thread calls \textit{mach\_msg\_send} and delivers message to anther thread which calls \textit{mach\_msg\_receive}. These two dependency events will eventually be used to create a link in the graph.  These categories are not disjoint.  For instance, most events are logged together with the call stacks in which they occur.  Similarly, the return from aforementioned \v{mach\_msg\_receive} is both a dependency event and a boundary event as it begins a new execution segment.

\subsection{Dependency Patterns}
\label{sec:patterns}

We encountered several instances of runtime event dependencies, correct or incorrect, between thread
contexts. We present several generalizable cases below.

\para{P1: Signal or interrupt handling}
Sometimes, a signal or interrupt handler happens to run within a thread's context, for
example, a timer interrupt.  As discussed in the previous section, \xxx logs wait and wake-up events inside the kernel to increase the completeness of tracing, so it may log events that occur in the call to signal or interrupt handler.  We identify the start and end of the
signal-handling code to splice it away from the containing context, since it is
usually unrelated.


{\footnotesize \begin{verbatim}
Notes application

stat()
interrupt() {
  state = save_contex  
  lapic_interrupt(intr, state)
}
wait(lock_mutex)


\end{verbatim}
}

\para{P2: Kernel takes over context}
As part of a thread context switch, an execution context may enter kernel
space. As shown below, the code will enter kernel scheduling by calling
\v{sched\_timehsare\_consider\_maintance}, which in turn wakes up another
\texttt{kernel\_task} thread.  Again, such a wake-up does not reflect the application's intent, and should be filtered out from the containing context.  

{\footnotesize \begin{verbatim}
thread_invoke(self, new_thread, reason) {
// thread switch, kernel space
  sched_timehsare_consider_maintance() {
    thread_wakeup(sched_timeshare_miantenance_continue);
  }
  ast_context(new_thread);
}
\end{verbatim}
}

To detect this case, we filter wakeups from the kernel timer, interrupt
handler, or kernel shared memory maintenance. Such cases represent spurious
dependencies. However, sometimes when a worker thread wakes up another worker
thread, this can represent a true dependency. The distinguishing feature is
whether a synchronization primitive (including shared memory) is used.

\para{P3: Batching and data dependency in event processing}
The WindowServer MacOS system daemon contains an event loop which waits on Mach
messages. Conceptually, it processes a series of independent events from
different processes. However, to presumably save on kernel boundary crossings, it uses a
single system call to receive data and send data for an unrelated
event. This batch processing artificially makes many events appear dependent, and we split
the execution segments to maintain the independence of the events.

This case also illustrates a causal linkage caused by data dependency
within one thread.  As the following code shows, WindowServer saves the
reply message in variable \v{\_gOutMsg} inside function
\v{CGXPostReplyMessage}.  When it calls \v{CGXRunOneServicePass}, it sends
out \v{\_gOutMsg} if there is any pending message.  This data dependency
needs to be captured in order to establish a causal link between the
handling of the previous request and the send of the reply.
Interestingly, it is an example of a data dependency within the same
thread.  \xxx uses watch point registers to capture events on these data
flags and establish causal links between them.

%// reduces kernel crossing presumably
%// only keeps one pending message
{\footnotesize \begin{verbatim}
  while() {
    CGXPostReplyMessage(msg) {
      // send _gOutMsg if it hasn't been sent
      push_out_message(_gOutMsg)
      _gOutMsg = msg
      _gOutMessagePending = 1
    }
    CGXRunOneServicePass() {
      if(_gOutMessagePending)
        mach_msg_overwrite(MSG_SEND | MSG_RECV, _gOutMsg)
      else
        mach_msg(MSG_RECV)
      ... // process received message
    }
  }
\end{verbatim}
}

\para{P4: CoreAnimation shared flags}
A worker thread can set a field \v{need\_display} inside a CoreAanimation object
whenever the object needs to be repainted. The main thread iterates over all
animation objects and reads this flag, rendering any such object. This shared-memory communication creates
a dependency between the main thread and the worker so accesses to these
field flags need to be tracked.  However, since each object has such a
field flag, \xxx cannot afford to monitor each using a watch point
register.  Instead, it uses instrumentation to modify the CoreAnimation
library to trace events on these flags.

% , but these dependencies are extremely common and do not communicate much information.

{\footnotesize \begin{verbatim}
  Worker thread that needs to update UI:
  ObjCoreAnimation->need_display = 1

  Main thread:
  traverse all CoreAnimationobjects
  if obj->need_display == 1
    render(obj)
\end{verbatim}
}

\para{P5: Spinning cursor shared flag}
Whenever the system determines that the main thread has hung for a certain
period, and the spinning beach ball should be displayed, a shared-memory flag
is set. Access to this flag is controlled via a lock, i.e. the lock is used for
mutual exclusion, and does not imply a happens before relationship.  Thus,
\xxx captures accesses to these flags using watch-point registers to add
causal edges correctly.

{\footnotesize \begin{verbatim}
 NSEvent thread                     
 CGEventCreateNextEvent() {
   if sCGEventIsMainThreadSpinning == 0x0
      if sCGEventIsDispatchToMainThread == 0x1
        CFRunLoopTimerCreateWithHandler{
          if sCGEventIsDispatchToMainThread == 0x1
            sCGEventIsMainThreadSpinning = 0x1
            CGSConnectionSetSpinning(0x1);
        }
 }

 Main thread
 Convert1CGEvent(0x1);
 if sCGEventIsMainThreadSpinning == 0x1
   CGSConnectionSetSpinning(0x0);
   sCGEventIsMainThreadSpinning = 0x0;
   sCGEventIsDispatchedToMainThread = 0x0;

\end{verbatim}
}

\para{P6: Dispatch message batching} The message dispatch service dequeues
messages from many processes and staggers processing of the messages. This
creates false dependencies between each message in the dispatch queue.  As
illustrated in the following code snipped from the \v{fontd} daemon,
function \v{dispatch\_execute} is installed as a callback to a dispatch
queue.  It subsequently calls \v{dispatch\_mig\_server()} which runs the
typical server loop and handles many messages.  To avoid incorrectly
linking many irrelevant processes through such batching processing
patterns, \xxx adopts the aforementioned heuristics to split an execution
segment when it observes that the segment sends out messages to two
distinct processes.

{\footnotesize \begin{verbatim}
// fontd daemon main thread
block = dispatch_queue.dequeue()
dispatch_execute(block)
{
  dispatch_mig_server()
}

dispatch_msg_server()
{
  for(;;) {
    mach_msg(send_reply, recev_request)
    call_back()
    set_reply()
  }
}

// fontd daemon worker thread
dispatch_queue.enqueue(block)
\end{verbatim}
}

This pattern does pose a challenge for automated causal tracing tools that
assume that the entire execution of a callback function is on behalf of
one request~\cite{xxx}.  The code shown uses a dispatch-queue callback,
but inside the callback, it does work on behalf of many different
requests.  Any application or daemon can implement its  own server loop
this way, which makes it fundamentally difficult to automatically infer
event handling boundaries.

\para{P7: Mach message mismatch}
When RPC-style inter-process communication is used, most systems would use
the same thread to send the call request and receive the return.  They
would also use one recipient thread to process the call.  However, in
MacOS, the thread sending the call request may be different from the one
receiving the return, and multiple threads may be used in the recipient
thread to handle the request.  Fortunately, the messages involved
typically carry metadata such as the \v{reply\_port} shown in the
following diagram, so \xxx can easily link the corresponding events
together.

{\footnotesize \begin{verbatim}
SCIM calls XPC_connection_send_msg
t1_scim   t2_chromium       t3_chromium     t4_scim
|         |                 |               |
mach_msg_send(remote_port, reply_port)      |
          |                 |               |
          dispatch_mach_msg_receive(remote_port, reply_port)
                            |               |
                            mach_msg_send(reply_port)
                                            |
                                    mach_msg_receive(reply_port)
\end{verbatim}
}

\para{P8: Runloop callbacks batch processing}
As is common in event driven programming, many methods can post a callback
and MacOS uses runloop as a common idiom to process callbacks.  As shown
in the following step-by-step description of the MacOS runloop, an
iteration of the runloop does 10 different stages of processing, each of
which may do work on behalf of completely irrelevant requests.  Since
there are no obvious events (\eg, a wait operation) to split the
execution, \xxx uses instrumentation to add beginning and ending points
for MacOS runloops.  In general, any application or daemon can create
its own version of the runloop, posing challenges for automated
inference of event processing boundaries.

% // another thread installs cb
% performSelector:onThread:withObject:waitUntilDone;

{\footnotesize \begin{verbatim}
Run loop sequence of events //developer.apple.com
1-3.Notify observers
4.Fire any non-port-based input sources
5.If a port-based input source is ready and waiting to fire,
    process the event immediately. Go to step 9.
6.Notify observers that the thread is about to sleep.
7.Put the thread to sleep until:
    //one of the following events occurs
    An event arrives for a port-based input source.
    A timer fires.
    The timeout value set for the run loop expires.
    The run loop is explicitly woken up.
8.Notify observers that the thread just woke up.
9.Process the pending event.
  If a user-defined timer fired,
    process the timer event
    restart the loop.
    Go to step 2.
  If an input source fired
    deliver the event.
  If the run loop was explicitly woken up, but not timed out,
    restart the loop. Go to step 2.
10.Notify observers that the run loop has exited.

\end{verbatim}
}

%% \para{P9: Timers}
%% Most timers in MacOS are repeat timers, meaning that the timer reregisters itself before finishing.
%% This creates complex dependencies because timers are invoked asyncronously during interrupts.

%% {\footnotesize \begin{verbatim}
%% \
%% timer_create
%%      |       \ (may or may not fire)
%%      |        timer_expire
%%      |
%% timer_cancel
%%      |
%% timer_create
%%      |
%% timer_create  // repeat timer?
%%             \timer_expire

%% \end{verbatim}
%% }

\subsection{Discussion}

These patterns reinforce our insights described in
Section~\ref{sec:intro}.  First, as illustrated by patterns P3, P6, and
P7, batch processing is frequently used to lump work on behalf of
different requests together, causing inaccurate edges that do not reflect
causality in the event graph.  Some amount of manual schema is needed, but
annotating all applications and daemons upfront to expose batch processing
would be strenuous and error-prone.  We believe \xxx's interactive
approach represents a better design tradeoff.  Second, as illustrated by
P3, P4, and P5, systems use data flags for causal linkage.  They do so in
both multithreaded and single-threaded environments.  Such data
dependencies can be crucial for diagnosing performance issues -- the
spinning cursor's display itself uses shared memory flags as shown in P5.
Third, as P1, P2, and P5 show, wake-ups do not necessarily reflect causal
linkage.  Lastly, systems may deviate from well-established patterns such
as in P7 or create their custom primitives such as in P6.
