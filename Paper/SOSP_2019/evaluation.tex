\section{Performance Evaluation}
\label{sec:evaluation}
%\begin{table}[h]
%\begin{tabular}{|l|c|c|}
%Apps &  Memory & CPU\\
%\hline
%iBecnh & & \\
%TextEdit & & \\
%CoEdit & & \\
%Notes & & \\
%HopperDiassembler & & \\
%Installer & &\\
%Squel Pro & & \\
%GetiPlayerAutomator & &\\
%\end{tabular}
%\caption{overhead}
%\label{tab:overhead}
%\end{table}
We deployed \xxx on a Mac OS X x86 system, model MacBookPro9,2.  The
model has the Intel Core i5-3210M CPU with 2 cores and 4 thread, 10GB DDR3
memory, and we replaced the hard disk with a 1T SSD.  The tracing tool is
running in the background 24X7.  Once the spinning cursor appears on the
screen, we store the tracing data for root cause analysis.

In previous section, we studied real-world software, including TextEdit,
Notes, Installer, System Preferences and Chromium. TextEdit, Notes, Installer and System
Preferences are distributed by Apple.  
Our tool can take over most burden searching and comparing work from the user,
and make the diagnosis much easier in the wild.

In this section we present the performance impact of the live deployment of Argus.
As we use the ring buffer to collect events, the storage cost can be adjusted
and is fixed to 2G in our experiments.
Since the internal memory used to collect data is fixed to 512M, so the overhead of the
memory is pretty low with regards to the memory usage of  morden applications.
We show the CPU overhead of \xxx first for the iBench Running on the system with the tracing
on and off. In the following description, we call the environment without the tracing on as
bare run, and otherwise tracing run.

\begin{table}[h]
\begin{tabular}{l|c|c|c|c|c}
\hline
 & 1st & 2nd & 3rd & 4th & 5th\\
\hline
 bare run & 5.98 & 6.23 & 6.18 & 6.05 & 6.28\\
\hline
 tracing run& 6.29 & 6.01 & 6.09 & 6.28 & 6.01\\
\hline
\end{tabular}
\caption{Score From iBench}
\label{tab:ibench}
\end{table}
We show the five runs of both cases in Table\ref{tab:ibench}.
For each run, the machine is clean boot for each run.
The scores are quite close and show no difference on the system running with and without \xxx.

We also perform the evaluation on benchmark of Chromium, while recoding the time usage.
The telemetry from Chromium project is used to measure the CPU overhead of \xxx.
In the experiment, it uses the chromium to launch webpage and scroll over.
The result is shown in Table~\ref{tab:chromium benchmark}. 
\begin{table}[h]
\begin{tabular}{l|c|c}
\hline
 & bare run & tracing run \\
\hline
real & 27.7s & 28.0s \\
\hline
user & 28.3s & 28.3s \\
\hline
sys &  5.0s & 5.7s\\
\hline
\end{tabular}
\caption{Chromium benchmark: telemetry}
\label{tab:chromium benchmark}
\end{table}

As shown in Table~\ref{tab:chromium benchmark}, the overhead for real, user and sys are 1\%, 0\% and
14\% respectively.
