\subsection{System Preferences spin}
System Preferences is the application in MacOS for users to modify various settings.                                        
The Preference Pane named Displays allows the user to rearrange the position of displays,                                   
the location of the menu bar and set parameters for display, such as the resolution, brightness, and rotation,              
but to disable a monitor online is not supported in Mac.                                                                    
DisableMonitor is an app to easily disable/enable a monitor.                                                                
It is implemented by calling to the API from Apple, CGSBeginDisplayConfiguration, CGSConfigureDisplayEnabled, and CGSCompleteDisplayConfiguration.
The bug appears when an external monitor is disabled with DisableMonitor, and the user drags the windows under the tab of arrangement in System Preference.
It makes the System Preferences window freeze for a few seconds.

%Without the source code, it is hard to tell the story behind.
One straightforward method for the single process app is to attach lldb, resume the process until the system throws out the spinning cursor.
However, in the very case, it only tells when the spinning cursor shows up, the main thread is busy calling thread\_switch from function CGSCompleteDisplayConfiguration in the call stack.
Further static analysis, for example, reverse engineering on the binary, cast light on the programming paradigms of the repeating.
In our case, it reveals the main thread is stuck in a loop repeating thread\_switch until a variable is set or time out.
Nevertheless, it is still hard to answer the question of why and how the variable gets set or if it is cleared by mistake.

Only the comparison of the buggy case and the normal execution can exhibit what is not expected.                         
One advantage of our tool is it is lightweight and does not impact the responsiveness of the thread,                     
we can identify the buggy case and normal case,                                                                          
compared to using lldb to trace through the whole API execution step by step.                                            
Another advantage is that it records the activities system-wide,                                                         
and the data covers from the beginning of the user input, which makes the case more explainable.                         
By adding the hardware breakpoints with our instrumentation tool,                                                        
we also record the whole history of the variable activity.                                                               
The tracing on the variable helps to make alignment of the graphs produced in the normal execution and freeze case, which makes the further comparison feasible.
It reveals datagram from the WindowServer will post notifications for the application.                                   
The particular datagram makes the application to finish reconfiguration and set the variable, while in the spinning case the reconfiguration gets initiated but not completed.
The handler display\_notify\_proc is not appropriately implemented to make the application exit the loop when there is an exception in the reconfiguration.

With the findings of our tool, as is shown in Figure \ref{fig: Thread communication graph with call stacks},             
we can now explain the root cause with the pseudocode in Figure \ref{fig: pseudo code for displaynotifyproc} and Figure \ref{fig: pseudo code for WaitReconfigurationCycle},
which provides insight for the developer to fix the bug.
