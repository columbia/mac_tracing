\section{Related Work}
While there is no system target for providing tools for users to assist the debugging of closed source applications on MacOS, several active research topics are closely related.

textbf{Correlate tracing events:} Aguilela \cite{blackbox} use timing analysis to correlate messages and treat the system as a black box.
Argus encompass the system knowledge to correlate messages, asynchronous operations, and share variables to achieve a finer grained analysis.                                                                                                                       
Magpie \cite{magpie} is the most closely one which is a system monitoring and modeling server workload on Windows system.
The goal of Magpie is to model a workload with normal behavior and with the data set to detect the anomaly one with statistic method. 
Our goal is to identify the root cause of anomaly with the relationship graph, which includes not only the normal execution but also the anomaly one.
Meanwhile, for the boundary of a request and causality of traced events, Magpie depends on the detailed knowledge of application semantics from the developer, while Argus defines them in the framework, and the semantics of the application is not required.
                                                                                                                         
Panappticon \cite{pannappticon} monitors the mobile system and use the trace to characterize the user transactions of mobile apps. 
Although it aims to track system-wide and correlate events without developer input, the design is base on two thread models in its assumption. 

AppInsight \cite{appinsight} instruments application to identify the critial execution path in user transaction.         
It leverages the semantics and opcode in high-level frameworks, which is not available in MacOs.                         
It does not track the request across process/app boundary. Thus it cannot reveal the root cause due to other processes.      
                                                                                                                         
XTrace and Pinpoint \cite{xtrace, pinpoint} both trace the path of a request through a system using a unique identifier attached to each request and stitch traces together with the identifier.
Argus does not use identifier due to the maintained overhead and propagation of such identifiers through the developing environments is not feasible with the closed source applications, frameworks, and libraries.
                                                                                                                         
\textbf{Performance anomaly:}\cite{} leverage the user logs and call stacks to identify the performance anomaly.         
From our daily sense, the anomaly bugs whose root cause is hard to reveal are from the inefficient codes that escape the developers' bullets.                                                                                                        
                                                                                                                         
\cite{} apply the machine learning method to identify the unusual event sequence as an anomaly.
\cite{} generates the wait and waken graph from sampled call stacks to stidy a case of performance anomaly.
