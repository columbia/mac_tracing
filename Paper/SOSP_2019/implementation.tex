\section{Implementation}
We now discuss how we collect tracing events of interest.

\subsection{Tracing Tool}
Current MacOS systems support a system-wide tracing infrastructure built by Apple. [traces what]
By default, the infrastructure temporarily stores events in memory and flushes them to screen or disk when an internal buffer is filled.
We extended this infrastructure to support larger-scale tests without filling up the disk by implementing a ring buffer backed by a file.
We store at most 2GB of data [per log?], which corresponds to approximately XXX events (XXX time).

\subsection{Instumentation}
Like Detour~\cite{detourXXXXXXXX}, we use static analysis to decide which instrumentation to perform, and then enact this instrumentation at runtime. 
On MacOS, most libraries as well as many of the applications used day-to-day are closed-source.
Adding tracing points to such code requires binary instrumentation.
Techniques such as library preloading to override individual functions are not applicable on MacOS, as libraries use two-level executable namespaces.
Hence, we implemented a binary instrumentation mechanism that allows developers to add tracing at any location in a binary image.

To add instrumentation, we insert 5-byte call instructions into the program. The user finds a location of interest in the code related to a specific event,
and we overwrite the victim instructions at that location. We create a new trampoline target function, whose first few instructions are those which were overwritten.
All of the trampoline functions are grouped together by our tool and a new library is generated.
This library provides the same public API as the original and is a drop-in replacement. We load and call the original code as an unmodified shared library.
The detours or trampoline calls are added by an initialization function in our new library; we temporarily mark the code region as writable with \texttt{mprotect}
to calculate offsets and perform the modifications. The initialization is called externally through \texttt{dispatch\_once}.
To use the modified libraries, we simply replace system libraries in their original locations (renaming them so that our code can access the originals).

One potential issue is that we use 5-byte call instructions with 32-bit displacements to jump from the original library to our new one.
This design requires that the libraries be loaded within +/- 2GB of each other in the 64-bit process address space.
However, since we list each original library as a dependency of our new libraries, the system loader will map each new and original library in sequence.
In practice, the libraries ended up very close to one another and we did not see the need to implement a more general long-jump mechanism.

\subsection{Incremental Tracing}
The graph should be incrementally improved with new tracing points.
The procedure to discover such programming paradigm can be repeated on regular executions before tracing for diagnosis.
The missing connections are much harder to explore.
As long as the remaining connections in the current graph help diagnosis, it is not necessary to explore.

\xxx provides two lightweight tools for users to collecting data with incremental tracing, instead of the lldb.
\begin {itemize}
\item For any given shared variable of interest, we take advantage of hardware watchpoints.
	Tracing points are recorded in the watchpoint handler when the variable is accessed.
	We hook the handler in CoreFoundation to make sure that it is loaded correctly into the address space of our target application.
	We set the hardware watchpoint in an ad-hoc manner with a custom command-line tool.
\item For any code location where user want to check its call stacks, our interface accepts a tag as input to distiguish.
	It unwindes the rbp from the user stack to store the valid return addresses in the buffer. The buffer is recorded into the log as tracing events.
	These address woulg go through the offline symbolicator in the graph construction phase.
\end{itemize}

\subsection{Debugger Scripting}
After the offline analysis on the graph, we take the API covers the fine range as input to our debugging scripts.
The debugging scripts go throught the instruction from application and higher level frameworks step by step.
The purpose is to capture the parameters results from the user interaction.
Once a new function begins by checking the instruction, we record the call stacks for comprehension. 
For API from the low level libraries, such as pthread, we step over and record the return value.
AS the operation are confined in the small range, the overhead is not too much.

Both the execution on normal case and problematic case are recorded, our tool further compares the log and report
the difference, with the full call stack.

