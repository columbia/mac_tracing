The spinning beachball is a painful sight for Mac users, signifying that the
application is non-responsive. It usually remains for minutes at a time,
leaving the user at a loss and unable to do anything productive.

\xxx shreds light on the design of the spinning wait cursor with its backward
path slicing. We begin the path from the node where spindump, a hang reporting
tool, is launched, considering the spindump shares the same triggering
condition. For our tracing data, we found out that Spindump is launched after
receiving a message from WindowServer which receives a message from the NSEvent
thread of the target application first.

We further added call stacks for the messages and revealed two shared
variables, ``\v{is\_mainthread\_spinning}'' and
``\v{dispatch\_to\_mainthread}'', are critical in the desgin. The NSEvent
thread of the target application fetches CoreGraphics events from WindowServer,
converts and creates NSEvents for the main thread. If the main thread is not
spinning with ``\v{is\_mainthread\_spinning}'' equals 0 ,
``\v{dispatch\_to\_mainthread}'' will be set to 1 and a timer is armed. If the
main thread processes the next event before the timer fires, nothing happens
and the timer gets re-armed. Otherwise, NSEvent thread sends a message inside
``\v{CGSConnectionSetSpinning}'' to WindowServer from the timer handler, and
WindowServer notifies the CoreGraphics to draw a spinning wait cursor over the
application window.
