The spinning wait cursor is a painful sight for Mac users, signifying that the application is non-responsive.
It usually remains for minutes at a time, leaving the user at a loss and unable to do anything productive.

Argus shreds light on the design of the spinning wait cursor with its backward path slicing.
We begin the path from the node where spindump, a hang reporting tool, is launched,
since spindump shares the same triggering condition.
%As is shown in Figure \ref{fig:spincursor},
Spindump is launched by the message from WindowServer, after WindowServer receives a message from the NSEvent thread of the targeted application.

We further added call stacks for the messages and revealed two shared variables, ``\v{is\_mainthread\_spinning}'' and ``\v{dispatch\_to\_mainthread}'', are critical in the desgin,
The NSEvent thread of the targeted App fetches CoreGraphics events from WindowServer, converts and creates NSEvents for the main thread.
If the main thread is not spinning with the cleared ``\v{is\_mainthread\_spinning}'', ``\v{dispatch\_to\_mainthread}'' is set and a timer is armed.
If the main thread processes the next event before the timer fires, nothing happens and the timer gets re-armed.
Otherwise, NSEvent thread sends a message to WindowServer from the timer,
and WindowServer notifies the CoreGraphics to draw a spinning cursor over the application window.
