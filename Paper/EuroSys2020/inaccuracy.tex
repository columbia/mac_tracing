\section{Inherent Inaccuracies in Causal Tracing}\label{sec:inaccuracy}
%figure dispatch message batching
\begin{figure}[t]
\begin{minipage}[t]{.25\textwidth}
\begin{lstlisting}
//worker thread in fontd:
//enqueue a block
block = dispatch_mig_sevice;
dispatch_async(block);
\end{lstlisting}
\end{minipage}\hfill
\begin{minipage}[t]{.21\textwidth}
\begin{lstlisting}
//main thread in fontd:
//dequeue blocks
block = dequeue();
dispatch_execute(block);
\end{lstlisting}
\end{minipage}

\begin{minipage}[t]{0.48\textwidth}
\begin{lstlisting}
//implementation of dipatch_mig_server
dispatch_mig_server()
for(;;) //batch processing
  mach_msg(send_reply, recv_request)
  call_back(recv_request)
  set_reply(send_reply)
\end{lstlisting}
\end{minipage}
%\vspace{-0.3cm}
    \caption{Dispatch message batching}
    \label{fig:dispatchmessagebatching}
\end{figure}

%figure batching in event processing
\begin{figure}[t]
\begin{lstlisting}
//inside a single thread
while() {
  CGXPostReplyMessage(msg) {
  // send _gOutMsg if it hasn't been sent
    push_out_message(_gOutMsg)
    _gOutMsg = msg
    _gOutMessagePending = 1
  }
  CGXRunOneServicePass() {
    if (_gOutMessagePending)
      mach_msg_overwrite(MSG_SEND | MSG_RECV, _gOutMsg)
    else
      mach_msg(MSG_RECV)
    ... // process received message
  }
}
\end{lstlisting}
%\vspace{-0.7cm}
    \caption{Batching in event processing}
    \label{fig:batchingineventprocessing}
\end{figure}

%% figure coreanimation
\begin{figure}[t!]
\begin{minipage}[t]{.20\textwidth}
\begin{lstlisting}
//Worker thread:
//needs to update UI:
obj->need_display = 1
\end{lstlisting}\hfill
\end{minipage}
\noindent\begin{minipage}[t]{.28\textwidth}
\begin{lstlisting}
//Main thread: 
//traverse all CA objects
if(obj->need_display == 1)
  render(obj)
\end{lstlisting}\hfill
\end{minipage}
%\vspace{-0.7cm}
    \caption{CoreAnimation shared flag}
    \label{fig:casharedflag}
\end{figure}

%%figure spinnin flags
\begin{figure}[t]
\begin{minipage}[t]{0.48\textwidth}
\begin{lstlisting}
//NSEvent thread:
CGEventCreateNextEvent() {
  if (sCGEventIsMainThreadSpinning == 0x0)
     if (sCGEventIsDispatchToMainThread == 0x1)
       CFRunLoopTimerCreateWithHandler{
         if (sCGEventIsDispatchToMainThread == 0x1)
           sCGEventIsMainThreadSpinning = 0x1
           CGSConnectionSetSpinning(0x1);
       }
}
\end{lstlisting}
\end{minipage}
\begin{minipage}[t]{0.48\textwidth}
\begin{lstlisting}
//Main thread:
{
	... //pull events from event queue
	Convert1CGEvent(0x1);
	if (sCGEventIsMainThreadSpinning == 0x1){
  		CGSConnectionSetSpinning(0x0);
  		sCGEventIsMainThreadSpinning = 0x0;
  		sCGEventIsDispatchedToMainThread = 0x0;
	}
}
\end{lstlisting}
\end{minipage}
%\vspace{-0.7cm}
    \caption{Spinning Cursor Shared Flags}
    \label{fig:spinningcursorsharedflags}
\end{figure}


As explained in \S\ref{sec:intro}, causal tracing builds a graph to connect
execution segments on behalf of a request that spread across separate threads
and processes. Based on our experience building a causal tracing system
on commercial, closed-source macOS, we believe such graphs are inherently
inaccurate and contain both \emph{over-connections} -- edges that do not really
map to causality) -- and \emph{under-connections} -- missing edges between two
vertices with one causally influencing the other. In this section, we present
several inherently inaccurate patterns we observed and their examples in macOS.

\subsection{Over Connections}

Over connections usually occur when (1) intra-thread boundaries are missing due
to unknown batch processing programming paradigms or (2) superfluous wake-ups
that do not always imply causality.

\paragraph{Dispatch message batching}

While traditional causal tracing assumes the entire execution of a callback
function is on behalf of one request, we found some daemons implement their
service loop inside the callback function and create false dependencies. In the
code snippet below from the \vv{fontd} daemon, function \vv{dispatch\_execute}
is installed as a callback to a work from dispatch queue. It subsequently calls
\vv{dispatch\_mig\_server()} which runs the typical server loop and handles
messages from different apps.

To avoid incorrectly linking many irrelevant processes through such batching
processing patterns, \xxx adopts the aforementioned heuristics to split an
execution segment when it observes that the segment sends out messages to two
distinct processes. Any application or daemon can implement its own server loop
this way, which makes it fundamentally difficult to automatically infer event
handling boundaries.

\paragraph{Batching in event processing}

Message activities inside a system call are assumed to be related traditionally.
However, to presumably save on kernel boundary crossings, WindowServer MacOS
system daemon uses a single system call to receive data and send data for an
unrelated event from different processed in its event loop. This batch processing
artificially makes many events appear dependent.

\paragraph{Mutual exclusion}

In a typical implementation of mutual exclusion, a thread's unlock operation
wakes up a thread waiting in lock. Such a wake-up may be, but is not always,
intended as causality. However, without knowing the developer intent, any
wake-up is typically treated as causality. %by traditional causal tracing tools.

\subsection{Under Connections}

We observe that under connections mostly result from missing data
dependencies.  This pattern is more general than shared-memory flags in ad
hoc synchronization~\cite{xiong2010ad} because it occurs even within a
single thread.

\paragraph{Data dependency in event processing}
The code for Batching in event processing above also illustrates a causal
linkage caused by data dependency in one thread. WindowServer saves the
reply message in variable \vv{\_gOutMsg} inside function
\vv{CGXPostReplyMessage}.  When it calls \vv{CGXRunOneServicePass}, it
sends out \vv{\_gOutMsg} if there is any pending message.

\paragraph{CoreAnimation shared flags}
%As shown in Figure~\ref{fig:casharedflag}, worker thread can set
As shown in the code snippet below, worker thread can set
a field \vv{need\_display} inside a CoreAnimation
object whenever the object needs to be repainted. The main thread iterates over
all animation objects and reads this flag, rendering any such object. This
shared-memory communication creates a dependency between the main thread and the
worker so accesses to these field flags need to be tracked.
%%However, since each object has such a field flag, \xxx cannot afford to monitor
%%each using a watch point register. Instead, it uses instrumentation to modify
%%the CoreAnimation library to trace events on these flags.

\paragraph{Spinning cursor shared flag}
As shown in Figure~\ref{fig:spinningcursorsharedflags},
whenever the system determines that the main thread has hung for a certain
period, and the spinning beach ball should be displayed, a shared-memory flag
is set. Access to the flag is controlled via a lock, i.e. the lock is used for
mutual exclusion, and does not imply a happens before relationship.

