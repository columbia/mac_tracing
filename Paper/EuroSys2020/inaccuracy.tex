\section{Handling Inaccuracies in Causal Tracing}\label{sec:inaccuracy}

As explained in \S\ref{sec:background}, causal tracing builds a graph to connect
execution segments on behalf of a request that spread across separate threads
and processes. Based on our experience building a causal tracing system on
commercial closed-source macOS, we believe such graphs are inherently inaccurate
and contain both \emph{over-connections} which do not really map to
causality, and \emph{under-connections}, missing edges between two vertices
with one causally influencing the other. In this section, we present
several inherently inaccurate patterns (\S\ref{subsec:pattern}) and \xxx's
mitigation methods (\S\ref{subsec:handle}).

\subsection{Inherent Inaccuracies} \label{subsec:pattern}

%figure dispatch message batching
\begin{figure}[t]
\begin{minipage}[t]{.25\textwidth}
\begin{lstlisting}
//worker thread in fontd:
//enqueue a block
block = dispatch_mig_sevice;
dispatch_async(block);
\end{lstlisting}
\end{minipage}\hfill
\begin{minipage}[t]{.21\textwidth}
\begin{lstlisting}
//main thread in fontd:
//dequeue blocks
block = dequeue();
dispatch_execute(block);
\end{lstlisting}
\end{minipage}

\begin{minipage}[t]{0.48\textwidth}
\begin{lstlisting}
//implementation of dipatch_mig_server
dispatch_mig_server()
for(;;) //batch processing
  mach_msg(send_reply, recv_request)
  call_back(recv_request)
  set_reply(send_reply)
\end{lstlisting}
\end{minipage}
%\vspace{-0.3cm}
    \caption{Dispatch message batching}
    \label{fig:dispatchmessagebatching}
\end{figure}

%figure batching in event processing
\begin{figure}[t]
\begin{lstlisting}
//inside a single thread
while() {
  CGXPostReplyMessage(msg) {
  // send _gOutMsg if it hasn't been sent
    push_out_message(_gOutMsg)
    _gOutMsg = msg
    _gOutMessagePending = 1
  }
  CGXRunOneServicePass() {
    if (_gOutMessagePending)
      mach_msg_overwrite(MSG_SEND | MSG_RECV, _gOutMsg)
    else
      mach_msg(MSG_RECV)
    ... // process received message
  }
}
\end{lstlisting}
%\vspace{-0.7cm}
    \caption{Batching in event processing}
    \label{fig:batchingineventprocessing}
\end{figure}

%% figure coreanimation
\begin{figure}[t!]
\begin{minipage}[t]{.20\textwidth}
\begin{lstlisting}
//Worker thread:
//needs to update UI:
obj->need_display = 1
\end{lstlisting}\hfill
\end{minipage}
\noindent\begin{minipage}[t]{.28\textwidth}
\begin{lstlisting}
//Main thread: 
//traverse all CA objects
if(obj->need_display == 1)
  render(obj)
\end{lstlisting}\hfill
\end{minipage}
%\vspace{-0.7cm}
    \caption{CoreAnimation shared flag}
    \label{fig:casharedflag}
\end{figure}

%%figure spinnin flags
\begin{figure}[t]
\begin{minipage}[t]{0.48\textwidth}
\begin{lstlisting}
//NSEvent thread:
CGEventCreateNextEvent() {
  if (sCGEventIsMainThreadSpinning == 0x0)
     if (sCGEventIsDispatchToMainThread == 0x1)
       CFRunLoopTimerCreateWithHandler{
         if (sCGEventIsDispatchToMainThread == 0x1)
           sCGEventIsMainThreadSpinning = 0x1
           CGSConnectionSetSpinning(0x1);
       }
}
\end{lstlisting}
\end{minipage}
\begin{minipage}[t]{0.48\textwidth}
\begin{lstlisting}
//Main thread:
{
	... //pull events from event queue
	Convert1CGEvent(0x1);
	if (sCGEventIsMainThreadSpinning == 0x1){
  		CGSConnectionSetSpinning(0x0);
  		sCGEventIsMainThreadSpinning = 0x0;
  		sCGEventIsDispatchedToMainThread = 0x0;
	}
}
\end{lstlisting}
\end{minipage}
%\vspace{-0.7cm}
    \caption{Spinning Cursor Shared Flags}
    \label{fig:spinningcursorsharedflags}
\end{figure}


\subsubsection{Over Connections} \label{subsec:overconnections}

Over connections usually occur when (1) intra-thread boundaries are missing due
to unknown batch processing programming paradigms or (2) superfluous wake-ups
that do not always imply causality.

\paragraph{Dispatch message batching}

While traditional causal tracing assumes the entire execution of a callback
function is on behalf of one request, we found some daemons implement their
service loop inside the callback function and create false dependencies. In the
code snippet in Figure~\ref{fig:dispatchmessagebatching} from the \vv{fontd}
daemon , function \vv{dispatch\_client\_callout} is installed as a callback to
a work from dispatch queue. It subsequently calls \vv{dispatch\_mig\_server()}
which runs the typical server loop and handles messages from different apps.
Applications or daemons frequently use batching for performance, which
necessarily intermingles handling of irrelevant events, making it
fundamentally difficult to automatically infer event handling boundaries.

\paragraph{Batching in event processing}

Message activities inside a system call are assumed to be related traditionally.
However, to presumably save on kernel boundary crossings, macOS system daemon
\vv{WindowServer} uses a single system call to receive data and send data for
an unrelated event from different processed in its event loop in Figure
~\ref{fig:batchingineventprocessing}. This batch processing artificially makes
many events appear dependent.

\paragraph{Mutual exclusion}

In a typical implementation of mutual exclusion, a thread's unlock operation
wakes up a thread waiting in lock. Such a wake-up may be, but is not always,
intended as causality. However, without knowing the developer intent, any
wake-up is typically treated as causality. 

\subsubsection{Under Connections}\label{subsec:underconnections}

We observe that under connections mostly result from missing data
dependencies.  This pattern is more general than shared-memory flags in ad
hoc synchronization~\cite{xiong2010ad} because it occurs even within a
single thread. 

\paragraph{Data dependency in event processing}
The code in Figure~\ref{fig:batchingineventprocessing} also illustrates a causal
linkage caused by data dependency in one thread. \vv{WindowServer} saves the reply
message in variable \vv{\_gOutMsg} inside function \vv{CGXPostReplyMessage}.
When it calls \vv{CGXRunOneServicePass}, it sends out \vv{\_gOutMsg} if there is
any pending message.

\paragraph{CoreAnimation shared flags}
As shown in the code snippet Figure~\ref{fig:casharedflag}, worker thread can set a
field \vv{need\_display} inside a CoreAnimation object whenever the object needs
to be repainted. The main thread iterates over all animation objects and reads
this flag, rendering any such object. This shared-memory communication creates
a dependency between the main thread and the worker so accesses to these field
flags need to be tracked.

%%However, since each object has such a field flag, \xxx cannot afford to monitor
%%each using a watch point register. Instead, it uses instrumentation to modify
%%the CoreAnimation library to trace events on these flags.

\paragraph{Spinning cursor shared flag}
The very fact that spinning cursors are displayed involves data
flags. Each application contains an \vv{NSEvent} thread which fetches
\vv{CoreGraphics} events from the \vv{WindowServer} daemon, converts them
to \vv{NSApp} events to dispatch to the application's main thread, sets
flag \vv{sCGEventIsDispatchToMainThread} as shown in
Figure~\ref{fig:spinningcursorsharedflags}, and arms a timer to monitor
the processing times of the main thread.  If any \vv{NSApp} event takes
longer than two seconds to process, the main thread will not clear flag
\vv{sCGEventIsDispatchToMainThread}.  When the timer in \vv{NSEvent}
fires, it sets flag \vv{sCGEventIsMainThreadSpinning} and invokes
\vv{WindowServer} to display a spinning cursor.  An interesting fact is
that accesses to these shared-memory flags are controlled via a lock --
the lock is used for mutual exclusion, and does not imply a happens before
relationship.

\subsection{Handling Inaccuracies}

In this section, we discuss how \xxx leverage heuristics to mitigate
over-connections and under-connections.

\subsubsection{Mitigating Over-Connections}\label{subsec:fix-over}

From a high-level, \xxx deals with over-connections by heuristically splitting
an execution segment that appears mixing handling of multiple requests.

Specifically, \xxx splits based on three criteria. First, \xxx recognizes a small
set of well-known batch processing patterns such as \vv{dispatch\_mig\_server()}
in \S\ref{sec:inaccuracy} and splits the batch into individual items. Second,
when a wait operation such as \vv{recv()} blocks, \xxx splits the segment at the
entry of the blocking wait. The rationale is that blocking wait is typically
done at the last during one step of event processing. Third, if a segment
communicates to too many peering processes, \xxx splits the segment when the
set of peers differs. Specifically, for each message, \xxx maintains a set
of two peers including (1) the direct sender or receiver of the message and
(2) the beneficiary of the message (macOS allows a process to send or receive
messages on behalf of a third process). \xxx splits when two consecutive message
operations have non-overlapping peer sets.

\subsubsection{Mitigating Under-Connections}\label{subsec:fix-under}

Causal tracing infers boundaries for the background thread with blocking wait.
It can break an "atomic" task in the worker thread, \eg a thread pauses for IO
resource. \xxx adds weak causal edges to mitigate under-connection. When a weak
edge is encountered during diagnosis, it queries users to decide whether to
follow the weak edge or stop (\S\ref{sec:overview}).

The other primary under-connections are due to data dependencies. Currently \xxx
queries the user to identify the data flags. It is conceivable to leverage
memory protection techniques to infer them automatically, as demonstrated in
previous record-replay work~\cite{king2005debugging, dunlap2008execution}. It
is out of the scope of this paper and we leave it for future work. Currently,
to discover a data flag, the user re-runs the application with \xxx to collect
instruction traces of the concurrent events in both the normal and spinning
cases and detects where the control flow diverges. \xxx exposes register
values for the basic blocks before the divergence and uncovers the address
of the data flag. Once the user identifies a data flag, \xxx traces it using
either binary instrument, such as the \vv{need\_display} flag in CoreAnimation
(\S\ref{sec:inaccuracy}), or with \xxx's watchpoint tool. \xxx add a causal edge
from a write to a data flag to its corresponding read.

