\section{Implementation}\label{sec:implementation}

In this section, we discuss how \xxx collects tracing events from both kernel
and libraries.
%, and user interactions which \xxx leverages to ease diagnosis.

\subsection{Event Tracing}

Current macOS supports a system-wide tracing
infrastructure~\cite{linktotracetool}. It by default stores events in memory and
flushes them to screen or disk periodically. \xxx extends this infrastructure
with a file-backed ring buffer to support larger-scale tests and avoid
exhausting the disk. The file size is set to 2GB by default, approximately 19
million events (about 5 minutes of normal operations). The size can be adjust by
users to accommodate the desired trace history.

Default tracing points in macOS provide too limited information to enact causal
tracing. We therefore instrument both the kernel~\cite{linkofxnusourcecode} (at
the source level) and key libraries (at the binary level) to gather adequate
data. We instrument the kernel with 1,193 lines of code, and binary-instrument
the following libraries: \vv{libsystem\_kernel.dylib}, \vv{libdispatch.dylib},
\vv{libpthread.dylib}, \vv{CoreFoundation}, \vv{CoreGraphics}, \vv{HIToolbox},
\vv{AppKit} and \vv{QuartzCore} in \nlibchanges different places.

%%Next subsection describes our binary instrumentation tool

\subsection{Instrumentation}

Most libraries and applications are closed-source in macOS. To hook their
functions with techniques such as library preloading are not applicable, as
libraries use two-level executable namespace~\cite{twolayernamespace}. Hence, we
implemented a binary instrumentation mechanism that allows users to add tracing
points inside a binary image.

Like Detour~\cite{hunt1999detours}, we use static analysis to decide where the
instrumentation performs and enact it at runtime. The user supplies a sequence
of instructions for \xxx to search the locations of interest in the image, and a
trampoline function which overwrites the sequence of instructions and produces
tracing data with API \vv{kdebug\_trace} from Apple. \xxx generates shell code
with the trampoline function to replace the victimzed instructions. All of
the trampoline functions are grouped into a new image with an initialization
function that triggers the drop-in replacement. \xxx configures the image in the
following steps to finish the instrumentation: (1) it re-exports all symbols
from the original image, so the original code can be called like a shared
library; (2) it renames the original image and applies original name to the
new one to ensure the modifications are properly loaded; (3) it invokes the
initialization function externally with \vv{dispatch\_once} when the library
loads.

\subsection{Tracing Data Flags} \label{subsec:tcp}

As described in (\S\ref{sec:inaccuracy}), under-connection due to the missing
data dependency requires users' interaction. Users specify that reads and writes
to a given variable should be considered data dependencies. Data flag tracing
is possible through \xxx's binary rewriting, but we also provide a simple
command line tool which uses watchpoint registers to record \dataflagwrite and
\dataflagread events. The tool \vv{bp\_watch} takes as input the process ID,
path to the relevant binary image, and the symbol name of the global variable.
Here is a simple example of how a user asks \xxx to trace \vv{\_gOutMsgPending}:

\begin{lstlisting}[language=c++,numbers=none]
bp_watch Pid/of/WindowServer Path/to/CGs _gOutMsgPending
\end{lstlisting}

\noindent \xxx loads the watchpoint handler into the address space of the target
application by hooking it to \vv{CoreFoundation}. The handler invokes Apple's
API \vv{kdebug\_trace} to record the value of the data flag and its operation
type (read or write).

\subsection{Tracing Instructions and Calls}

Users may need to gather more information, such as individual instructions and
call stacks, to come up with and verify a binary patch. \xxx integrates with
\vv{lldb scripts} to capture this information and add it to the corresponding
vertices in the event graph. Our \vv{lldb scripts} gather call stacks at
relevant locations and parameters tainted by user inputs. To reduce the data
collection overhead, only at each beginning of a function call does the script
record a full call stack. While \vv{lldb} steps into functions from apps and
frameworks to record parameters tainted, it steps over and only records the
return value of APIs from low-level libraries (i.e. those with the filename
extension \vv{.dylib}).

The combination of instruction-level tracing and occasional call-stacks offers
more than enough detail to diagnose even the most arcane issues, and in our
experience has been very helpful in multiple steps of an \xxx diagnosis.
