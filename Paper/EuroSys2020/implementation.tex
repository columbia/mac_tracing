\section{Implementation}\label{sec:implementation}

In this section, we discuss how \xxx collects tracing events from both kernel
and libraries.
%, and user interactions which \xxx leverages to ease diagnosis.

\subsection{Event Tracing}

Current macOS supports a system-wide tracing
infrastructure~\cite{linktotracetool}. By default, it stores events in
memory and flushes them to screen or disk periodically. \xxx extends this
infrastructure to support larger-scale tests and avoid exhausting the disk with
a file-backed ring buffer. The log file is constrainted to 2GB by default,
approximately 19 million events (about 5 minutes with normal operations).

The default tracing points in macOS provide too limited information
to enact causal tracing. As a result, we instrumented both the
kernel~\cite{linkofxnusourcecode} (at the source level) and key libraries
(at the binary level), to gather more tracing data. We instrumented the kernel
with 1,193 lines of code, and binary-instrumented the following libraries:
\vv{libsystem\_kernel.dylib}, \vv{libdispatch.dylib}, \vv{libpthread.dylib},
\vv{CoreFoundation}, \vv{CoreGraphics}, \vv{HIToolbox}, \vv{AppKit} and
\vv{QuartzCore} in \nlibchanges different places.

%%Next subsection describes our binary instrumentation tool

\subsection{Instrumentation}

Most libraries as well as many of the applications used day-to-day are
closed-source in macOS. To add tracing points to such code, techniques such as
library preloading to override individual functions are not applicable on macOS,
as libraries use two-level executable namespace~\cite{twolayernamespace}. Hence,
we implemented a binary instrumentation mechanism that allows developers to add
tracing at any location in a binary image.

Like Detour~\cite{hunt1999detours}, we use static analysis to decide which
instrumentation to perform, and then enact this instrumentation at runtime.
Firstly, the user finds a location of interest in the image related to a
specific event by searching a sequence of instructions. Then the user replaces a
call instruction to invoke a trampoline target function, in which we overwrite
the victimized instructions and produce tracing data with API from Apple.
All of the trampoline functions are grouped into a new image, as well as an
initialization function which carries out the drop-in replacement. Then command
line tools from \xxx help to configure the image with the following steps: (1)
re-export all symbols from the original image so that the original code can be
called like an shared library; (2) rename the original image, and use original
name for the new one to ensure the modifications are properly loaded; (3) invoke
the initialization function externally through \texttt{dispatch\_once} during
the loading.

%%One potential issue is that we use 5-byte call instructions with 32-bit
%%displacements to jump from the original library to our new one.  This design
%%requires that the libraries be loaded within +/- 2GB of each other in the
%%64-bit process address space.  However, since we list each original library as
%%a dependency of our new libraries, the system loader will map each new and
%%original library in sequence.  In practice, the libraries ended up very close
%%to one another and we did not see the need to implement a more general
%%long-jump mechanism.

%\subsection{User Interaction}\label{subsec:tcp}
%\para{Tracing Custom Primitives}

\subsection{Tracing Data Flags} \label{subsec:tcp}
%%XXX give a simple command line example of how a user can ask \xxx to trace a
%%data flag
%%XXX say what we do in watch point exception handler (record instruction so
%%can determine read or write, and reg values)

As described in (\S\ref{sec:inaccuracy}), under-connection due to the missing
data dependency requires users' interaction. Users specify that reads and
writes to a given global variable should be considered data dependencies.
Global-variable tracing is possible through \xxx's binary rewriting, but we
also provide a simple command line tool which uses watchpoint registers to
record \dataflagwrite and \dataflagread events. The tool takes as input the
process ID, path to the relevant binary image, and the symbol name of the
global variable. Here is a simple example of how a user would ask \xxx to trace
\vv{\_gOutMsgPending}:

\begin{lstlisting}
./breakpoint_watch Pid/of/WindowServer Path/to/CoreGraphics \
	_gOutMsgPending
\end{lstlisting}

At load time, \xxx hooks the watchpoint break handler in CoreFoundation to
make sure that it is loaded correctly into the address space of our target
application. The handler invokes Apple's event tracing API to record the value
of the data flag and the operation type (read or write).

\subsection{Tracing Instructions and Calls}

Users may need to gather more information, such as individual instructions and
call stacks, to come up with and verify a binary patch. \xxx integrates with
\vv{lldb} to capture this information and add it to the corresponding vertices
in the event graph.

We gather call stacks only at relevant locations, to reduce the data collection
overhead. Our \vv{lldb} scripts go through the instructions of apps and
frameworks step by step to capture the parameters tainted by user inputs. Only
at each beginning of a function call does the script record a full call stack.
We also step over and record the return value of APIs from low-level libraries
(i.e. those with the filename extension \vv{.dylib}).

The combination of instruction-level tracing and occasional call-stacks offers
more than enough detail to diagnose even the most arcane issues, and in our
experience has been very helpful in multiple steps of an \xxx diagnosis.

