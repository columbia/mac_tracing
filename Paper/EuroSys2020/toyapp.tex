\subsection{Inaccuracy study} \label{sec:toystudy}

In the section, we demonstate the inaccuracy of request graph extracted with
causal tracing and the necessary interactions in path slicing with study on
a micro user event handler as shown in Figure~\ref{fig:toyapp}. It accesses the
\vv{Contacts} and search with last name \vv{Smith}. We inject two tracing
data in the source code at the beginning and end of the handler to mimic an
end-to-end transaction.

\begin{figure}[t]
\begin{lstlisting}
//insert debug info begin
ABAddressBook *AB = [ABAddressBook sharedAddressBook];
ABSearchElement *nameIsSmith =
	[ABPerson searchElementForProperty:kABLastNameProperty...];
NSArray *ret = [AB recordsMatchingSearchElement:nameIsSmith]; 
//insert debug info end
\end{lstlisting}
\vspace{-0.5cm}
    \caption{Code snippet: access contacts}
    \label{fig:toyapp}
\end{figure}

%We study the request graph extracted by Panappticon for the handler. Their
%graph is inaccurate for the following reasons: missing data dependency, missing
%boundaries of execution intervals in background threads, batch processing in
%both user threads and system threads, and contraditions to the assumption that
%no unrelated work is processed in a particular task from a queue.

The inherent accuracy of request graph is exposed with umbiguous edges. The
asynchronous calls are always connected to their invocations in Panappticon.
However, some of them are hard to decide. For example, the causality between
timer create and timer cancel introduces unrelated applications, e.g. \vv{mds}
and \vv{trace}. Lack of the causality on the other hand will break the
connections of line 2 and line3. It makes the request graph incomplete in
Panappticon. \xxx mitigates the under-conneciton with the tracking of shared
flags, \vv{AB} and \vv{nameIsSmith}.

%The generated request graph, with the causality between timer create and timer
%cancel, include 13,842 edges and 4,180 nodes. 53 of them have more than 2
%incoming edges.
%of them are encountered in slicing the path from the first to the last line in the code slicing.
