\subsection{Microbenchmark} \label{sec:toystudy}

In the section, we demonstrate the inaccuracy of request graph extracted with
prior causal tracing system. The study is carried on a toy application from
Apple's sample code AppList~\cite{applist}. The main window in the toy app shows all running
applications. \vv{Hide}, \vv{Unhide}, and \vv{Terminate} buttons are beside the
window for users to manipulate a chosen app. 

%The event handler for hide is as shown in
%Figure~\ref{fig:toyapp}, and other buttons are similar.
%\begin{figure}[t]
%\begin{lstlisting}
%NSRunningApplication *selectedApp = [arrayController.selectedObjects objectAtIndex:0];
%[selectedApp hide];
%\end{lstlisting}
%\vspace{-0.5cm}
%    \caption{microbenchmark ui event handler}
%    \label{fig:toyapp}
%\end{figure}

We study the mouse click on the \vv{hide} button and find the ground truth by
manual checking the tracing log.  The user input event is passed from the
WindowServer to the toy app. Inside the toy app, its event handler invokes
appleeventsd and launchserviced, which communicate with the chosen app. Finally
the chosen app hide itself. In the process, daemons such as \vv{cfprefsd},
\vv{distnoted}, \vv{fseventsd}, \vv{syslogd}, \vv{systemstatsd} are invoked for
the system's purpose.

The graph generated with Panappticon contains more information than what we
identified.  It has 20 processes involved. We identify at least 7 of them are
unrelated to the app. 4 of them are connected because of the WindowServer's
batching in event processing as shown in
Figure~\ref{fig:batchingineventprocessing}, and 3 of them are from kernel
task's batching in timer processing. In addition, 4 UI events(mouse click,
mouse move, system defined, and mouse exit event) for the toy app are present
because of batching in runloop blocks~\cite{runloop}. All the UI updates are
batching in a callout function from the runloop. Panappticon treats the path
from the user input to the last UI update as critical path.  Without tracking
the flags in Figure ~\ref{fig:casharedflag}, Panappticon can not seperate them.
Without the Apple developer's efforts, the instrumentation of runloop in \xxx
is not enough to identify all execution boundaried, therefore the graph
generated with our system exposes 39 vertices with multiple incoming edges.

AppInsight is effective in tracking the boundary of the event handler in toy
app, however, it does not track the execution boundaries in daemons. Thus, it 
can not track the activities in \vv{appleeventsd}, \vv{launchserviced} and the chosen
application.
