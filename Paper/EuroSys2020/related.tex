\section{Related Work}
\label{sec:related-work}

While there is currently no system that can help users debug performance
issues in closed-source applications on proprietary macOS, several active
research topics are closely related.

\paragraph{Event tracing.}

Panappticon \cite{zhang2013panappticon} monitors a mobile system and uses the
trace to characterize the user transactions of mobile apps. Although it aims to
track system-wide events and correlate them without developer input, it supports
only two models of communication: work queue and thread pooling. AppInsight
\cite{ravindranath2012appinsight} instruments application to identify the
critical execution path in a user transaction. It supports the event callback
pattern, and does not trace across process or app boundaries.
Magpie~\cite{barham2004using} monitors server applications in Windows with the
goal to model the normal behaviors of a server application in response to a
workload. This model further helps detecting anomalies statistically. Magpie
requires a manual-written event schema for all involved applications to capture
precise request graphs, whereas \xxx has a simple, application-agnostic schema
for system-wide tracing and enables users to provide more application-specific
knowledge on demand.

Aguilela \cite{aguilera2003performance} uses timing analysis to correlate
messages to recover their input-output relations while treating the application
as a black box. XTrace, Pinpoint and \etc ~\cite{fonseca2007x, chen2002pinpoint,
chow2014mystery} trace the path of a request through a system using a unique
identifier attached to each request and stitch traces together with the
identifier. \xxx comes up violation patterns and does not assume the presence of
a unified identifier in closed-source, third-party applications, frameworks, and
libraries.


\paragraph{Performance anomaly detection.} Several systems detect performance
anomalies automatically. \cite{han2012performance, yuan2012conservative}
leverage the user logs and call stacks to identify the performance anomaly.
\cite{cohen2004correlating, saidi2008full, xu2009detecting, du2017deeplog}
apply the machine learning method to identify the unusual event sequence as an
anomaly. \cite{yu2014comprehending} generates the wait and waken graph from
sampled call stacks to study a case of performance anomaly.

These systems are orthogonal to \xxx as \xxx's goal is to diagnose an
already-detected performance anomaly. These systems can help \xxx by detecting
more accurately when a performance issue arises.
