\section{Background} \label{sec:background}

In this section, we illustrate causal tracing and prior work on it. Causal
tracing collects events standing for instrucions executed in CPU and generates
a graphical representation with traced events as vertices. Two events always
following a sequential constraint reflects causality, which is represented
as an edge. The graph helps users understand the complex causal behaviors across
thread/process boundaries and attribute bugs to their root causes. Prior works
have different definitions of vertices, edges, and root causes based on what
events are collected.

AppInsight instruments all the upcalls from the framework to the application.
It traces user input, display update, the begin and end of procedual call, the
invocation of callback function, exception and blocking events in threads. Each
event is reflected as a vertex in the request graph. Therefore, the request
graph connects (1)user input event to (2)the beginning of event handler, which
in turn connects to (3)the beginning of callback in background threads. The
vertices like (2) and (3) will connect to (4)the end of the procedual call
or lead to (5)exception. Besides, they will also connect to (6)blocking for
signal, if the execution requires synchronization. The goal of AppInsight is to
help developers understand the performance bottlenecks with critical paths or
exception paths. It defines the root cause as the state of a function execution,
long blocking or exception in the application.

To be unobtrusive, Panappticon instruments the system to collect low-level
and fine grained events from libraries and kernel, including user input,
display update, asynchronous call and callback, inter-process comminication,
synchronization mechanism, and resource accounting. Every event is a vertex.
Panappticon connects continuous vertices which stem from atomic work in a
thread, e,g, , a worker thread processing one task from a task queue, into
an execution interval. Two execution intervals are connected if the earlier
interval triggers the latter one. For example, a user input triggering an
enqueue message in the same thread reflects as an execution interval, where
two vertices are connected with a temperal ordering edge. In another thread,
dequeuing the message and submiting an asynchronous task generate another
execution interval. The two intervals are connected with a causal edge. With the
resouces analysis in every user transaction, from user input to display update,
Panapption pinpoints root causes from the application design flaw, harmful
interaction, to underpowered hardware.
