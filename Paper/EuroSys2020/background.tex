\section{Background} \label{sec:background}

In this section, we illustrate the causal tracing and prior work on it.

Causal tracing generates a graphical representations of execution trace, with
traced events as vertices and their causal relationships as edges. The graph
helps users to understand the complex causal behaviors across thread/process
boundaries and attribute bugs to their root causes.

The types of events traced by the system determins the definitions of vertices
and edges, as well as the root causes. AppInsight instruments all the upcalls
from the framework to the application. It traces user input, display update,
the begin and end of procedual call, the invocation of callback function,
exception and blocking events in threads. Each event is reflected as a vertex
in the request graph. Therefore, the request graph connects (1)user input
event to (2)the beginning of event handler, which in turn connects to (3)the
beginning of callback in background threads. The vertices like (2) and (3) will
connect to (4)the end of the procedual call or lead to (5)exception. Besides,
they will also connect to (6)blocking for signal, if the execution requires
synchronization. The goal of AppInsight is to help developers understand the
performance bottlenecks with critical paths or exception paths. It defines the
root cause as the state of a function execution, long blocking or exception in
the application.

To be unobtrusive, Panappticon instruments the system to collect low-level
and fine grained event from libraries and kernel, including user input,
display update, asynchronous call and callback, inter-process comminication,
synchronization mechanism, and resource accounting. Every event is a vertex.
Panappticon infers boundaries for execution intervals which represent atomic
work in a thread, e.g,, a worker thread processing one task from a task queue.
All vertices inside the intervals are connected with temporal ordering edge,
while two intervals connected with causal edge if the earlier interval triggers
the latter one. For example, a user input triggering an enqueue message reflects
as an execution interval, where two vertices are connected with a temperal
ordering edge. Another thread dequeues the message and submit an asynchronous
task generates another execution interval. These two intervals are connected
with a causal relationship. The root causes Panapption pinpoints from the graph
ranges from the application design flaw, harmful interaction, to underpowered
hardware.
