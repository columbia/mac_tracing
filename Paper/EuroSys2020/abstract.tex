Prior systems have used causal tracing, a powerful technique that traces low-level
events and builds request graphs, to diagnose performance issues. However, they
all assume that accurate causality can be inferred from low-level tracing with
supported communication patterns, or can be obtained from a developer's schema upfront
for all involved components.  Unfortunately, based on our study and experience
building a causal tracing system on macOS, we show it is difficult, if not
impossible, to get accurate request graphs.
We present \xxx, a practical system for debugging performance issues in modern
desktop applications despite the inaccuracy of causal tracing. \xxx lets a user
inspect current diagnostics and provide domain knowledge on demand to counter
the inherent inaccuracy of causal tracing. We evaluated \xxx on \nbug
real-world, open spinning-cursor issues in popular macOS applications. The
root causes of these issues were largely previously unknown.  Our results
show that \xxx effectively helps locate all root causes of the issues
and incurs only \cpuoverhead CPU overhead for its system-wide tracing.
