\subsubsection{Long Running}

In this section, we discuss the cases where the \spinningnode is busy on the
CPU. Most text editing apps fall into this bug category. We studied bugs on
TeXstudio, TextEdit, Microsoft Word, SublimeText, TextMate and CotEditor,
to reveal the root causes.

\paragraph{5-TeXStudio}

\vv{TeXstudio}~\cite{TeXStudio} is an integrated writing environment for
creating LaTeX documents. We noticed a user reported spinning cursor on the
modification of bibliography (bib) file. Although the issue was closed by the
developer for insufficient information, we reproduced it with our bib file
around 500 items in a \vv{TeXStudio} tab. When we touch the file in
other editors like vim, a spinning cursor appears in \vv{TeXStudio}'s window.

\xxx recognizes the \spinningnode belongs to the category of LongRunning. The
causal path sliced from the \spinningnode by \xxx reveals the long-running
function is a callback from daemon \vv{fseventd}, and the long processing segment
is busy to CalculateGrowingBlockSize, even without modifications to the file.
The advantage of \xxx over other debugging tools is it narrows down the root
cause with the inter-processes execution path.

\paragraph{6-TextEdit}

\vv{TextEdit} is a simple word processing and text editing tool from Apple,
which often hangs on editing large files.  When \xxx is used to diagnose
this issue, the heuristics of choosing the most recent edge is powerful enough 
to get the causal path.

The event graph reveals a communication pattern where a kernel thread is woken
from I/O by another kernel thread; the woken kernel thread processed
a timer callback function armed by \vv{TextEdit} and finally woke a \vv{TextEdit}
thread. In the pattern, the kernel thread has two incoming edges. One is from
another kernel thread's IO completion, and the other is from \vv{TextEdit}'s timer
creation. It not hard to reveal the high level semantics. \vv{TextEdit} arms a timer
for IO work, and the kernel thread gets the notification for the completion of IO
and processes the timer callback. The success of heuristics is not surprising
because the most recent edge in the vertex reflects the purpose of the execution
segment. Although the heuristics works for the kind of simple application, it is
not general enough to succeed for all patterns.

\paragraph{7-MSWord}

\vv{Microsoft Word} is a large and complex piece of software. \xxx can analyze
the event graph, but it identifies multiple possible root causes: the length of
path interactively sliced from the \spinningnode is 67, while the slicing with
heuristics of choosing the most recent edge generates a path of 136 vertices.

We compared the paths and find they diverge from the third vertex backward
from \spinningnode. In the vertex, a \vv{Microsoft Word} thread is
woken by another \vv{Microsoft Word} thread, and launches a service
\vv{NSServiceControllerCopyServiceDictionarie}. The woken thread sends a message
to \vv{launchd} to register the new service and waits for a reply message.
With the most recent edge heuristics in automatic slicing, \xxx chooses the
reply from \vv{launchd} as its predecessor. However, a user can more accurately
identify that the execution segment is on behalf of another \vv{Microsoft Word}
thread. We rely on user interaction in this case to find the true root cause,
since \xxx is likely identifies all possibilities without priority.

\paragraph{Other Editing Apps}

Select, copy, paste, delete, insert and save are common operations for text
editing. However, these operations on a large context usually trigger spinning
cursors. In Table~\ref{table:texteditapps}, we list the root causes reported by
\xxx, including the most costly functions in the event handler, and the user
input event (derived from path slicing).

\begin{table}[tb]
\vspace{-0.2cm}
\footnotesize
\centering
  \begin{tabularx}{\columnwidth}{l|l|l}
  \hline
                  &                     &\\
  \textbf{BUG-ID} & \textbf{costly API} &UI\\
  \hline
  \hline
  8-Notes         & \begin{tabular}{@{}l@{}}
  					\vv{1)NSDetectScrollDevicesThe}\\
					\vv{\xspace -nInvokeOnMainQueue}\\
					\end{tabular}
   		          & \begin{tabular}{@{}l@{}}
				  	\vv{system defined event}
					\end{tabular}
				  \\
  \hline
  9-SlText   & \begin{tabular}{@{}l@{}} 
					\vv{1)px\_copy\_to\_clipboard}\\
  					\vv{2)\_\_CFToUTF8Len}\\
  					\end{tabular}
				  & \vv{key c}
				  \\
  \hline
  10-TextMate      & \begin{tabular}{@{}l@{}}
  					\vv{1)-[OakTextView paste:]}\\
					\vv{2)CFAttributedStringSet}\\
					\vv{3)TASCIIEncoder::Encode}\\
  					\end{tabular}
				  & \vv{key v}
				  \\
  \hline
  11-CotEditor    & \begin{tabular}{@{}l@{}}
  					\vv{1)CFStorageGetValueAtIndex}\\
					\vv{2)-[NSBigMutableString}\\
					\vv{\xspace characterAtIndex:]}\\
  					\end{tabular}
   		          & \begin{tabular}{@{}l@{}}
				  	\vv{key Return}
  					\end{tabular}

				  \\
  \hline
  \end{tabularx}
  \caption{Root cause of spinning cursor in editing Apps}
  \label{table:texteditapps}
  \vspace{-0.2cm}
\end{table}

