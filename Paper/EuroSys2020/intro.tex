\section{Introduction} \label{sec:intro}

Today's web and desktop applications are predominantly parallel or distributed,
making performance issues in them extremely difficult to diagnose because the
handling of an external request is often spread across many threads, processes,
and asynchronous contexts instead of in one sequential execution
segment~\cite{harter2012file}. To manually reconstruct this graph of execution
segments for debugging, developers have to sift through a massive amount of log
entries and potentially code of related application
components~\cite{chen2002pinpoint, zhao2016non, xu2009detecting,
nagaraj2012structured, yuan2012conservative}. More often than not, developers
give up and resort to guessing the root cause, producing ``fixes'' that
sometimes make the matter worse. For instance, a bug in the Chrome browser
engine causes a spinning (busy) cursor in macOS when a user switches the input
method~\cite{chromiumbugreport}. It was first reported in 2012, and developers
attempted to add timeouts to work around the issue. Unfortunately, the bug has
remained open for seven years and the timeouts obscured diagnosis further.

Prior work proposed what we call \emph{Causal tracing}, a powerful technique to
construct request graphs automatically~\cite{reynolds2006pip, fonseca2007x,
benjamin2010dapper, zhang2013panappticon, ravindranath2012appinsight}.  It does
so by inferring (1) the beginning and ending boundaries of the execution
segments (vertices in the graph) involved in handling a request; and (2) the
causality between the segments (edges)---how a segment causes others to do
additional handling of the request.  Compared to debuggers such as \spindump
that capture only the current system state, causal tracing is quite effective
at aiding developers to understand complex causal behaviors and pinpoint
real-world performance issues.

Prior causal tracing systems all assume certain programming idioms to automate
inference. For instance, if a segment sends a message, signals a condition
variable, or posts a task to a work queue, it wakes up additional executions,
and prior systems assume that wake-ups reflect causality. Similarly, they
assume that the execution segment from the beginning of a callback invocation
to the end is entirely for handling the request that causes the callback to be
installed.  Unfortunately, based on our own study and experience of building a
causal tracing system for the commercial operating system, macOS, we found that
modern applications frequently violate these assumptions. Hence, the request
graphs computed by causal tracing are inaccurate in several ways.

First, an inferred segment may be larger than the actual event handling segment
due to batch processing. Specifically, for performance, an application or its
underlying frameworks may bundle together work on behalf of multiple requests
with no clear distinguishing boundaries. For instance, WindowServer in macOS
sends a reply for a previous request and receives a message for the current
request using one system call \vv{mach\_msg\_overwrite\_trap}, presumably to
reduce user-kernel crossings.

Second, the graphs may be missing numerous causal edges. For instance, consider
data dependencies in which the code sets a flag (\eg, ``\vv{need\_display} = 1''
in macOS animation rendering) and later queries the flag to process a request
further. This pattern is broader than ad hoc synchronization~\cite{xiong2010ad}
because data dependency occurs even within a single thread (such as the buffer
holding the reply in the preceding WindowServer example). Although the number of
these flags may be small, they often express critical causality, and not tracing
them would lead to many missing edges in the request graph. However, without
knowing where the flags reside in memory, a tool would have to trace all memory
operations, incurring prohibitive overhead and adding many superfluous edges to
the request graph.

Third, in any case, many inferred edges may be superfluous because wake-ups do
not necessarily reflect causality. Consider an \vv{unlock()} operation waking up
an thread waiting in \vv{lock()}. This wake-up may be just a happens-stance and
the developer intent is only mutual exclusion. However, the actual semantics of
the code may also enforce a causal order between the two operations.

We believe that, without fully understanding of application semantics, request
graphs computed by causal tracing are \emph{inherently} inaccurate and both
over- and under-approximate reality. Although developer annotations can help
improve precision~\cite{reynolds2006pip, fonseca2007x}, modern applications use
more and more third-party libraries, whose source code is not available. 
%In the
%case of tech-savvy users debugging performance issues of an application for
%daily use, the application's code is often not available. 
Given the frequent use of custom synchronizations, work queues, and data flags
in modern applications, it is hopeless to count on manual annotations to ensure
accurate capture of request graphs.

In this work, we present \xxx, a dramatically different approach in the design
space of causal tracing. It is desgined for tech-savvy users who are intersted
in compiling useful bug reports for daily use applications whose codes are
oftern not available.  Thus, \xxx enables users to provide necessary schematic
information on demand, as opposed to full manual schema upfront for all
involved applications and daemons~\cite{barham2004using, reynolds2006pip,
fonseca2007x}. \xxx does not construct a per-request causal graph which
requires extremely accurate causal tracing. Instead, it calculates an event
graph for a duration of the system execution. To aid diagnosis, it searches
root cause via both true causality edges as well as weak ones.  Specifically,
it keeps humans in the loop, as a debugger should rightly do.  \xxx queries
users a judicially few times to (1) resolve a few inaccurate edges that
represent false dependencies and (2) identify potential dependency due to data
flags. \xxx is the first to apply interactive steps on causal tracing system
for debugging performance issues in modern desktop applications, despite the
inaccuracy of causal tracing.

We implement \xxx in macOS, a widely used commercial operating system macOS,
which is closed-source on its frameworks and many applications.  This
closed-srouce environment therefore provides a true test of \xxx. We address
multiple nuances of macOS that complicate causal tracing, and built a
system-wide, low-overhead, always-on tracer. \xxx enables users to optionally
increase the granularity of tracing (\eg, logging call stacks and instruction
streams) by integrating with existing debuggers such as \vv{lldb}.

We evaluate \xxx on \nbug real-world, open spinning-cursor issues in widely
used applications such as Chromium browser engine and macOS System Preferences,
Installer, and Notes. The root causes of all \nbug issues were previously
unknown to us and, to a large extent, the public. Our results show that \xxx is
effective: it helps us non-developers of the applications find all root causes
of the issues, including the Chromium issue that remained open for seven years.
\xxx mostly needs only less than 3 user queries per issue but they are crucial
in aiding diagnosis. \xxx is also fast: its systems-wide tracing incurs only
\cpuoverhead CPU overhead overall.

%% consider adding comparison to prior approaches
%% Our techniques effectively removed false and added missing edges in the
%% event graph.  Even with our techniques, the resultant graph remain too
%% inaccurate for traditional causal tracing, but, fortunately, an average
%% of XXX user queries suffice to locate the root causes accurately.

We make the following contributions: 
\begin{enumerate}

\item We demonstrate conceptual realization that causal tracing is inherently
inaccurate, and introduce interactive approach in the design space of causal
tracing.

\item We build \xxx, performing system-wide tracing in macOS with little
overhead, and handle several macOS trickeries that complicate causal tracing.

\item We use \xxx to diagnose real-world spinning cursors and find root causes for
performance issues that have remained open for several years.

\end{enumerate}

%This paper is organized as follows.
%In Section~\ref{sec:background}, we introduce the causal tracing technology used in prior works. 
%In Section~\ref{sec:overview}, we present
%an overview of using \xxx and a Chromium example. 
%In Section ~\ref{sec:inaccuracy}, we report inherently inaccuracy
%patterns observed in macOS.  Section~\ref{sec:graphcomputing} describes our
%event graph from causal tracing, and Section~\ref{sec:implementation} describes
%our tracing implementation and tools for user interaction. Section
%~\ref{sec:method} demonstrates the methodology. In Section~\ref{sec:casestudy}
%we present other case studies, and Section~\ref{sec:evaluation} contains
%performance evaluation.  We summarize related work in
%Section~\ref{sec:related-work}, and end with conclusion in Section
%~\ref{sec:conclusion}.
