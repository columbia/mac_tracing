\section{Introduction} \label{sec:intro}

Today's web and desktop applications are predominantly parallel or
distributed, making performance issues in them extremely difficult to
diagnose because the handling of an external request is often spread across
many threads, processes, and asynchronous contexts instead of in one
sequential execution segment~\cite{harter2012file}. To manually reconstruct
a graph of execution segments for debugging, developers have to sift
through massive amounts of log entries and potentially code of related
application components~\cite{chen2002pinpoint, zhao2016non, xu2009detecting,
nagaraj2012structured, yuan2012conservative}. More often than not, developers
give up and resort to guessing the root cause, producing ``fixes'' that
sometimes make the matter worse. For instance, a bug in the Chrome browser
engine causing a spinning cursor in macOS when a user switches the input
method~\cite{chromiumbugreport}, was first reported in 2012. Developers
attempted to add timeouts to work around the issue. Unfortunately, the bug has
remained open for seven years and the timeouts obscured diagnosis further.

Prior work proposed what we call \emph{causal tracing}, a powerful technique
to construct request graphs automatically~\cite{reynolds2006pip, fonseca2007x,
benjamin2010dapper, zhang2013panappticon, ravindranath2012appinsight}. It
does so by inferring (1) the beginning and ending boundaries of the execution
segments (vertices in the graph) involved in handling a request, and (2) the
causality between the segments (edges)---how a segment causes others to do
additional handling of the request. Compared to debuggers such as \spindump that
capture only the current system state, causal tracing is effective at helping
developers understand complex causal behaviors and pinpoint the root causes
for real-world performance issues.

Prior causal tracing systems all assume certain programming idioms to automate
inference. For instance, if a segment sends a message, signals a condition
variable, or posts a task to a work queue, it wakes up additional execution
segments. Prior systems assume that wake-ups reflect causality. Similarly, they
assume that the execution segment, from the beginning of a callback invocation
to the end, is entirely for handling related work in a request. Unfortunately,
based on our study and experience of building a causal tracing system for macOS,
we find that modern applications violate these assumptions. Hence, the request
graphs computed by causal tracing are inaccurate in several ways.

First, an inferred segment can be larger than the actual event handling segment
due to batch processing. Specifically, for performance, an application or its
underlying frameworks may bundle together work on behalf of multiple requests
with no clear distinguishing boundaries. For instance, WindowServer in macOS
sends a reply for a previous request and receives a message for the current
request using one system call \vv{mach\_msg\_overwrite\_trap}, presumably to
reduce user-kernel crossings. Second, the graphs can miss numerous causal
edges. For instance, consider data dependencies in which the code sets a flag
(\eg, ``\vv{need\_display} = 1'' in macOS animation rendering) and later
queries the flag to process a request further. This pattern is broader than
ad hoc synchronization~\cite{xiong2010ad} because data dependency occurs
even within a single thread (such as the buffer holding the reply in the
preceding WindowServer example). Although the number of these flags may be
small, they often express critical causality, and not tracing them would lead
to many missing edges in the request graph. Third, inferred edges can be
superfluous because wake-ups do not necessarily reflect
causality. Consider an \vv{unlock()} operation waking up an thread waiting
in \vv{lock()}. This wake-up can just be happenstance and the developer's
intent is mutual exclusion. However, the two operations can also enforce a
causal order.

We believe that, without fully understanding of application semantics, request
graphs computed by causal tracing are \emph{inherently} inaccurate and both
over- and under-approximate reality. Although developer annotations can help
improve accuracy~\cite{reynolds2006pip, fonseca2007x}, modern applications use
more and more third-party libraries, whose source code is not available. Given
the frequent use of custom synchronizations, work queues, and data flags in
modern applications, it is hopeless to count on manual annotations to ensure
accurate capture of request graphs of the entire system.

In this work, we present \xxx, a dramatically different approach in the design
space of causal tracing. As opposed to full manual schema upfront for
all involved applications and daemons~\cite{barham2004using, reynolds2006pip,
fonseca2007x}, \xxx calculates an event graph for a duration of the system
execution with both true causal edges and weak ones, and enables users to
provide necessary schematic information on demand in diagnosis,
as a debugger should rightly do. Specifically, \xxx queries users a
judicially few times to (1) resolve a few inaccurate edges that represent false
dependencies and (2) identify potential dependency due to data flags.

\xxx targets two groups of users in mind: (1) technical users who wanted
to compile informative bug reports for the applications crucial to them
and (2) developers of the application or system who wanted to either
pinpoint the issue in their component or identify which component is
responsible so they can assign the issue to the corresponding developers.
Today's applications are often built on top of many third-party,
closed-source libraries and frameworks such as Cocoa~\cite{cocoa}, and the
operating system itself is often closed-source.  Besides technical users,
even developers of the application or system may not have access to the
code of some critical components in the software stack.  We thus
explicitly built \xxx to work in closed-source settings.

We implement \xxx in macOS whose frameworks and applications are mostly
closed-source. This closed-srouce environment therefore provides a true
test of \xxx. We address multiple nuances of macOS that complicate causal
tracing, and build a system-wide, low-overhead, always-on tracer. \xxx
enables users to optionally increase the granularity of tracing (\eg,
logging call stacks and instruction streams) by integrating with existing
debuggers such as \vv{lldb}.

We evaluate \xxx on \nbug real-world, open spinning-cursor issues in widely
used applications such as Chromium browser engine and macOS System Preferences,
Installer, and Notes. The root causes of all \nbug issues were previously
unknown to us and, to a large extent, the public. Our results show that \xxx is
effective: it helps us non-developers of the applications find all root causes
of the issues, including the Chromium issue that remained open for seven years.
\xxx mostly needs only less than 3 user queries per issue but they are crucial
in aiding diagnosis. \xxx is also lightweight: its systems-wide tracing incurs
only \cpuoverhead CPU overhead overall.

%% consider adding comparison to prior approaches
%% Our techniques effectively removed false and added missing edges in the
%% event graph.  Even with our techniques, the resultant graph remain too
%% inaccurate for traditional causal tracing, but, fortunately, an average
%% of XXX user queries suffice to locate the root causes accurately.

This paper makes three main contributions: (1) our conceptual realization
that causal tracing is inherently inaccurate, and our approach of
interactive causal tracing that explores a novel point in the design
space; (2) our system \xxx that performs system-wide causal tracing with
little overhead, embraces inaccuracy in the event graphs, and diagnoses
issues using human-in-the-loop algorithms; and (3) our results using \xxx
to uncover the root causes of 11 real-world macOS spinning cursors, many
of which have remained open for years.

This paper is organized as follows. Section~\ref{sec:background}
introduces causal tracing and prior work. Section~\ref{sec:overview},
presents an overview of \xxx including its design choices, work flow,
diagnosis algorithm, and limitations.  Section ~\ref{sec:inaccuracy}
reports inherently inaccuracy patterns observed in macOS and \xxx's
mitigation. Section~\ref{sec:implementation} describes several
implementation details of \xxx.  Section~\ref{sec:eval} explains our
evaluation methodology and results.  Section~\ref{sec:related-work}
summarizes related work.  Section~\ref{sec:conclusion} concludes.
