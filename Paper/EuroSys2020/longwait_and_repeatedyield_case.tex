\subsubsection{Long Wait and Repeated Yield}

In this section, we discuss the cases where the \spinningnode is blocking on
wait event or yielding loop, corresponding to LongWait and RepeatedYield.
These root causes mostly can be verified by themselves, as they are
manifested by comparing to normal scenarios.

\para{2-SystemPref}

\vv{System Preferences} provides a central location in macOS to
customize system settings, \eg additional monitors configuration.
\vv{DisableMonitor}~\cite{disablemonitor} provides more functionality,
enable/disable monitors online. We catch the spinning cursor when we disable
an external monitor and rearrange windows in the \vv{Display} panel.

The log collected by \xxx contains 2 cases: 1) a baseline scenario where
the displays are rearranged with the enabled external monitor, and 2) a
spinning scenario as we described above. The \spinningnode in the main
thread is dominated by system calls, \vv{mach\_msg} and \vv{thread\_switch},
which falls into the category of Repeated Yield. We discovered data flags,
``\vv{\_gCGWillReconfigureSeen}'' and ``\vv{\_gCGDidReconfigureSeen}'', which
signify the configuration status and break the thread-yield loop. \xxx reveals
that the main thread of \vv{System Preferences}, in the baseline scenario, sets
the flags after receiving specific datagrams from \vv{WindowServer}. Conversely,
the setting of ``\vv{\_gCGDidReconfigureSeen}'' is missing in the spinning case,
and the main thread thus repeatedly sent messages to \vv{WindowServer} for
datagram.

In conclusion, we discovered that the bug is inherent in the design of the
\vv{CoreGraphics} framework, and would have to be fixed by Apple. We also
verified this diagnosis by creating a dynamic binary patch to fix the deadlock.
The patched library makes \vv{DisableMonitor} work correctly, while preserving
correct behavior for other applications.

\para{3-SequelPro}

\vv{Sequel Pro}~\cite{SequelPro} is a fast, easy-to-use Mac database management
application for \vv{MySQL}. It allows a user to connect to a database with sockets or ssh.
We experienced the non-responsiveness of Sequel Pro when it lost the network
connection and tried reconnections.

The tracing log contains two cases: 1) a quick network connection during login,
and 2) Sequel Pro lost the connection for a while. Although \xxx identified the
\spinningnode and \similarnode with ease, the backward slicing from \similarnode
encountered multiple incoming edges, including one from a kernel thread.
The kernel thread processes tasks from different applications in a batch.
Interaction in the path slicing is helpful to reduce the noise. With comparison
to the causal path in normal case, \xxx reveals the main thread is blocking on a
kernel thread, which waits for a ssh thread. Thus we conclude the root cause is
the main thread blocking for network IO.

\para{4-Installer}

\vv{Installer}~\cite{Installer} is an application that extracts and installs
files out of \vv{.pkg} packages in macOS. When \vv{Installer} pops up a window
for privileged permission during the installation of \vv{jdk-7u80-macosx-x64},
moving the cursor out of the popup window triggers a spinning cursor.

\xxx successfully records the baseline scenario with the following operations.
We first type in password in the pop-up window and then click the back button to
reproduce the spinning case by moving cursor. Examining the \spinningnode and
its \similarnode, \xxx reveals the main thread is waiting for \vv{authd} which
blocks on a semaphore. Further diagnosis on \vv{authd} reveals the root cause
is the \vv{SecurityAgent}. It processes user input and wakes up \vv{authd} in
normal case, but fails to notify \vv{authd} in spinning case. We would suggest
an operation for \vv{SecurityAgent} to inform \vv{authd} when the user input box
loses cursor, so that the main thread in \vv{Installer} receives proper signal
from \vv{authd}.

%We also discovered a communication pattern in \vv{Installer} underpinning the
%crucial of interactive debugging. It involves four vertices in four threads,
%vertex $Vertex_{main}$ in the main thread, and $Vertex_1$ to $Vertex_3$ in
%three worker threads. First, the main thread wakes up three worker threads.
%Then one worker thread is scheduled to run. At its end, another worker thread,
%which waits on mutex lock, is woken in $Vertex_2$, which in turn wakes up the
%next worker thread in $Vertex_3$. While \xxx is slicing backward, $Vertex_3$
%has two incoming edges: one is from $Vertex_{main}$, and the other one is from
%$Vertex_2$. Since users can peek the edges before making decision, they are
%likely to figure out that the three worker threads contend with mutex lock, and
%all of them are successors of $Vertex_{main}$.
