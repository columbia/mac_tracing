\subsection{Methodology} \label{sec:methodology}

%%In this section, we first describe how the performance issues are selected for
%%study. Then we compare \xxx to its closest prior system, and describe how we
%%evaluate the manual efforts needed in diagnosis.

Prior work depends on the request graph per transaction to identify bottleneck
and speculate possible causes. Panappticon for Android is the closest causal
tracing system to ours in desgin. The definition of event types and causality in
their paper is a subset of our \xxx. Thus we first study Panappticon's request
graph with micro benchmarks. Extracting a transaction from user input to a
corresponding display fails with Panappticon due to the batching processing in
runloop~\cite{runloop}. It is not uncommon to see two and more \vv{NSApp} events
are connected with an \vv{CA::Transaction::commit()} function. We therefore
carry out the inaccuracy study in ~\S\ref{sec:toystudy} with a Non-GUI toy
program.

For real-world case study, we collect performance issues of popular
applications, because they likely represent the bugs attractive to tech-savvy
users. Among the 26 bugs from the github reports, we reproduced 3 of them
successfully. Others are failed either due to the version capacity in ElCapitan
or insufficient information in the bug report. 8 performance issues are
collected from the daily use applications in the auther's laptop. As a result,
we study 11 reproducible cases in this paper.

%Also explain what metrics/outcomes we look for. "whether these tools enable a
%developer to identify root cause of a given performance issues" or "quantify the
%manual effort needed to ..."

Now we describe how we measure the effects of hueristics \xxx uses to
mitigate the graph inaccuracy, and manual efforts required in the diagnosis.
First, we enable tracing component in the background and reproduce the
performance isssues.  When \xxx constructs event graphs with the trace log, we
measure the number of vertices introduced and merged by heuristics.  Then we
run \xxx diagnosis algorithm on the event graph with a human in the loop. We
count the times when multiple incoming causal edges or weak edges are presents
and \xxx requires users' guidance in path slicing. Users can query the event
graph for assistance and make decisions with domain knowledge. In the worst
case that users make a wrong decision, before reaching the end of path slicing,
\xxx allows them to relocate the path to a particular vertex. Specifically, the
number presented in the following section reflects our experience of debugging.
It does not include the situation of relocating vertex in path slicing, which
we did not encounter in our case studies.
