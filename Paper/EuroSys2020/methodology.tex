\subsection{Methodology} \label{sec:methodology}

In this section, we first describe how the performance issues are selected for
study. Then we compare \xxx to its closest prior system, and describe how we
evaluate the manual efforts needed in diagnosis.

We collect performance issues in the deployed machine in daily use and search
from github bug reports with the keywords "spinning beachball". Only bug
reports from popular applications are selected since they likely represent the
bugs attractive to tech-savvy users. The performance issues that we fail to
reproduce with a week, due to the version capacity or reproducibility, are
discarded. As a result, we select 11 reproducible cases in macOS Elcapitan for
study in this paper.

%Also explain what metrics/outcomes we look for. "whether these tools enable a
%developer to identify root cause of a given performance issues" or "quantify the
%manual effort needed to ..."

Panappticon is the closest system to \xxx in desgin. We implement it in macOS
as a baseline with event types and causal relationships defined in its paper.
It depends on the resource usage analysis for each transaction to identify
the performance bottleneck and speculate the possible causes. Further manual
efforts are required to investigate the root causes. Although it is effective
in identifying performance issue due to resource contention in Android, our
implementation in macOS fails in the extraction of end-to-end transactions
for the following reasons: untracked data dependency, missing boundaries of
execution intervals in background threads, batch processing in both user threads
and system threads, and contraditions to the assumption that no unrelated work
is processed in a particular task from a queue. It is common to find two and
more user input events, even events from unrelated applications are collapsed
into a request graph.

Below we demonstrate how we measure the effects of hueristics \xxx uses to
mitigate the graph inaccuracy, and manual efforts required in the diagnosis.
First, we enable tracing component in the background and reproduce the
performance isssues.  When \xxx constructs event graphs with the trace log, we
measure the number of vertices introduced and merged by heuristics.  Then we
run \xxx diagnosis algorithm on the event graph with human in the loop. We
count the times when multiple incoming causal edges are presents and \xxx
requires users' guidance in path slicing. Users can query the event graph for
assistance and make decisions with domain knowledge. In the worst case that
users make a wrong decision, before reaching the end of path slicing, \xxx
allows them to relocate the path to a particular vertex. Specifically, the
number presented in the following section reflects our experience of debugging.
It does not include the situation of relocating vertex in path slicing, which
we did not encounter in our case studies.
