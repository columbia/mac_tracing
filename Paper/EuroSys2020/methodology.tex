\subsection{Methodology} \label{sec:methodology}

%%In this section, we first describe how the performance issues are selected for
%%study. Then we compare \xxx to its closest prior system, and describe how we
%%evaluate the manual efforts needed in diagnosis.

Prior work relies on the request graph per transaction to identify bottlenecks
and speculate about possible causes. One of them, Panappticon for Android, is
closest to ours in design. Their traced events and causality are a subset
of ours. We demonstrate the inherent inaccuracy of the request graph extracted
with Panappticon's causal tracing, we therefore carry out the study in
~\S\ref{sec:toystudy} with microbenchmarks. This section also evaluates the methodology of
another tool, AppInsight.

In ~\S\ref{sec:casestudy}, we carry out real-world case studies with collected
performance issues of popular applications, which likely represent the
bugs attractive to tech-savvy users. Among the 26 bugs we examined from
Github reports, most of them we are not able to reproduce due to our system version
or insufficient report details. We thus decide to focus on 3 bugs which we
successfully reproduced. 8 additional performance issues are collected from
daily-used applications on the auther's laptop. As a result, we study 11
reproducible cases.
%Also explain what metrics/outcomes we look for. "whether these tools enable a
%developer to identify root cause of a given performance issues" or "quantify the
%manual effort needed to ..."
Next, we describe how we measure the heuristics \xxx uses to mitigate graph
inaccuracy, and manual effort required in the diagnosis. We first enable tracing
component in our laptop and reproduce the 11 performance isssues. The tracing
data are collected to construct event graphs, which contain vertices with
multiple incoming edges or weak edges. We run the \xxx diagnosis algorithm, and
count how many times we encounter those vertices. In the worst case that we make
a wrong decision, before reaching the end of path slicing, \xxx allows us to
relocate the path to a nearby vertex.
