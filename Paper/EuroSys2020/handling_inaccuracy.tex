\subsection{Handling Inaccuracies}

In this section, 
we discuss how \xxx mitigates
over-connections (\S\ref{subsec:fix-over}) and under-connections
(\S\ref{subsec:fix-under}) in them.

\subsubsection{Mitigating Over-Connections}\label{subsec:fix-over}

From a high-level, \xxx deals with over-connections by heuristically
splitting an execution segment that appears mixing handling of multiple
requests. It adds weak causal edges between the split segments in case
the splitting was incorrect. When a weak edge is encountered during
diagnosis, it queries users to decide whether to follow the weak edge
or stop (\S\ref{sec:overview}).

Specifically, \xxx splits based three criteria.  First, \xxx recognizes a
small set of well-known batch processing patterns such as
\vv{dispatch\_mig\_server()} in \S\ref{sec:inaccuracy} and splits the
batch into individual items.  Second, when a wait operation such as
\vv{recv()} blocks, \xxx splits the segment at the entry of the blocking
wait.  The rationale is that blocking wait is typically done at the last
during one step of event processing.  Third, if a segment communicates to
too many peering processes, \xxx splits the segment when the set of peers
differs.  Specifically, for each message, \xxx maintains a set of two
peers including (1) the direct sender or receiver of the message and (2)
the beneficiary of the message (macOS allows a process to send or receive
messages on behalf of a third process).  \xxx splits when two consecutive
message operations have non-overlapping peer sets.

\subsubsection{Mitigating Under-Connections}\label{subsec:fix-under}

Under-connections are primarily due to data dependencies.  Currently \xxx
queries the user to identify the memory locations of the data flags. It
is conceivable to leverage memory protection techniques to infer them
automatically, as demonstrated in previous record-replay
%% https://www.usenix.org/legacy/events/usenix05/tech/general/king/king.pdf
%% https://web.eecs.umich.edu/~pmchen/papers/dunlap08.pdf
work~\cite{king2005debugging, dunlap2008execution}, it is out of the scope
of this paper and we leave it for future work. Currently, to discover a data
flag, the user re-runs the application with \xxx to collect instruction traces
of the concurrent events in both the baseline and spinning cases and detects
where the control flow diverges. She then reruns the application with \xxx to
collect register values for the basic blocks before the divergence and uncovers
the address of the data flag. Once the user identifies a data flag, \xxx traces
it using either binary instrument, such as the \vv{need\_display} flag in
CoreAnimation (\S\ref{sec:inaccuracy}), or with watchpoints. \xxx add a causal
edge between a write to a data flag to the corresponding read to the flag.
For the known data flags mentioned above, \xxx traces them by default.
