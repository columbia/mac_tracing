\subsection{Handling Inaccuracies} \label{subsec:handle}

\subsubsection{Mitigating Over-Connections}\label{subsec:fix-over}

From a high-level, \xxx deals with over-connections by heuristically splitting
an execution segment that appears mixing handling of multiple requests.

Specifically, \xxx splits based on three criteria. First, \xxx recognizes a small
set of well-known batch processing patterns such as \vv{dispatch\_mig\_server()}
(\S\ref{subsec:overconnections}) and splits the batch into individual items. Second,
when a wait operation such as \vv{recv()} blocks, \xxx splits the segment at the
entry of the blocking wait. The rationale is that blocking wait is typically
done as the last step in event processing. Third, if a segment
communicates to too many peering processes, \xxx splits the segment when the
set of peers differs. Specifically, for each message, \xxx maintains a set
of two peers including (1) the direct sender or receiver of the message and
(2) the beneficiary of the message (macOS allows a process to send or receive
messages on behalf of a third process). \xxx splits when two consecutive message
operations have non-overlapping peer sets.

\subsubsection{Mitigating Under-Connections}\label{subsec:fix-under}

The heuristics in the preceding subsection splits execution segments to
reduce over-connections, but they may occasionally remove true causal
edges. \xxx thus adds weak causal edges connecting the split segments to
mitigate under-connection. When a weak edge is encountered during
diagnosis, it queries users to decide whether to follow the weak edge or
stop (\S\ref{sec:overview}).

The other primary under-connections are due to data
dependencies. Currently \xxx queries the user to identify the data
flags. It is conceivable to leverage memory protection techniques to infer
them automatically, as demonstrated in previous record-replay
work~\cite{king2005debugging, dunlap2008execution}, which we leave it for
future work.

To discover a data flag, the user re-runs the application with \xxx to
collect instruction traces of both the normal and spinning cases.  These
traces record operand values of each instruction, so comparing them
reveals the address of the data flag that leads to their divergence.  Once
the user identifies the address of the data flag, \xxx traces it using
either binary instrumentation (\eg, the \vv{need\_display} flag in
CoreAnimation; see \S\ref{subsec:underconnections}), or its watchpoint
tool (\S\ref{subsec:tcp}). \xxx add a causal edge from a write to a data
flag to its corresponding read.
