\subsection{Case Studies}\label{sec:casestudy}

In this section, we demonstrate how \xxx helps to diagnose \nbug
spinning-cursor cases in popular applications. Table~\ref{table:bugs-desc}
describes these spinning-cursor cases. In Table~\ref{table:comps}, we compare
\xxx with Panappticon and AppInsghit. In the table we list the missing tracing
data for their failure. However, \xxx event graph does not fully support
automatic diagnosis. The user interaction is still required but not
overwhelming. As shown in Table ~\ref{table:results}, up to 3 user queries in
most cases suffice to find root cause path accurately. Although complex
applications like MicosoftWord and Chromium require more queries, 13 and 22
respectively, many of them result from repeated patterns. They can be easily
identified by users.

%Last paragraph: %In the remaining of this section, we
%present the case studies by category in (\S\ref{XXX}).


%TableXXX describes these spinning-cursor cases.
%Should have a table for issue description
%Bug ID |  Application  |  Bug Description
\begin{table}[ht]
\footnotesize
\centering
  \begin{tabularx}{\columnwidth}{l|cl}
    \hline
    \textbf{Bug ID} & \textbf{Application} & \textbf{Bug description}\\
    \hline
	\hline
	 0-Chormium & Chormium & \begin{tabular}{@{}l@{}}
	 Typing non-english in\\
	 search box on yahoo\\
	 webpage quickly triggers\\
	 spinning cursor\\
	 \end{tabular}
	 \\
     \hline
	 1-SystemPref & \begin{tabular}{@{}l@{}} 
	 System Preferences\\
	 \end{tabular}
	 & \begin{tabular}{@{}l@{}}
	 Disabling an online\\
	 external monitor and\\
	 rearranging windows\\
	 cause the freeze.
	 \end{tabular}
	 \\
     \hline
	 2-SequelPro& \begin{tabular}{@{}l@{}} 
	 Sequel Pro
	 \end{tabular}
	 & \begin{tabular}{@{}l@{}}
	 Lost connection freezes\\
	 the whole application.
	 \end{tabular}
	 \\
     \hline
	 3-TexStudio & \begin{tabular}{@{}l@{}} 
	 TeXStudio
	 %: LaTeX editor
	 \end{tabular}
	 & \begin{tabular}{@{}l@{}}
	 Modification on bib file\\
	 from vim causes the main\\
	 window spinning.
	 \end{tabular}
	 \\
     \hline
	 4-Installer & \begin{tabular}{@{}l@{}} 
	 Installer
	 \end{tabular}
	 & \begin{tabular}{@{}l@{}}
	 Move cursor out of the\\
	 authentication window \\
	 causes spinning.
	 \end{tabular}
	 \\
     \hline
	 5-Notes& \begin{tabular}{@{}l@{}} 
	 Notes\\
	 \end{tabular}
	 & \begin{tabular}{@{}l@{}}
	 launching Notes with \\
	 a saved relatively\\
	 long note.\\
	 \end{tabular}
	 \\
     \hline
	 6-TextEdit & \begin{tabular}{@{}l@{}}
	 TextEdit
	 \end{tabular}
	 & \begin{tabular}{@{}l@{}}
	 copy on text over 30M\\
	 cause freeze.
	 \end{tabular}
	 \\
     \hline
	 7-MSWord & \begin{tabular}{@{}l@{}}
	 Microsoft Words
	 \end{tabular}
	 & \begin{tabular}{@{}l@{}}
	 copy a whole document\\
	 over 400\\
	 pages cause hang.
	 \end{tabular}
	 \\
     \hline
	 8-SlText & Sublime Text
	 & \begin{tabular}{@{}l@{}}
	 Copy and paste in file\\
	 over 49000 lines.
	 \end{tabular}
	\\
    \hline
	 9-TextMate & TextMate 
	 & \begin{tabular}{@{}l@{}}
	 Paste text over 4000\\
	 lines causes spinning.
	 \end{tabular}
	\\
    \hline
	 10-CotEditor & CotEditor
	 & \begin{tabular}{@{}l@{}}
	 Paste in file over 4000\\
	 lines causes spinning.
	 \end{tabular}
	\\
	 \hline
  \end{tabularx}
  \caption{Bug Descriptions. We assign each bug in Column \textbf{Bug ID} to ease discussion}
  \label{table:bugs-desc}
\end{table}

%  {\caption{Inaccuracy handling compared to Panappticon} 
%  	\vspace{-0.3cm}
%  	{
%		We calculate the over-connections mitigated with our splitting heuristics, and
%	the under-connections where wait on primitive is treated as the end boundary,
%	which Panappticon applies to background threads but breaks an "atomic" task
%	of a common callout. The percentages are calculated over the total edges in event graph.
%    }
%  \label{table:statistics}
%  }
%

\begin{table}[ht]
\footnotesize
\centering
  \begin{tabularx}{\columnwidth}{l|XXX}
  \hline
Bug ID & Panappticon & AppInsight & \xxx\\
\hline
\hline
2-SystemPref & \mycross(data flags) & \mycross(WindowServer) & \mycheck (2 data flags) \\
3-SequelPro & \mycross(batching in kernel task) & \mycross(sshd) & \mycheck(2 ME)*\\
4-Installer & \mycross(batching in worker thread) & \mycross(SecurityAgent, authd)& \mycheck(1 WE, 1 ME)*\\
5-TeXStudio & \mycross(appkit event, timer, batching in WindowServer) & \mycross(fseventd) & \mycheck(3 ME)*\\
\hline
  \end{tabularx}
  \caption{Compare Effectiveness of Causal Tracing System. *WE stands for user interaction on Weak Edges, *ME stands for user interaction on Multi-incoming Edges in \xxx.} 
  \label{table:comps}
\end{table}

%\begin{table}[ht]
%\footnotesize
%\centering
%  \begin{tabularx}{\columnwidth}{l|cccc}
%  \hline
% 	   &       &\multicolumn{2}{c}{size of path}& hueristics\\
%       & user  & \multicolumn{2}{c}{with}  & over \\
%Bug ID & interactions & interaction & heuristics  & interaction\\
%\hline
%\hline
%1-Chromium  & 13 & 32 & 303 & 9.47\\
%2-SystemPref & 1 & 2 & 30 & 15.00\\
%3-SequelPro  & 2 & 5 & 264 & 52.80\\
%4-Installer  & 2 & 6 & 36  & 6.00\\
%5-TeXStudio  & 3 & 6 & 44  & 7.33\\
%6-TextEdit  & 3 & 21 & 21 & 1.00\\
%7-MSWord  & 22 & 67 & 136 & 2.03\\
%8-Notes  & 2 & 10 & 42 & 4.20\\
%9-SlText  & 1 & 3 & 3 & 1.00\\
%10-TextMate  & 0 & 3 & 3 & 1.00\\
%11-CotEditor  & 1 & 4 & 6 & 1.50\\
%\hline
%  \end{tabularx}
%  \parbox{\columnwidth}
%  {\caption{Path slicing for buggy cases. 
%	We noticed that \xxx's query on incoming edges can be answered with the most
%	recent one, thus we automated the path slicing in \xxx with the heuristics.
%	The length of the path is usually much longer due to excessive traverses to
%	daemons and kernel task.}
%  \label{table:results}
%  }
%\end{table}




\subsubsection{Long Wait and Repeated Yield}

In this section, we discuss the cases where the \spinningnode is blocking on
wait event or yielding loop, corresponding to LongWait and RepeatedYield.
These cases are mostly can be verified by themselve, as the root causes are
manifested by comparing to normal scenarios.

\paragraph{2-SystemPref}

\vv{System Preferences} provides a central location in macOS to customize system
settings, \eg additional monitors configuration.
\vv{DisableMonitor}~\cite{disablemonitor} provides more functionality, \eg
enable/disable monitors online. We caught the spinning cursor while we disable
an external monitor and rearrange windows in \vv{Display} panel.

The log collected by \xxx contains 2 cases: 1) a baseline scenario where the
displays are rearranged with the enabled external monitor, and 2) a spinning
scenario as we described above. The \spinningnode in the main thread is dominated
by system calls, \vv{mach\_msg} and \vv{thread\_switch}, which falls into the
category of Repeated Yield. We discovered two missing data flags,
``\vv{\_gCGWillReconfigureSeen}'' and ``\vv{\_gCGDidReconfigureSeen}'', which
signify the configuration status and break the thread-yield loop. 
\xxx's diagnosis result reveals that the main thread in \vv{System Preferences} sets
them after receiving specific datagrams from \vv{WindowServer}. Conversely, the
setting of ``\vv{\_gCGDidReconfigureSeen}'' is missing in the spinning case.
The main thread thus repeatedly sent messages to \vv{WindowServer} for datagram.

In conclusion, we discovered that the bug is inherent in the design of the
\vv{CoreGraphics} framework, and would have to be fixed by Apple. We verified this
diagnosis by creating a dynamic binary patch to fix the deadlock. The
patched library makes \vv{DisableMonitor} work correctly, while preserving correct
behavior for other applications.

\paragraph{3-SequelPro}

\vv{Sequel Pro}~\cite{SequelPro} is a fast, easy-to-use Mac database management
application for \vv{MySQL}. It allows user to connect to database with socket or ssh.
We experienced the non-responsiveness of Sequel Pro when it lost network
connection and tried reconnections.

The tracing log contains two cases: 1) a quick network connection during login,
and 2) Sequel Pro lost connection for a while. Although \xxx identified the
\spinningnode and \similarnode with ease, the backward slicing from
\similarnode encountered multiple incoming edges, including one from a kernel
thread, where batching processing from different applications happens.
Interaction is helpful and reduces the noise in the path.  Diagnosis on the
\spinningnode, comparing with the normal causal path, tells that the main thread
is blocking on a kernel thread, which in turn waits for a ssh thread. Thus the
root cause is the blocking for network IO.

\paragraph{4-Installer}

\vv{Installer}~\cite{Installer} is an application that extracts
and installs files out of \vv{.pkg} packages in macOS. When \vv{Installer} pops up a
window for privileged permission during the installation of
\vv{jdk-7u80-macosx-x64}, moving the cursor out of the popup window triggers a
spinning cursor.

\xxx successfully records the baseline scenario with our operations. We first
type in password in the pop-up window and then click the back button to
reproduce the spinning case described above.  Examining the \spinningnode and
its \similarnode, \xxx reveals the daemon \vv{authd} blocks on semaphore while
the main thread is waiting for \vv{authd}. Further checking on \vv{authd}, \xxx
finds out \vv{SecurityAgent} processes user input and wakes up \vv{authd} in
baseline scenario. In conclusion, moving the mouse out of the authentication
window causes the missing edge from \vv{SecurityAgent} to \vv{authd}, which in
turn blocks \vv{Installer}.

We also discovered a communication pattern in \vv{Installer} underpinning the
crucial of interactive debugging. It involves four vertices in four threads,
vertex $Vertex_{main}$ in the main thread, and $Vertex_1$ to $Vertex_3$ in
three worker threads. First, the main thread wakes up three worker threads.
Then one worker thread is scheduled to run. At its end, another worker thread,
which waits on mutex lock, is woken in $Vertex_2$, which in turn wakes up the
next worker thread in $Vertex_3$. While \xxx is slicing backward, $Vertex_3$
has two incoming edges: one is from $Vertex_{main}$, and the other one is from
$Vertex_2$. Since users can peek the edges before making decision, they are
likely to figure out that the three worker threads contend with mutex lock, and
all of them are successors of $Vertex_{main}$.


\subsubsection{Long Running}

In this section, we discuss the cases where the \spinningnode is busy on the
CPU. Most text editing apps fall into this bug category. We studied bugs on
TeXstudio, TextEdit, Microsoft Word, SublimeText, TextMate and CotEditor,
to reveal the root causes.

\para{5-TeXStudio}

\vv{TeXstudio}~\cite{TeXStudio} is an integrated writing environment for
creating LaTeX documents. We noticed a user reported spinning cursor on the
modification of bibliography (bib) file. Although the issue was closed by the
developer for insufficient information, we reproduced it with our bib file
around 500 items in a \vv{TeXStudio} tab. When we touch the file in
other editors like vim, a spinning cursor appears in \vv{TeXStudio}'s window.

\xxx recognizes the \spinningnode belongs to the category of LongRunning. The
causal path sliced from the \spinningnode by \xxx reveals the long-running
function is a callback from daemon \vv{fseventd}, and the long processing segment
is busy to CalculateGrowingBlockSize, even without modifications to the file.
The advantage of \xxx over other debugging tools is it narrows down the root
cause with the inter-processes execution path.

\para{6-TextEdit}

\vv{TextEdit} is a simple word processing and text editing tool from Apple,
which often hangs on editing large files.  When \xxx is used to diagnose
this issue, the heuristics of choosing the most recent edge is powerful enough 
to get the causal path.

The event graph reveals a communication pattern where a kernel thread is woken
from I/O by another kernel thread; the woken kernel thread processed
a timer callback function armed by \vv{TextEdit} and finally woke a \vv{TextEdit}
thread. In the pattern, the kernel thread has two incoming edges. One is from
another kernel thread's IO completion, and the other is from \vv{TextEdit}'s timer
creation. It not hard to reveal the high level semantics. \vv{TextEdit} arms a timer
for IO work, and the kernel thread gets the notification for the completion of IO
and processes the timer callback. 
Although the most recent incoming edge to the vertex reflects the purpose of
the execution segment in this case, it is not general enough to for all vertex
patterns.

\para{7-MSWord}

\vv{Microsoft Word} is a large and complex piece of software. \xxx can analyze
the event graph, but it identifies multiple possible root causes: the length of
path interactively sliced from the \spinningnode is 67, while the slicing with
heuristics of choosing the most recent edge generates a path of 136 vertices.

We compared the paths and find they diverge from the third vertex backward
from \spinningnode. In the vertex, a \vv{Microsoft Word} thread is
woken by another \vv{Microsoft Word} thread, and launches a service
\vv{NSServiceControllerCopyServiceDictionarie}. The woken thread sends a message
to \vv{launchd} to register the new service and waits for a reply message. In
this case, a user can accurately identify that the execution segments is on
behalf of \vv{Microsoft Word}, instead of \vv{launchd}. where the most recent
incoming edge comes from. Heuristics is likely to identify all possibilities
without priority. We rely on user interaction in this case to find the true root
cause. We identified \vv{Microsoft Word} spends most time on file accesses and
the copy command triggers those accesses.


\para{Other Editing Apps}

Select, copy, paste, delete, insert and save are common operations for text
editing. However, these operations on a large context usually trigger spinning
cursors. In Table~\ref{table:texteditapps}, we list the root causes reported by
\xxx, including the most costly functions in the event handler, and the user
input event (derived from path slicing).

\begin{table}[tb]
\vspace{-0.2cm}
\footnotesize
\centering
  \begin{tabularx}{\columnwidth}{l|l|l}
  \hline
                  &                     &\\
  \textbf{BUG-ID} & \textbf{costly API} &UI\\
  \hline
  \hline
  8-Notes         & \begin{tabular}{@{}l@{}}
  					\vv{1)NSDetectScrollDevicesThe}\\
					\vv{\xspace -nInvokeOnMainQueue}\\
					\end{tabular}
   		          & \begin{tabular}{@{}l@{}}
				  	\vv{system defined event}
					\end{tabular}
				  \\
  \hline
  9-SlText   & \begin{tabular}{@{}l@{}} 
					\vv{1)px\_copy\_to\_clipboard}\\
  					\vv{2)\_\_CFToUTF8Len}\\
  					\end{tabular}
				  & \vv{key c}
				  \\
  \hline
  10-TextMate      & \begin{tabular}{@{}l@{}}
  					\vv{1)-[OakTextView paste:]}\\
					\vv{2)CFAttributedStringSet}\\
					\vv{3)TASCIIEncoder::Encode}\\
  					\end{tabular}
				  & \vv{key v}
				  \\
  \hline
  11-CotEditor    & \begin{tabular}{@{}l@{}}
  					\vv{1)CFStorageGetValueAtIndex}\\
					\vv{2)-[NSBigMutableString}\\
					\vv{\xspace characterAtIndex:]}\\
  					\end{tabular}
   		          & \begin{tabular}{@{}l@{}}
				  	\vv{key Return}
  					\end{tabular}

				  \\
  \hline
  \end{tabularx}
  \caption{Root cause of spinning cursor in editing Apps.}
  \label{table:texteditapps}
  \vspace{-0.2cm}
\end{table}



\subsubsection{Summary}
Overall, in the case of simple text editing applications, \xxx can identify the
UI event that causes a spinning cursor by merely relying on a few heuristics.
However, these heuristics may make the wrong decision in complicated cases, and
misidentify the relationships between intra/inter-thread events. It is unlikely
that there exists a single graph search method that works in all cases, e.g.
when given the choice between multiple incoming edges, the most recent match is
sometimes correct, but sometimes not. This is why our system relies on expert
knowledge of users to reconstruct a developer's intent and accurately diagnose
performance issues.
