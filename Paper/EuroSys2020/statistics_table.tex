
\begin{table}[ht]
\footnotesize
\centering
  \begin{tabularx}{\columnwidth}{l|ccc}
  \hline
       & \% & \% & user provided \\
Bug ID & over-connections& under-connections & data flag \\
\hline
\hline
1-Chromium & 0.02 & 0.02 & 0 \\
2-SystemPref & 0.56 & 2.48 & 2 \\
3-SequelPro & 0.49 & 0.35 & 0 \\
4-Installer & 4.39 & 2.83 & 0 \\
5-TeXStudio & 2.43 & 0.58 & 0 \\
6-Notes & 2.97 & 11.53 & 0 \\
7-TextEdit & 7.97 & 0.72 & 0 \\
8-MSWord & 6.72 & 1.04 & 0 \\
9-SlText & 4.07 & 0.92 & 0 \\
10-TextMate & 2.15 & 2.18 & 0 \\
11-CotEditor & 4.81 & 5.32 & 0 \\
\hline
  \end{tabularx}

  \parbox{\columnwidth}
  {\caption{Inaccuracy handling compared to Panappticon} 
  	{
		We calculate the over-connections mitigated with our splitting heuristics, and
	the under-connections where wait on primitive is treated as the end boundary,
	which Panappticon applies to background threads but breaks an "atomic" task
	of a common callout. The percentages are calculated over the total edges in event graph.
    }
  \label{table:statistics}
  }
\end{table}


\begin{table}[ht]
\footnotesize
\centering
  \begin{tabularx}{\columnwidth}{l|cccc}
  \hline
 	   &       &\multicolumn{2}{c}{size of path}& hueristics\\
       & user  & \multicolumn{2}{c}{with}  & over \\
Bug ID & interactions & interaction & heuristics  & interaction\\
\hline
\hline
1-Chromium  & 13 & 32 & 303 & 9.47\\
2-SystemPref & 1 & 2 & 30 & 15.00\\
3-SequelPro  & 2 & 5 & 264 & 52.80\\
4-Installer  & 2 & 6 & 36  & 6.00\\
5-TeXStudio  & 3 & 6 & 44  & 7.33\\
6-TextEdit  & 3 & 21 & 21 & 1.00\\
7-MSWord  & 22 & 67 & 136 & 2.03\\
8-Notes  & 2 & 10 & 42 & 4.20\\
9-SlText  & 1 & 3 & 3 & 1.00\\
10-TextMate  & 0 & 3 & 3 & 1.00\\
11-CotEditor  & 1 & 4 & 6 & 1.50\\
\hline
  \end{tabularx}
  \parbox{\columnwidth}
  {\caption{Path slicing for "buggy" cases} 
	{
	we noticed that \xxx's query on incoming edges can be answered with the most
	recent one, thus we automated the path slicing in \xxx with the heuristics.
	The length of the path is usually much longer due to excessive traverses to
	daemons and kernel task.
    }
  \label{table:results}
  }
\end{table}
