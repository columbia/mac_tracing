\subsection{System Preferences}

System Preferences provides a central location in MacOS to customize system
settings. The ``Display'' pane allows users to configure additional monitors,
mirroring or extending their workspace. However, the software does not support
disabling monitors that are online (the user must physically unplug monitors).
There is a tool called DisableMonitor~\cite{disablemonitor}, distributed via
GitHub, which addresses this functionality. Surprisingly, we find performance
bug in System Preferences when we disable an external monitor, and rearrange
the windows in the Display panel afterward. The System Preferences windows
freezes for seconds in this situation.

To diagnose this issue, we run the System Preferences app with \xxx. We
rearrange the displays with two active monitors, and repeat the process with
one of them deactivated. \xxx collects 132MB data, and constructs dependency
graph system-wide in the period, which contains 428,785 vertexes and 320,554
edges.

To diagnose the root cause, \xxx first finds out when the NSEvent thread in
System Preferences notifies WindowServer to draw the spinning cursor and marks
it as \textit{t}. The node in the main UI thread causing the non-responsiveness
must overlap the time interval \textit{(t - 2s, t)}. It is not hard for
\xxx to tell the spinning node in the main UI thread, either busy processing or
blocking.

The spinning node in the main UI thread is dominated by \textit{mach\_msg} and
\textit{thread\_switch}, both of which are meant to wait for available data
ping from WindowServer. Noticing the timeouts of \text{thread\_switch}, \xxx
classifies this case into the third category (\S\ref{subsec:debug}) and heads
to find a comparable normal node in the same thread. As its semantics are not
descriptive enough to identify its comparable node, \xxx extends the comparison
with its proceeding nodes.

In the normal case, System Preferences gets rid of the thread\_switch
quickly after receiving messages from WindowServer. It proceeds
to \textit{displayReconfigured}. On the other hand, the spinning
node ends up sending message for the available datagram ping with
\textit{CGSSnarfAndDispatchDatagrams}. By checking the lightweight callstacks,
\xxx figures out the messages from WindowServer in the previous nodes are
responses to data available pings from \textit{activeDisplayNotificationHandler}.

Given the information, we launched the interactive debugging by 
feeding the debugging script with those APIs mentioned above. We set the method
\textit{activeDisplayNotificationHandler} as a breakpoint where the script begins
debugging. \textit{displayReconfigured} and \textit{CGSSnarfAndDispatchDatagrams}
are used to indicate the end of debugging for the normal case and spinning case
respectively.

By diff'ing the two logs, we notice the different
branches in \textit{display\_notify\_proc} called by
\textit{activeDisplayNotificationHandler}. The handler depends on received
datagram and two shared variables ``\v{\_gCGWillReconfigureSeen}'' and
``\v{\_gCGDidReconfigureSeen}'' to finish a display configuration. In
both case the first variable is set to indicate the begin of display
configuration. In normal case, it receives a datagram which set the variable
``\v{\_gCGDidReconfigureSeen}'' in \textit{display\_notify\_proc} and
finish display reconfiguration, while in the spinning case such datagram
is never received. Instead, an alternative datagram just drive it through
\textit{display\_notify\_proc} without setting any variable, which causes the
repeating \textit{thread\_switch} in the handler.

[In conclusion, the bug is... would have to be fixed by... we provide a binary fix...]
