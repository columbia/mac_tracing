\subsection{Spinning Node detection}

Spinning beachball is a painful sight for Mac users, signifying that the
application is non-responsive. It usually remains for minutes at a time, leaving
the user at a loss and unable to do anything productive. Under this situation, a
hang reporting tool, spindump, usually invoked by root to sample for debugging.

To figure out the spinning node in the main thread, we turned to the event
graph, and sliced path backward from the launch of spindump, considering it is
another indicator of non-responsive in app. The path showed it was launched
after receiving a message from WindowServer, which in turn received a message
from the NSEvent thread of the freezing app. The call stacks attached to the
messages further revealed NSEvent thread per process fetches CoreGraphics events
from WindowServer, converts and creates NSApp\_event for the main thread. If the
main thread is not spinning ,a timer is armed. If the main thread processes
the next event before the timer fires, nothing happens and the timer gets
re-armed. Otherwise, NSEvent thread sends a message to WindowServer via the
API ``\vv{CGSConnectionSetSpinning}''from the timer handler, and WindowServer
notifies the CoreGraphics to draw a spinning wait cursor over the application
window.

Moreover, with the call stack symbols found above as input, our detail debugging
script detected two share variables ``\vv{is\_mainthread\_spinning}'' and
``\vv{dispatch\_to\_mainthread}'', are used to note the main thread status. With
these discovery, we can make use of either the API or the shared variable to
identify the spinning node in the main thread and the UI event cause the hanging.
