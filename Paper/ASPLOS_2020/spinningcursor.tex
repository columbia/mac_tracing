%\subsection{Spinning Node detection}
\indent
%Spinning beachball is a painful sight for Mac users, signifying that an
%application is non-responsive and leaving the user at a loss and unable to do
%anything productive.
When the spinning beachball shows up, a hang reporting tool, spindump usually
kicks in automatically to sample for debugging. To figure out the spinning
node in the main thread, we turn to the event graph, and slice path backward
from the launch of spindump.

The path shows the spindump is launched after receiving a message from
WindowServer, which received a message from the NSEvent thread of the freezing
app. The Back\_trace events attached to the messages further reveal NSEvent
thread per process fetches CoreGraphics events from WindowServer, converts and
creates NSApp\_event for the main thread. If the main thread is not spinning,
a timer is armed to count down for the NSApp\_event. If the main thread
fails to process it before the timer fires, NSEvent thread sends a message
to WindowServer via the API ``\vv{CGSConnectionSetSpinning}''from the timer
handler, and WindowServer notifies the CoreGraphics to draw a spinning wait
cursor over the application window. Moreover, while exploiting how NSEvent
thread communicates with the main thread, we found two shared variables,
``\vv{is\_mainthread\_spinning}'' and ``\vv{dispatch\_to\_mainthread}'' are used
to identify the main thread status. As a result, \xxx can make use of either the
API or the shared variables to identify the spinning node in the main thread.
