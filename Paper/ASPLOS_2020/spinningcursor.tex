%\subsection{Spinning Node detection}

\paragraph{How \xxx Detects Spinning Cursors}
When the spinning cursor shows up, a hang reporting tool, \spindump usually
kicks in automatically to sample callstacks for debugging. To figure out the
\spinningnode in the main thread, we turn to the event graph, and slice path
backward from the launch of \spindump.
The path shows that \spindump is launched after receiving a message from
\vv{WindowServer}, which received a message from the \vv{NSEvent thread} of
the freezing app. The call stack attached to the messages further reveals
\vv{NSEvent thread} per process fetches \vv{CoreGraphics} events from
\vv{WindowServer}, converts and creates \vv{NSApp events} for the main UI
thread. If the main thread is not spinning, a timer is armed to count down
for the NSApp event. If the main thread fails to process it before the timer
fires, \vv{NSEvent thread} sends a message to \vv{WindowServer} via the API
``\vv{CGSConnectionSetSpinning}'' from the timer handler, and \vv{WindowServer}
notifies the \vv{CoreGraphics} to draw a spinning cursor over the application
window. Moreover, while exploiting how \vv{NSEvent thread} communicates with
the main thread, we found two variables, ``\vv{is\_mainthread\_spinning}'' and
``\vv{dispatch\_to\_mainthread}'', indicating the main thread status. As a
result, \xxx can make use of either the API or the data flag to identify the
\spinningnode in the main thread.
