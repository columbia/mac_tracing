Prior systems used causal tracing, a powerful technique that traces low-level
events and builds dependency graphs, to diagnose performance issues. However,
they all assume that accurate dependencies can be inferred from low-level
tracing by either limiting applications to using only a few supported
communication patterns or relying on developers to manually provide dependency
schema upfront for all involved components. Unfortunately, based on our own
study and experience of building a causal tracing system for a widely used,
closed-souce operating system macOS, we found that it is extremely difficult,
if not impossible, to build accurate dependency graphs. We report patterns
such as data dependency and batch processing that introduce inaccuracy. We
present \xxx, a practical system for effectively debugging performance issues
in modern desktop applications despite the inaccuracy of causal tracing. \xxx
lets a user easily inspect current diagnostics and interactively provide
more domain knowledge on demand to counter the inherent inaccuracy of causal
tracing. We implemented \xxx in macOS and evaluated it on \nbug real-world, open
spinning-cursor issues in widely used applications. The root causes of these
issues were largely previously unknown. Our results show that \xxx effectively
helped us locate all root causes of the issues and incurred only \cpuoverhead
CPU overhead in its system-wide tracing.
