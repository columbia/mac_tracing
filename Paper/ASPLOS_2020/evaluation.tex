
\begin{table}[h]
\footnotesize
\centering
\begin{tabular}{l|ccccc}
\hline
 & 1st & 2nd & 3rd & 4th & 5th\\
\hline
 without \xxx& 5.98 & 6.23 & 6.18 & 6.05 & 6.28\\
 with \xxx& 6.29 & 6.01 & 6.09 & 6.28 & 6.01\\
\hline
average overhead& \multicolumn{5}{c}{0.13\%}\\
\hline
\end{tabular}
\caption{Score From iBench}
\label{tab:ibench}
\end{table}


\begin{table}[h]
\footnotesize
\centering
\begin{tabular}{ll|rrr}
\hline
 & kb/s & With Argus & Without Argus & overhead\\
 \hline
\textbf{bonnie++}&read char & 21922 & 22149 & 0.01\\
 seqential& read block & 226931 & 244089 & 0.07\\
 & rewrite & 246807 & 267491 & 0.08\\
 & write char & 22924 & 22936 & 0.00\\
 & write block & 4073361 & 4396387 & 0.07\\
 \hline
 seq& file create & 17391 & 17381 & 0.00\\
 & file delete & 18089 & 19401 & 0.07\\
 \hline
 random& create & 17472 & 17887 & 0.02\\
 & delete & 8849 & 9567 & 0.08\\
 \hline
 \hline
\textbf{iozone} & initial write & 1199453 & 1318572 & 0.09\\
 & rewerite & 3663066 & 4059912 & 0.10\\
 \hline
 & average & - & - & 0.05\\
 \hline
\end{tabular}
\caption{IO throughput with bonnie++ and iozone}
\label{tab:iothroughput}
\end{table}


\begin{table*}[ht]
\footnotesize
\centering
\begin{tabular}{l|rrr|rrr|ccc}
	\hline
	\hline
Chromium Benchmark&\multicolumn{3}{c}{\textbf{with Argus}} & \multicolumn{3}{c}{\textbf{without Argus}} & \multicolumn{3}{c}{Overhead}\\
(in seconds)& real & user & sys & real & user & sys & real & user & sys \\
\hline
system\_health.memory\_desktop & 11592 & 18424 & 1821 & 11317 & 18401 & 1415 & 0.02 & 0.00 & 0.29\\
rasterize\_and\_record\_micro.top\_25 & 1579 & 2142 & 135 & 1654 & 2166 & 116 & -0.05 & -0.01 & 0.16\\
blink\_perf & 16210 & 17227 & 959 & 15877 & 16724 & 766 & 0.02 & 0.03 & 0.25\\
webrtc & 726 & 2023 & 225 & 725 & 2130 & 168 & 0.00 & -0.05 & 0.34\\
memory.desktop & 1231 & 2238 & 267 & 1188 & 2200 & 190 & 0.04 & 0.02 & 0.41\\
loading.desktop.network\_service & 24580 & 52751 & 6294 & 23696 & 52327 & 4197 & 0.04 & 0.01 & 0.50\\
dromaeo & 206 & 227 & 15 & 192 & 212 & 12 & 0.07 & 0.07 & 0.29\\
dummy\_benchmark.histogram & 49 & 48 & 8 & 33 & 36 & 4 & 0.50 & 0.32 & 0.96\\
v8.browsing\_desktop & 2462 & 4489 & 491 & 2325 & 4440 & 303 & 0.06 & 0.01 & 0.62\\
octan.desktop & 112 & 142 & 8 & 98 & 124 & 5 & 0.14 & 0.15 & 0.44\\
speedometer & 618 & 802 & 31 & 600 & 782 & 24 & 0.03 & 0.03 & 0.32\\
page\_cycler\_v2.typical\_2 & 8020 & 14435 & 1453 & 7847 & 14215 & 1019 & 0.02 & 0.02 & 0.43\\
smoothness.oop\_rasterization.top\_25\_smooth & 864 & 1450 & 156 & 833 & 1412 & 126 & 0.04 & 0.03 & 0.24\\
\hline
AVERAGE & \multicolumn{3}{c}{-} &  \multicolumn{3}{c}{-}  & 0.07 & 0.05 & 0.4\\
\hline
\hline
\end{tabular}
\caption{Chromium benchmark}
\label{tab:chromium benchmark}
\end{table*}

\section{Performance Evaluation}\label{sec:evaluation}

In this section we present the performance impact of the live deployment of
\xxx. We deploye \xxx on a MacBookPro9,2, which has Intel Core i5-3210M CPU with
2 cores and 4 thread, 10GB DDR3 memory and a 1T SSD.

\xxx has a very small space overhead with the configuration of its tracing
tool. It uses the ring buffer with configured 2G by default to collect tracing
events. The memory used to store events is fixed to 512M by Apple, which is pretty
low with regards to the memory usage of modern applications. In the remaining
of this section, we measure \xxx's overhead overall with iBench scores, IO
throughtput degradation with bonnie++, iozone and CPU overhead with chromium
benchmarks.

\paragraph{iBench}

We first show the five runs of iBench with and without \xxx to evaluate the
overall performance. The machine is clean booted for each run, and the higher
score means it perfoms better. As shown in Table\ref{tab:ibench}, their
performance are almost of no difference, only 0.13\% degradation on average.

\paragraph{IO Throughput}

Next, we evaluate the IO throughput with bonnie++ and iozone. As shown in the
Table~\ref{tab:iothroughput}, the throughputs of sequential read and write
by characters with and without \xxx are almost same. Read and write by block
imposes less than 10\% overhead in the both microbenchmarks, bonni++ and iozone.
With the selected event types in our system, the tracing tool integrated in \xxx
only adds 5\% IO overhead on average.


\paragraph{CPU}

We evaluate \xxx's CPU overhead with chromium benchmarks by recording their time
usage on real, user and sys. Although the sys time overhead is realatively
high due to the tracing events usually across the kernel boundary, they are not
tiggered too frequently in our daily software usage, including browsers. The
time cost is mostly under 5\%, except the \vv{dummy\_benchmark.histogram}. As
shown in Table ~\ref{tab:chromium benchmark}, the time overhead for real, user
and sys are 7\%, 5\% and 40\% respectively.

