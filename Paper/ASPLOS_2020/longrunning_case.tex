\subsection{Long Running}

In this section, we discuss the cases where the \spinningnode is busy on the
CPU. Most text editing apps fall into this bug category. We studied bugs on
TeXstudio, TextEdit, Microsoft Word, Sublime Text, Text Mate and CotEditor,
listed in Table~\ref{table:bugs-desc} to reveal their root causes.

%represent cases in this category.

\paragraph{3-TeXStudio}

TeXstudio~\cite{TeXStudio} is an integrated writing environment for creating
LaTeX documents. We noticed a user reported spinning cursor when he
modified his bib file. Although the issue was closed by the developer, due to
insufficient information to reproduce the bug, we reproduced it with a large bib
file opened in a tab. Each time we touched the file through another editor, vim
for example, the application window showed a spinning cursor.

\xxx recognizes the \spinningnode belongs to the category of \textbf{Long
Running}. Slicing causal path from the vertex, \xxx reaches daemon
``\vv{fseventd}'' and figures out that the long-running function is invoked by a callback
function from this daemon. The advantage of \xxx over other debugging tools is
it helps to narrow down the root cause with the path. If the user's bug report
had included details captured with \xxx, it may have provided the developer with
enough information to reproduce the bug successfully.

\paragraph{6-TextEdit}

TextEdit is a simple word processing and text editing tool shipped by Apple, which
often hangs on the editing of large files. 
When \xxx is used to diagnose this issue, the automated heuristics are frequently
powerful enough to find the true root cause.
%\xxx reveals the same causal path with heuristics as with user
%interaction.
The graph reveals a communicating pattern in the vertices where a kernel
thread was woken up from blocking IO by another kernel thread; and it processed
the timer armed by TextEdit and woke up one of its threads. The first incoming
edge is from the second kernel thread, and the second incoming edge is from
TextEdit(from vertex where the timer armed to where it is processed). Users can
make decision on the vertex base on the event sequence, which implies the story:
TextEdit first arms the timer for IO work, then kernel threads work for it, and
finally it processes the timer and wakes up TextEdit when finished.

The success of our automated analyses 
is not surprising because the pattern of vetices fits our expected bug structure,
where we choose the most recent incoming edge.
Although the automatic heuristics works for this particular application, it is
not general enough to succeed for all patterns.

\paragraph{7-MSWord}

Microsoft Word is a large and complex piece of software. \xxx can analyze the
event graph, but it identifies multiple possible root causes: the length of path
interactively sliced from the \spinningnode is 67, while the automatic slicing
generates a path of 136 vertices.

%We rely on user interactions to help speed up the path slicing.

We compared the path and find that the earliest difference exists in the
predecessor of the third vertex in backward paths. In the vertex, user
can learn from the call stack that \vv{Microsoft Word} launches a service
\vv{NSServiceControllerCopyServiceDictionarie} after being woken by another
\vv{Microsoft Word} thread; this thread then sends a message to \vv{launchd}
to register the new service and waits for a reply message. With the most
recent edge heuristics in automatic slicing, \xxx chose \vv{launchd} as its
predecessor, but the user can more precisely identify that the execution segment
is on behalf of the first thread. We rely on user interaction in this case
to find the true root cause, since \xxx has identified multiple
possibilities.

\paragraph{Other Editing Apps}

Select, copy, paste, delete, insert and save are common operations for text
editing. However, these operations on a large context usually trigger spinning
cursors. Due to their implementation, CotEditor and TextMate successfully
avoid hangs on copy and selection operations. \xxx can helps the developer
to figure out the more efficient way to implement event handlers. In
Table~\ref{table:texteditapps}, we list the reports from each application's
\spinningnode, including the event handler and most costly functions, and the
user input event (derived from path slicing) that triggers the hang.

\begin{table}[tb]
\vspace{-0.2cm}
\footnotesize
\centering
  \begin{tabularx}{\columnwidth}{l|l|l}
  \hline
  \hline
                  &                     &\\
  \textbf{BUG-ID} & \textbf{costly API} &UI\\
  \hline
  \hline
  5-Notes         & \begin{tabular}{@{}l@{}}
  					\vv{1)NSDetectScrollDevices}\\
					\vv{\xspace ThenInvokeOnMainQueue}\\
					\end{tabular}
   		          & \begin{tabular}{@{}l@{}}
				  	\vv{system}\\
					\vv{define}\\
					\vv{event}
					\end{tabular}
				  \\
  \hline
  6-TexEdit       & \begin{tabular}{@{}l@{}}
  					\vv{1)[NSTextView(NSPasteboard) \_write}\\
					\vv{\xspace RTFDInRanges:toPasteboard:]}\\
					\vv{2)get\_vImage\_converter}\\
  					\vv{3)get\_full\_conversion\_code\_fragment}\\
					\end{tabular}
				  & \vv{key c}
				  \\
  \hline
  7-MSWord        & \begin{tabular}{@{}l@{}}
					\vv{1)-[NSPasteboard setData:}\\
					\vv{\xspace forType:index:usesPboardTypes:]}\\
 					\vv{2)\_CFStringCreateImmutableFunnel3}\\
  					\vv{3)platform\_memmove}\\
					\vv{4)lseek}, \vv{5)fstat64}, \vv{6)fcntl}\\
					\end{tabular}
				  & \vv{key c}
				  \\
  \hline
  8-SlText   & \begin{tabular}{@{}l@{}} 
					\vv{1)px\_copy\_to\_clipboard}\\
  					\vv{2)\_\_CFToUTF8Len}\\
  					\end{tabular}
				  & \vv{key c}
				  \\
  \hline
  9-TextMate      & \begin{tabular}{@{}l@{}}
  					\vv{1)-[OakTextView paste:]}\\
					\vv{2)CFAttributedStringSet}\\
					\vv{3)TASCIIEncoder::Encode}\\
  					\end{tabular}
				  & \vv{key v}
				  \\
  \hline
  10-CotEditor    & \begin{tabular}{@{}l@{}}
  					\vv{1)CFStorageGetValueAtIndex}\\
					\vv{2)-[NSBigMutableString}\\
					\vv{\xspace characterAtIndex:]}\\
  					\end{tabular}
   		          & \begin{tabular}{@{}l@{}}
				  	\vv{key}\\
				  	\vv{Return}
  					\end{tabular}

				  \\
  \hline
  \end{tabularx}
  \caption{Root cause of spinning cursor in editing Apps}
  \label{table:texteditapps}
  \vspace{-0.2cm}
\end{table}

