\section{Overview}\label{sec:overview}

\subsection{Argus Work Flow}
\begin{figure}[tb]
    \centering
    %\documentclass{article}
%\usepackage{tikz}
%\usepackage{tikzpeople}
%\usetikzlibrary{shapes, shapes.misc}
%\usetikzlibrary{arrows, arrows.meta, decorations.markings}
%\usetikzlibrary{patterns}

%\usetikzlibrary{calc,backgrounds}
%\usepackage[active,tightpage]{preview}
% taken from manual
\makeatletter
\pgfdeclareshape{document}{
\inheritsavedanchors[from=rectangle] % this is nearly a rectangle
\inheritanchorborder[from=rectangle]
\inheritanchor[from=rectangle]{center}
\inheritanchor[from=rectangle]{north}
\inheritanchor[from=rectangle]{south}
\inheritanchor[from=rectangle]{west}
\inheritanchor[from=rectangle]{east}
% ... and possibly more
\backgroundpath{% this is new
% store lower right in xa/ya and upper right in xb/yb
\southwest \pgf@xa=\pgf@x \pgf@ya=\pgf@y
\northeast \pgf@xb=\pgf@x \pgf@yb=\pgf@y
% compute corner of ‘‘flipped page’’
\pgf@xc=\pgf@xb \advance\pgf@xc by-10pt % this should be a parameter
\pgf@yc=\pgf@yb \advance\pgf@yc by-10pt
% construct main path
\pgfpathmoveto{\pgfpoint{\pgf@xa}{\pgf@ya}}
\pgfpathlineto{\pgfpoint{\pgf@xa}{\pgf@yb}}
\pgfpathlineto{\pgfpoint{\pgf@xc}{\pgf@yb}}
\pgfpathlineto{\pgfpoint{\pgf@xb}{\pgf@yc}}
\pgfpathlineto{\pgfpoint{\pgf@xb}{\pgf@ya}}
\pgfpathclose
% add little corner
\pgfpathmoveto{\pgfpoint{\pgf@xc}{\pgf@yb}}
\pgfpathlineto{\pgfpoint{\pgf@xc}{\pgf@yc}}
\pgfpathlineto{\pgfpoint{\pgf@xb}{\pgf@yc}}
\pgfpathlineto{\pgfpoint{\pgf@xc}{\pgf@yc}}
}
}
\makeatother

%\begin{document}
\begin{center}

%%\resizebox{0.8\textwidth}{0.4\textwidth}{%
\resizebox{0.48\textwidth}{!}{%
\begin{tikzpicture}[>=latex]

% We need layers to draw the block diagram
\pgfdeclarelayer{background}
\pgfdeclarelayer{foreground}
\pgfsetlayers{background,main,foreground}

\tikzstyle{every node}=[font=\Large]
\tikzstyle{apps} = [draw, very thick, minimum height=3em, minimum width=5em, fill=white, rectangle, font={\sffamily\bfseries}]
\tikzstyle{systemComp} = [draw, very thick, minimum height=3em, minimum width=15.1em, fill=white, rectangle, font={\sffamily\bfseries}]
\tikzstyle{actionLabel} = [draw, pattern=north west lines, pattern color = red!20, ellipse, minimum width = 15em, minimum height= 5em]%shape aspect=3, minimum size=30, diamond]

\tikzstyle{doc} = [draw, thick, align=left, color=black, shape=document, minimum width=18em, minimum height=12em, shape=document, inner sep=2ex]
\tikzstyle{commanddoc} = [draw, thick, align=left, color=black, shape=document, minimum width=5em, minimum height=8em, shape=document, inner sep=2ex]
\tikzstyle{debuglogdoc} = [draw, thick, align=left, color=black, shape=document, minimum width=8em, minimum height=6em, shape=document, inner sep=2ex]
\tikzstyle{mininode} = [draw, rectangle, minimum height=1em, minimum width=1em]

\tikzstyle{prearrow} = [-, thick, line width=1em] %double distance = 1.2em, shorten >=-0.2em]
\tikzstyle{vecarrow} = [->, thick, line width=1em] %double distance = 1.2em, shorten >=-0.2em]
%\tikzstyle{vecarrow} = [thick, decoration={markings, mark=at position
   %1 with {\arrow[xshift=1.2em, scale=0.6] {triangle 90}}},
   %1 with {\arrow[xshift=1.5em]{Straight Barb[length=2pt 0.7]}}},
   %double distance=1.2em,
   %postaction= {decorate}]

%draw MacOS components
\node [apps, name=software1] at (0, 0) {chromium};
\node [apps, name=software2, right of=software1, node distance=5em] {daemon1};
\node [apps, name=software3, right of=software2, minimum width=5.2em, node distance=5em] {daemon2};
\node [systemComp, name=libs, below left = 0em and -10.21em of software2] {libs};
\node [systemComp, name=kernel, below of=libs, node distance=3em] {kernel};
\node (MacOS) [below of = kernel, minimum width=15em, node distance = 3em] {MacOS};

% draw Tracing Event log
\node (TracingEventLog) [minimum height=3em, minimum width=20em, right of=software3, node distance=22em] {TracingEventLog};
\node [doc, below of=TracingEventLog, name=log, node distance=4em] {\#timestamp, event\_type, att1, att2...\\30.4 Mach\_message 0x4ea20 0x3...\\31.7 Mach\_message 0x4ea20 0x3...\\33.2 Wake\_up 0xea45 0x16...};

%draw arrow from MacOS to Tracing Event log
%\draw[vecarrow, shorten >= -0.05em] (libs.east)+(0.5, 0) -- (log.west);%(TracingEventLog.west);
\draw[vecarrow] (libs.east)+(0.5, -0.35) -- (log.west);%(TracingEventLog.west);

%draw Arrow for Constructing Graph
\node [actionLabel, below of=log, node distance=9.5em, name=action1] {Construct Graph};
\draw [prearrow] (log.south) -- (action1.north);

%draw Dependency Graph
\node (DependencyGraph) [minimum height=3em, minimum width=20em, below of=action1, node distance=8em] {Dependency Graph};
\node [mininode, below of=DependencyGraph] (1c3) {};
\node [mininode, left of=1c3, node distance=2em] (1c2) {};
\node [mininode, below of=1c2, node distance=2em] (2c2) {};
\node [mininode, left of=2c2, node distance=2em] (2c1) {};
\node [mininode, left of=2c1, node distance=2em] (2c0) {};
\node [mininode, right of=2c2, node distance=2em] (2c3) {};
\node [mininode, right of=2c3, node distance=2em] (2c4) {};
\node [mininode, below of=2c0, node distance=2em] (3c0) {};
\node [mininode, right of=3c0, node distance=2em] (3c1) {};
\node [mininode, right of=3c0, node distance=6em] (3c2) {};
\node [mininode, below of=3c0, node distance=2em] (4c0) {};
\draw [->] (1c2.south) -- (2c0.north);
\node (DGendline)[minimum height=1em, minimum width=20em, below of=DependencyGraph, node distance=10em]{};

%\draw [vecarrow, shorten <=-0.2em, shorten >= 0.1em] (action1.south) -- (DependencyGraph.north);
\draw [vecarrow] (action1.south) -- (DependencyGraph.north);

%draw interactive Debugging part
\node [commanddoc, left of=DependencyGraph, node distance=18em](command){Debugging command\\search\_node\\check\_node\\lldb:br\\lldb:si};
\node[alice, minimum size=3em, left of=command, node distance=8em](user) {};
\node [debuglogdoc, left of=user, node distance=8em](debuglog){Debugging log:\\path: ...\\node\_id: ...\\execution\_list:\\  movq \%rax, \%rcx \\...};%\\  movq \$1, \%rbx\\  ...};
\node [actionLabel, below of=user, node distance=8em, name=action2] {Interactive Debugging};
\draw [->] (user.east) to [out=330,in=60] (action2.north east);
\draw [->] (action2.north west) to [out=120,in=210] (user.west);
\draw [->] (user.north)+(0, 0.1) to (libs.south);
\draw [prearrow](DGendline.north west) + (0, 0.5) -- (action2.east);
%\draw [prearrow, -, shorten >= -1.8em](DGendline.north west) + (0, 0.5) -- (action2.east);
%\node (result) [draw, rectangle, align=left, minimum height=2em, minimum width=20em, below of = action2, node distance=6em] {RootCause:\\UI thread blocking in Chromium because of\\wating for message from Render thread};
%\draw [vecarrow, shorten <= -0.2em] (action2.south) -- (result.north);

\node (result) [draw, rectangle, align=left, minimum height=20em, minimum width=9em, text width=8em, left of=debuglog, node distance=12em] {RootCause:\\UI thread of Chromium blocks due to wating for message from render thread};
\node (alignresult)[left of=action2, node distance=16em]{};
%\draw [vecarrow, shorten <= -1.8em] (action2.west) -- (alignresult.east);
\draw [vecarrow] (action2.west) -- (alignresult.east);

\begin{pgfonlayer}{background}
%%draw background rectangle for macos, graph and debugging log
\path (software1.west |- software3.north)+(-0.5, 0.5) node (a) {};
\path (MacOS.south east)+(0.5, -0.5) node (b) {};
\path[fill=yellow!20,rounded corners, draw=black!50, dashed] (a) rectangle (b);
%%draw backgroud for dependancy graph
\path (DependencyGraph.west |- DependencyGraph.north) node (a) {};
\path (DGendline.south east) node (b){};
\path[fill=yellow!20,rounded corners, draw=black!50, dashed] (a) rectangle (b);
%%draw background for interactive debugging
%\path (IDbeginline.west |- IDbeginline.north) node (a) {};
%\path (IDendline.south east) node (b) {};
%\path[fill=yellow!20,rounded corners, draw=black!50, dashed] (a) rectangle (b);

\end{pgfonlayer}
\end{tikzpicture}
}
\end{center}
%\end{document}

    \caption{\xxx Work Flow}
    \label{fig:argus-overview}
\end{figure}

In this section, we describe the steps a user takes to investigate a performance
anomaly with \xxx. Figure~\ref{fig:argus-overview} shows \xxx's work flow, which
consists of two phases. A user runs command ``\vv{\xxx start}'' to enter the
system-wide tracing phase, within which \xxx logs events as listed in Table
~\ref{table:event_types} (\S\ref{subsec:eventgraph}). Whenever a user detects
a performance issue such as a spinning cursor, she runs ``\vv{\xxx debug}'' to
enter the diagnosis phase.

%%In the cases we consider, the flow of information across threads and processes
%%is essential to discovering the system state that leads to a performance bug.
%%\xxx recovers UI actions from logged data rather than being told the actions
%%that a user performs, because not all of them may be relevant to the true bug.

Central to our system is our \emph{event graph}, a generalized control-flow
graph which includes inter-thread and inter-process dependencies. Diagnosis
and inferences are performed within this graph, in a semi-automated fashion:
\xxx performs searches and subgraph matching to trace logical events as they
flow through the system, and the user can interactively query this graph to
understand the problem or provide guidance to \xxx. The event graph is described
in further detail in \S\ref{subsec:eventgraph}. Next, we describe the graph
operations \xxx performs leading to a diagnosis.

\subsection{Diagnosis with Graph}\label{subsec:debug}

\begin{algorithm}[ht!]
    \caption{Main \xxx algorithm.}
    \label{alg:alg-main}
\begin{algorithmic}[1]
\Require{Program to run + buggy test case
			+ (optional) similar baseline test case}
\Ensure{Sequence of UI events triggered the performance problem 
		+ Nodes from Event Graph}
\Statex
\Function{\xxx Main}{}
\State {$TracingEvents$ $\gets$ Run program under \xxx,
		trigger baseline case (if possible) and buggy test case}\label{algmain:collectdata}
\State {$HeuristicsSet$ $\gets$ $\{default$ $ $ $heuristics\}$}
\State {$EventGraph$ $\gets$ ComputeGraph($HeuristicsSet$, $TracingEvents$)}
\label{algmain:recompute}
\State {$SpinningNode$ $\gets$ search $EventGraph$}\label{algmain:findspinningnode}
\If {IsCPUBusyExecuting($SpinningNode$)} 
	\State{run backward traversal of main thread until UI event found}\label{algmain:1st}
\Else
	\While{NotFound(BlockingFromUI \Or CircularDependency)}
	\State{$N_0$ $\gets$ $SpinningNode$}\label{algmain:2and3begin}
	\State{$N_1$ $\gets$ FindSimilarNode($EventGraph$, $N_0$)}
	\State{call AssistedGraphDiff($EventGraph$, $N_0$, $N_1$)}
	\State{ManuallyAdjust($N_0$, $N_1$)}
	\If {new\_hueristics were added}
		\State \Goto{algmain:recompute}
	\EndIf\label{algmain:2and3end}
	\EndWhile
\EndIf
\State{Return set of UI events and root cause related Nodes}
\EndFunction
\end{algorithmic}
\end{algorithm}

Algorithm~\ref{alg:alg-main} shows how \xxx investigates a performance
anomaly for users, given a buggy test case and (potentially) a baseline
test case for comparison purposes. We assume the user has a program to
execute and a buggy test case that can be probabilistically reproduced;
the tracing log collected by \xxx would contains both a buggy case and a
baseline case at the best, or the user can provide a normally executed test
case for comparison purpose, listed in line~\ref{algmain:collectdata}.
The tracing tool can run all the time in the background to capture the
probabalistic bug. \xxx initializes this debugging phase by constructing
an event graph from all logged events up to user-specified duration in
line~\ref{algmain:recompute}. The detail of algorithm ComputeGraph is presented
in Algorithm ~\ref{alg:graphcomputing}(\S\ref{subsec:graphcomputing}), described
later.

The exact searches and queries performed on the graph depend on the bug under
investigation. Consider a common performance bug on macOS, the \emph{spinning
cursor}, which indicates the current application's main thread has not processed
any UI events for over two seconds. \xxx queries the event graph to find the
ongoing event in the application's main thread concurrent to the display of
the spinning cursor shown in line~\ref{algmain:findspinningnode}. Upon examining
what the main thread is actually doing, the user may encounter three potential
cases. First, the thread may be busy performing lengthy CPU operations (which
take longer than two seconds). Second, the thread may be in a blocking wait.
Third, the main thread may be in a yield loop, which is highly indicative
it is waiting on a data flag (\eg, ``while(!done) thread\_switch();'').
Line ~\ref{algmain:1st} to ~\ref{algmain:2and3end} is to track events and
synchronization primitives throughout the system to figure out the root cause,
in terms of user input events and bug in code, for the three cases.

\paragraph{B1}

For the first case, shown in line~\ref{algmain:1st}, \xxx will examine
the stack trace and find a sequence of events that led to the current CPU
processing. If more specific tracing is required, the user can rerun the
program with more heavyweight instrumentation enabled for any portion of the
code's execution, gathering a precise sequence of calls or even instructions
executed.

In other cases, \xxx locates the node $N_1$ for the rest
cases, either a blocking wait or a yield loop, in our graph
when the spinning beachball is triggered, as is shown between
line~\ref{algmain:2and3begin} and line~\ref{algmain:2and3end}. Then \xxx applies
Algorithm~\ref{alg:alg-findsimilarnode} to locate a node $N_0$ in our graph that
corresponds to the baseline version of the hanging node $N_1$.

\paragraph{B2}

For the second case of block waiting, \xxx takes the assumption there usually
exits a chain of blocking thread and the initial blocking thread is the most
valuable for debugging. Consequently, \xxx diffs the subgraphs between the
blocking version and the baseline version to determine their difference with
Algorithm ~\ref{alg:alg-graphdiff}, described later.

\paragraph{B3}

In the third case, the main thread is in a yield loop, which is
highly indicative it is waiting on a data flag (\eg, ``while(!done)
thread\_switch();''). To discover a data flag, the user re-runs the application
with \xxx to collect instruction traces of the concurrent event in both the
normal and spinning cases and detects where the control flow diverges. She
then reruns the application with \xxx to collect register values for the basic
blocks before the divergence and uncovers the address of the data flag. She
then configures \xxx to log accesses to the flag during system-wide tracing.
Finally, she can recursively apply \xxx to further diagnose ``the culprit of the
culprit''.

Based on our results and experience, the first case is the most common, but
the second and third represent more severe bugs. Long-running CPU operations
tend to be more straightforward to diagnose with existing tools. Blocking
or yielding cases involve multiple threads and processes, and are extremely
hard to understand and fix even for the application's original developers.
Therefore, issues remain unaddressed for years and significantly impact the user
experience. Algorithm~\ref{alg:alg-main} is semi-automated but can integrate
user input at each stage to leverage hypotheses or expert knowledge as to why a
hang may occur.

\begin{algorithm}[ht!]
    \caption{\xxx Find similar node algorithm.}
    \label{alg:alg-findsimilarnode}
\begin{algorithmic}[1]
%%  Output: Similar node set
\Require{EventTypesSet + SpinningNode + Graph}
\Ensure{SimilarNodeSet}
\Statex
\Function{FindSimilarNode}{}
\State{$thread$ $\gets$ thread of $Node_{blocking}$}
\For {$Node_i$ in $thread$ other than $Node_{blocking}$}
%%	\State{$iter_i$ $\gets$ iterator of first event in $Node_i$}
%%	\State{$iter_b$ $\gets$ iterator of first event in $Node_{blocking}$}
%%	\While {$iter_b$ $\ne$ end \And $iter_i$ $\ne$ end}
%%		\While {$iter_i$ $\ne$ end \And $Event$($iter_i$) $\notin$ $EventTypesSet$}
%%			\State{$iter_i$++}
%%		\EndWhile
%%		\While {$iter_b$ $\ne$ end \And $Event$($iter_b$) $\notin$ $EventTypesSet$}
%%			\State{$iter_b$++}
%%		\EndWhile
	\For{$iter_i$ $\gets$ event iter in $Node_i$}
		\State{$iter_b$ $\gets$ NextEventInEventTypeSet($Node_{blocking}$)}
		\State{$iter_i$ $\gets$ NextEventInEventTypeSet($Node_{blocking}$)}
		%%\If {$iter_b$ $\neq$ end \And $iter_i$ $\neq$ end}
		\If{CompareEvent($Event$($iter_b$), $Event$($iter_i$)) $==$ False}
			%%\If {$if_eq$ $==$ False}
			\State{Break}
		\EndIf
		%%\EndIf
	\EndFor
	\If {$iter_b$ $==$ end \And $iter_i$ $==$ end}
		\If {wresult of $Node_i$ $\neq$ wresult of $Node_{blocking}$
		\Or timespan of $Node_{blocking}$ - timespan of $Node_i$ $\geq$ threshhold}
			\State{Put $Node_i$ into $SimilarNodeSet$}
		\EndIf
	\EndIf
\EndFor
\EndFunction
\end{algorithmic}
\end{algorithm}

\begin{algorithm}[ht!]
    \caption{\xxx Assisted graph diff algorithm.}
    \label{alg:alg-graphdiff}
\begin{algorithmic}[1]
\Require{SpinningNode + BaselineNode + Graph}
\Ensure{Possible root cause of thread blocking}
\Statex
\Function{AssistedGraphDiff}{}
\State{$ResultSet$ $\gets$ $\{\}$}
\State{$SlicingPath$ $\gets$ BackwardSlicing on $BaseLineNode$}
\State{$T_{spinning}$ $\gets$ timestamp of spinning beachball for $SpinningNode$}
\For{$thread$ $\gets$ thread of $Node_i$ in $SlicingPath$}
	\State{$T_{normal}$ $\gets$ timestamp of last event in $Node_i$}
	\State{$Node_{blocking}$ $\gets$ $thread$ blocks during ($T_{normal}$, $T_{spinning}$)}
	\If {$Nodeblocking$  $\neq$ $NULL$}
		\State{put $Node_{blocking}$ into $ResultSet$}
	\EndIf
	\If {$UI Events$ in Main UI thread}
		\State{put $UI Event$ into $ResultSet$}
	\EndIf
\EndFor
\State{Return $ResultSet$}
\EndFunction
\end{algorithmic}
\end{algorithm}


\subsection{\xxx Assisting Algorithm}

We mentioned previously in \S\ref{subsec:debug} that \xxx explores the event
graph semi-automatically to narrow in on relevant nodes in event graph for
diagnosis. Algorithms that require no user intervention are described
in this section.

\paragraph{Find Similar Node Algorithm}

FindSimilarNode in Algorithm~\ref{alg:alg-findsimilarnode} is meant to find
nodes which have the same high level semantics but different execution
results compared to the spinning node. For example, in both cases a thread
waits on a lock, but one returns successfully and the other not. \xxx
defines a subset of events that preserve semantics, which consists of
System\_call, Back\_trace, NSApp\_event, Mach\_message, Wait, Dispatch\_enqueue,
Dispatch\_invoke, Runloop\_submit, Runloop\_invoke, Share\_flag\_read and
Share\_flag\_write. The comparison of events depends on event type and its
category in Table~\ref{table:event_types}, described in \S~\ref{sec:eventgraph}.
For connection events, \xxx compares their peers. For semantics Events, \xxx
compares the content of the event attributes, for example, the system call
number. The wait results of wait event usually differentiate the normal
execution and buggy case. \xxx differentiate the cases with wait results by
default. Depending on the report detail of the spinning node, user can change
the comparing metrics or request \xxx to compare proceeding nodes to narrow down
the result set reported.

\paragraph{Graph Diff Algorithm}

The Algorithm~\ref{alg:alg-graphdiff} is a single step for the semi-automatic
debugging phase. It is design for figure out the root cause for
\textbf{B2}(\S\ref{subsec:debug}). \xxx assumes that the blocking in the main
thread is usually caused by the unresponsiveness of another thread. In the
algorithm, with the input of the blocking node in main thread, \xxx carries
out a backward path slicing from the node from the baseline case in the graph.
The path consists of nodes from multiple threads. \xxx traverses every thread
and confines the search within the time interval from the baseline node in the
thread and the timestamp when the blocking node appears in main thread. \xxx
expects to find a blocking call in some threads and reports such cases. At the
same time \xxx also collects the UIEvents into return set.

\subsection{Chromium Spinning Cursor Example}

%   resolve symbol, save to log
%     search for set\_spinning
%     if found,
%       find main thread node at the time of set\_spinning
%     find fontd
%     manually check nodes in each thread immediately after the nodes in the slice (normal abnormal boundary)
%  output is a node, and a HTML dump of node and immediate predecessors and successors
% systems preference          
%  spinning node in UI thread is not waiting
%  so we look for messages, and find the diff

One of the authors experienced first-hand the aforementioned performance issue
in Chromium, an open-source browser engine that powers Google Chrome and,
starting recently, Microsoft Edge~\cite{chromiumurl}.  She tried to type in the
Chromium search box a Chinese word using SCIM, the default Chinese Input Method
Editor that ships with MacOS.  The browser appeared frozen and the spinning
cursor occurs for a few seconds.  Afterwards everything went back to normal.
This issue is reproducible and always ruins her experience, but it is quite
challenging to diagnose because two applications Chromium and SCIM and many
daemons ran and exchanged messages.  This issue was reported by other users for
other non-English input methods, too.

To diagnose this issue with \xxx, the author started system-wide tracing, and
then reproduced the spinning cursor with a Chinese search string typed via SCIM
while the page was loading. It produced normal cases for the very first few
characters, and the browser got blocked with the rest input as spinning cases.
The entire session took roughly five minutes.

She then ran \xxx to construct the event graph. The graph had 2,749,628 vertexes
and 3,606,657 edges, almost fully connected. It spans across 17 applications;
109 daemons including \vv{fontd}, \vv{mdworker}, \vv{nsurlsessiond} and
helper tools by applications; 126 processes; 679 threads, and 829,287
IPC messages. Given the scale of the graph and the diverse communication
patterns, it would be extremely challenging for prior automated causal tracing
tools~\cite{aguilera2003performance, zhang2013panappticon, attariyan2012x,
cohen2004correlating} because they handle a fairly limited set of patterns.
Tools that require manual schema~\cite{barham2004using, reynolds2006pip}, would
be prohibitive because developers would have to provide schema for all involved
applications and daemons.

Next she ran \xxx to find the spinning node in the main thread of the browser
process. \xxx returned a \vv{Wait} event on condition variable with timeout that
blocked the main thread for a few seconds. Thus \xxx compares the spinning node
to a similar one in normal case where the \vv{Wait} was signaled quickly with
Algorithm~\ref{alg:alg-findsimilarnode}. \xxx reported three, and confirmed with
the user which one she wanted.

\xxx then found the normal-case wake-up path which connects five threads. The
browser main thread was signaled by a browser worker thread, which received IPC
from a worker thread of \vv{renderer} where the rendering view and WebKit code
run. The worker thread is woken up by the \vv{renderer} main thread, which in
turn woken by fontd, the font service daemon. \xxx further compared the wake-up
path with the spinning case with the Algorithm ~\ref{alg:alg-graphdiff}, and
returned the \vv{wait} event on semaphore in the \vv{renderer} main thread, the
culprit that delayed waking up the browser main thread over 4 seconds.

What caused the wait in the \vv{renderer} main thread though? She thus continued
diagnosis and recursively applied \xxx to the wait in \vv{renderer}, and got
the wake-up path. The culprit that delayed the semaphore was the timeouts in
the browser's main thread. At this point, a circular wait formed. To understand
what exactly happens in the situation, she inspected the full call stacks by
\xxx scripts, taking the reported nodes from the renderer and the browser as
input. Inspection reveals that the \vv{renderer} requested the browser's help to
render Javascript and waited for reply with semaphore. The browser was waiting
for the \vv{renderer} to return the string bounding box and the \vv{renderer}
was waiting for the browser to help render Javascript. This circular wait was
broken by a timeout in the browser main thread (the \vv{wait} on cv timeout was
1,500 ms). While the system was able to make progress, the next key press caused
the spinning cursor to display for another 1,500 ms. The timeout essentially
converted a deadlock into a livelock.

The finding was verified with chromium source code. Shortening the timeout interval
in the main browser thread proportionally shortens the waiting of the main
render thread on processing the javascript. Skipping certain javascripts
processing in the renderer thread cuts down the success rate of spinning case
reproducing.

%%Next she ran \xxx to find the event in the main thread of the browser process.
%%\xxx returned a \vv{cv\_timed\_wait} event %(Figure~\ref{fig:chromium-trace})
%%that blocked the main thread for a few seconds. Inspection of the
%%lightweight call stack revealed that this wait happened within a call to
%%\vv{TextInputClientMac::GetFirstRectForRange}. Without knowing the application's
%%semantics, she could not understand this method. Thus she ran \xxx to compare
%%the spinning case to a normal case. \xxx searched in the main thread of the
%%browser process for vertexes similar to this wait waiting vertexes similar to
%%this wait, found three, and confirmed with the user which one she wanted.
%%
%%\begin{figure*}[p]
%%    \centering
%%	%%\documentclass{article}
%%\usepackage{tikzpeople}
%%\usepackage{tikz}
%%\usetikzlibrary{shapes, shapes.misc}
%%\usetikzlibrary{arrows, arrows.meta, decorations.markings}
%%\usetikzlibrary{patterns}

%%\begin{document}
\begin{center}
\resizebox{\columnwidth}{!}{%
\begin{tikzpicture}[>=latex]
\tikzstyle{every node}=[font=\Large]
\tikzstyle{app} = [draw, very thick, minimum height=40em, minimum width=20em, fill=white, rectangle, font={\sffamily\bfseries}]
\tikzstyle{actionpoint} = [minimum width = 1em, fill = white]
\tikzstyle{point} = [thick, draw=red, cross out, inner sep = 0pt, minimum width = 0.5em, minimum height =0.5em]
%\tikzset{
%%ext/.pic={
%%\path [fill=white] (-0.2,0)to[bend left](0,0.1)to[bend right](0.2,0.2)to(0.2,0)to[bend left](0,-0.1)to[bend right](-0.2,-0.2)--cycle;
%%\draw (-0.2,0)to[bend left](0,0.1)to[bend right](0.2,0.2) (0.2,0)to[bend left](0,-0.1)to[bend right](-0.2,-0.2);
%%},
%point/.style={
%    thick,
%    draw=red,
%    cross out,
%    inner sep=0pt,
%    minimum width=4pt,
%    minimum height=4pt,
%}}

%draw applications
\node [app, name = browser]{};
\node (browsertag)[minimum width = 20em, above = -2em of browser] {Chromium Browser};
\node (browserend)[minimum width = 20em, below = 38em of browsertag] {};
\node (bt1begin)[minimum width = 8em, below left = 3em and -10em of browsertag]{main thread};
\node (bt1end)[minimum width = 8em, below left = 38em and -10em of browsertag]{};
\node (bt2begin)[minimum width = 8em, below right = 3em and -10em of browsertag]{worker thread};
\node (bt2end)[minimum width = 8em, below right = 38em and -10em of browsertag]{};
\draw [solid] (bt1begin) -- (bt1end);
\draw [solid] (bt2begin) -- (bt2end);
\node (1)[actionpoint, below of = bt1begin, node distance = 8em]{1};
\node (2)[actionpoint, below of = bt2begin, node distance = 9em]{2};
\node (9)[actionpoint, below of = 2, node distance = 16em] {9};
\node (10)[actionpoint, below of = 1, minimum width = 6em, minimum height = 2em, node distance = 18em, rotate=270]{timed wait};
%\node (11)[actionpoint, below of = 10, node distance = 4em]{};

\node [app, name = renderer, right of = browser, node distance = 25em]{};
\node (renderertag)[minimum width = 20em, above = -2em of renderer] {Chromium Renderer};
\node (rendererend)[minimum width = 20em, below = 38em of renderertag]{};
\node (bt3begin)[minimum width = 8em, below left = 3em and -10em of renderertag]{worker thread};
\node (bt3end)[minimum width = 8em, below left = 38em and -10em of renderertag]{};
\node (bt4begin)[minimum width = 8em, below right = 3em and -10em of renderertag]{main thread};
\node (bt4end)[minimum width = 8em, below right = 38em and -10em of renderertag]{};
\draw [solid] (bt3begin) -- (bt3end);
\draw [solid] (bt4begin) -- (bt4end);
\node (3)[actionpoint, below of = bt3begin, node distance = 10em]{3};
\node (4)[actionpoint, below of = bt4begin, node distance = 11em]{4};
\node (7)[actionpoint, below of = 4, node distance = 11em, rotate=270] {sema};
\node (8)[actionpoint, below of = 3, node distance = 13em]{8};

\node [app, name = fontd, right of = renderer, node distance = 25em]{};
\node (fontdtag)[minimum width = 20em, above = -2em of fontd] {fontd};
\node (fontdend)[minimum width = 20em, below = 38em of fontdtag]{};
\node (bt5begin)[minimum width = 20em, below = 3em of fontdtag]{worker thread};
\node (bt5end)[minimum width = 20em, below = 38em of fontdtag]{};
\draw [solid] (bt5begin) -- (bt5end);
\node (5)[actionpoint, below of = bt5begin, node distance = 14em] {5};
\node (6)[actionpoint, below of = 5, node distance = 3em] {6};

\draw [->] (1) -- (2);
\draw [->] (2) -- (3);
\draw [->] (3) -- (4);
\draw [->] (4) -- (5);
\coordinate[name = notarrive7, point, right of = 7, node distance = 1em];
\draw [->] (6) -- (notarrive7);
\coordinate[name = before7, above of = 7, node distance = 1em]{};
\draw [->] (before7) -- (8);
\draw [->] (8) -- (9);
\coordinate[name = notarrive10, point, right of = 10, node distance = 1em]{};
\draw [->] (9) -- (notarrive10);

\node (findfirstrect)[minimum width = 40em, minimum height = 4em, right of = 1, node distance = 13em, rotate = 355] {TextInputClientMsg\_FirstRectForCharacterRange};
\node (javascript)[minimum width = 4em, minimum height = 4em, above left = 0em and 2em of 7, rotate = 15] {Run\_Javascript};
\end{tikzpicture}
}
\end{center}
%%\end{document}
 
%%    \caption{Chromium case study.}
%%    \label{fig:chromium-trace}
%%\end{figure*}
%%%need to adjust based on the new figure
%%\xxx then found the normal-case wake-up path shown in the figure, which
%%connects five threads.  The browser main thread was signaled by a browser
%%worker thread as shown in step \textcircled{1} of backward slicing in Figure
%%\ref{fig:chromium-trace}, which in turn \vv{read\_file} in step \textcircled{2}
%%for IPC from a worker thread of \vv{renderer}, the daemon for rendering screens.
%%The \vv{renderer} worker thread is woken up by the \vv{renderer} main thread to
%%\vv{read\_file} \textcircled{3}, which in turn \vv{recv\_msg} \textcircled{4}
%%from \vv{fontd}, the font service daemon.  From this path, we could guess that
%%\vv{GetFirstRectForRange} was for the browser to understand the bounding box of
%%the search string.  \xxx further compared the wake-up path with the spinning
%%case, and returned the \vv{wait\_semaphore} event in the \vv{renderer} main
%%thread, the culprit that delayed waking up the browser main thread over 4
%%seconds.

%%What caused the wait in the \vv{renderer} main thread though?  She thus
%%continued diagnosis and recursively applied \xxx to the wait in \vv{renderer},
%%and got the wake-up path shown in the figure for this wait.  Inspection reveal
%%that the \vv{renderer} requested the browser's help to render Javascript and was
%%waiting for a reply.  At this point, a circular wait formed because the browser
%%was waiting for the \vv{renderer} to return the string bounding box and the
%%\vv{renderer} was waiting for the browser to help render Javascript.  This
%%circular wait was broken by a timeout in the browser main thread (the
%%\vv{cv\_timed\_wait} timeout was 1,500 ms).  While the system was able to make
%%progress, the next key press caused the spinning cursor to display for another
%%1,500 ms.  The timeout essentially converted a deadlock into a livelock.
