\section{Overview}\label{sec:overview}

\subsection{Argus Work Flow}
\begin{figure}[tb]
    \centering
	%\documentclass{article}
%\usepackage{tikzpeople}
%\usepackage{tikz}
%\usepackage{graphicx}
%\usetikzlibrary{shapes, shapes.misc}
%\usetikzlibrary{arrows, arrows.meta, decorations.markings}
%\usetikzlibrary{patterns}
%\usetikzlibrary{positioning}


\makeatletter
% copied from the manual
\pgfdeclareshape{document}{%
    \inheritsavedanchors[from=rectangle] % this is nearly a rectangle
    \inheritanchorborder[from=rectangle]
    \inheritanchor[from=rectangle]{center}
    \inheritanchor[from=rectangle]{north}
    \inheritanchor[from=rectangle]{south}
    \inheritanchor[from=rectangle]{west}
    \inheritanchor[from=rectangle]{east}
	\inheritanchor[from=rectangle]{south west}
	\inheritanchor[from=rectangle]{north west}
	\inheritanchor[from=rectangle]{south east}
	\inheritanchor[from=rectangle]{north east}
    % ... and possibly more
    \backgroundpath{% this is new
        % store lower right in xa/ya and upper right in xb/yb
        \southwest \pgf@xa=\pgf@x \pgf@ya=\pgf@y
        \northeast \pgf@xb=\pgf@x \pgf@yb=\pgf@y
        \def\hangout{5pt}
        % compute corner of "flipped page"
        \pgf@xc=\pgf@xb \advance\pgf@xc by-\hangout % this should be a parameter
        \pgf@yc=\pgf@yb \advance\pgf@yc by-\hangout
        % construct main path
        \pgfpathmoveto{\pgfpoint{\pgf@xa}{\pgf@ya}}
        \pgfpathlineto{\pgfpoint{\pgf@xa}{\pgf@yb}}
        \pgfpathlineto{\pgfpoint{\pgf@xc}{\pgf@yb}}
        \pgfpathlineto{\pgfpoint{\pgf@xb}{\pgf@yc}}
        \pgfpathlineto{\pgfpoint{\pgf@xb}{\pgf@ya}}
        \pgfpathclose
        % add little corner
        \pgfpathmoveto{\pgfpoint{\pgf@xc}{\pgf@yb}}
        \pgfpathlineto{\pgfpoint{\pgf@xc}{\pgf@yc}}
        \pgfpathlineto{\pgfpoint{\pgf@xb}{\pgf@yc}}
        \pgfpathlineto{\pgfpoint{\pgf@xc}{\pgf@yc}}
        \pgfpathclose
        % add lines
        %\pgfmathsetmacro{\step}{round(abs(\pgf@yb-\pgf@ya)/6)}
        %\pgfmathsetmacro{\lines}{round(abs(\pgf@yb-\pgf@ya)/\step)-1}
        %\foreach \y in {1,...,\lines}{%
        %    \pgfpathmoveto{\pgfpoint{\pgf@xa+\step*1pt+\hangout}{\pgf@yb-\y*\step*1pt}}
        %    \pgfpathlineto{\pgfpoint{\pgf@xc-\step*1pt}{\pgf@yb-\y*\step*1pt}}
        %}%  
    }%
}%
\makeatother

%\begin{document}
\begin{center}
\resizebox{\columnwidth}{!}{%
\begin{tikzpicture}[>=latex]
\tikzstyle{every node}=[font=\Large]
\tikzstyle{apps} = [draw, very thick, minimum height=30pt, minimum width=50pt, fill=white, rectangle, font={\sffamily\bfseries}]
\tikzstyle{systemComp} = [draw, very thick, minimum height=30pt, minimum width=150pt, pattern= north west lines, pattern color = black!20!, rectangle, font={\sffamily\bfseries}]
\tikzstyle{doc} = [draw, thick, shape=document, align=left, color=black, minimum width=150pt, minimum height=40pt, inner sep=2ex]
\tikzstyle{vecarrow} = [->, thick, line width=5pt]
\tikzstyle{gthread} = [label={[font=\small, red]below:#1}]
\tikzstyle{gnode} = [draw, thick, font=\small, align =left, minimum width=18pt, minimum height=36pt]

\tikzset{ext/.pic={
\path [fill=white] (-0.2,0)to[bend left](0,0.1)to[bend right](0.2,0.2)to(0.2,0)to[bend left](0,-0.1)to[bend right](-0.2,-0.2)--cycle;
\draw (-0.2,0)to[bend left](0,0.1)to[bend right](0.2,0.2) (0.2,0)to[bend left](0,-0.1)to[bend right](-0.2,-0.2);
}}

%%draw MacOS components
\node [apps, name = software1] at (0, 0) {\includegraphics[width=20pt]{./chromium_logo.png}};
\node [apps, name = software2, right of=software1, node distance=50pt] {daemon1};
\node [apps, name = software3, right of=software2, node distance=50pt] {daemon2};
\node [systemComp, name = libs, below of = software2, node distance =30pt] {Argus Instrumented Libraries};
\node [systemComp, name = os, below of = libs, node distance =30pt] {Argus Instrumented OS};

%% draw Tracing Event log
\node (TracingEventLog) [minimum height=30pt, minimum width=200pt, right of=software3, node distance=220pt]{};
\node (log) [doc, below of=TracingEventLog, node distance=30pt] {\#timestamp, event\_type, attr1, attr2...\\30.4, sendMessge, tid1, tid2 ..\\31.7, wakeUp, tid0, tid2...\\33.2, wakeUp, tid2, tid3...\\34.2, sendMessage, tid3, tid2...\\34.9, wakeUp, tid2, tid0...\\ ... ... };
\draw [vecarrow] (libs.east)+(0.5, 0) -- (log.west);

%draw Interactive Debugger 
\node (InteractiveDebugger) [systemComp, rounded corners=20pt, minimum height=280pt, minimum width=300pt, below left = 20pt and -160pt of log] {};
\node (debugger)[minimum width=220pt, above = -30pt of InteractiveDebugger]{Interactive Debugger};
\node (graph) [minimum height = 220pt, minimum width = 220pt, below = 10pt of debugger] {};
\node (shape) [doc, minimum height = 220pt, minimum width = 220pt, below = 10pt of debugger]{};
%\node [minimum width = 100pt, above = -20pt of graph, label=right:{graph}]{};
\node(t0s)[gthread = browser, minimum width = 2pt, below left = -200pt and -50pt of graph, font=\small]{tid0};
\node(t0e)[gthread = browser, minimum width = 2pt, below left = -20pt and -44pt of graph]{};
\node(t1s)[gthread = kern\_task, minimum width = 2pt, below left = -200pt and -85pt of graph, font=\small]{tid1};
\node(t1e)[gthread = kern\_task, minimum width = 2pt, below left = -20pt and -79pt of graph]{};
\node(t2s)[gthread = renderer, minimum width = 2pt, below left = -200pt and -120pt of graph, font=\small]{tid2};
\node(t2e)[gthread = renderer, minimum width = 2pt, below left = -20pt and -114pt of graph]{};
\node(t3s)[gthread = renderer, minimum width = 2pt, below left = -200pt and -155pt of graph, font=\small]{tid3};
\node(t3e)[gthread = renderer, minimum width = 2pt, below left = -20pt and -149pt of graph]{};
\node(t4s)[gthread = fontd, minimum width = 2pt, below left = -200pt and -185pt of graph, font=\small]{tid4};
\node(t4e)[gthread = fontd, minimum width = 2pt, below left = -20pt and -179pt of graph]{};

\node(brw0) [gnode, fill = blue!20, below = 20pt of t0s]{msg\\msg};
\node(brw1) [gnode, below = 15pt of brw0]{write\\msg};
\node(brw2) [gnode, below = 20pt of brw1]{read\\write};
\node(k0) [gnode, fill = red!20, below = 25pt of t1s] {msg};
\node(k1) [gnode, below = 50pt of k0] {msg\\wait};
\node(rdw0) [gnode, fill = yellow!20, below = 30pt of t2s]{msg\\read\\msg};
\node(rdw1) [gnode, below = 30pt of rdw0]{msg\\msg};
\node(rdm0) [gnode, below = 40pt of t3s]{read\\msg};
\node(rdm1) [gnode, minimum height = 20pt, below = 20pt of rdm0]{write\\msg};
\node(rdm2) [gnode, thick, fill = blue!20, minimum height = 10pt, below = 18pt of rdm1, font=\small]{msg\\sema\_wait};
\node(fontd0) [gnode, minimum height=20pt, below = 50pt of t4s]{msg\\msg};

\draw[-, solid](t0s.south) -- (brw0);
\draw[-, dotted](brw0) -- (brw1);
\draw[-, dotted](brw1) -- (brw2);
%\draw[-, solid](brw2) --(t0e.south);
\draw[-, solid](t1s.south) -- (k0);
\draw[-, dotted](k0) -- (k1);
\draw[-, solid](k1) -- (t1e.south);
\draw[-, solid](t2s.south) -- (rdw0);
\draw[-, dotted](rdw0) -- (rdw1);
\draw[-, solid](rdw1) -- (t2e.south);
\draw[-, solid](t3s.south) -- (rdm0);
\draw[-, dotted](rdm0) -- (rdm1);
\draw[-, solid](rdm1) --pic{ext}(rdm2);
\draw[-, solid](t4s) -- (fontd0);
\draw[-, dotted](rdm2) -- (t3e.south);
\draw[-, solid](fontd0) --(t4e.south);

\draw[->] (brw0.east) -- node[font=\small, minimum width=5pt, pos=0.1, below]{e0} (rdw0.west);
\draw[->] (k0.east) -- node[font=\small, minimum width=5pt, above]{e1} (rdw0.west);
\draw[->] (rdw0.east) -- (rdm0.west);
\draw[->] (rdm0.east) -- (fontd0.west);
\draw[->] (fontd0.west) -- (rdm1.east);
\draw[->] (rdm1.west) -- (rdw1.east);
\draw[->] (rdw1.west) -- (brw2.east);


\draw [vecarrow] (log.south) -- (InteractiveDebugger);

\node (legend)[minimum height =180pt, minimum width = 80pt, right of = InteractiveDebugger, node distance= 200pt]{};
\node (1)[below left = -20pt and -20pt of legend] {};
\node (2)[right=8pt of 1]{};
\draw [-, solid] (1.east) -- (2);
\node [right= 10pt of 1, font=\small]{thread\_running};

\node (3)[below left = -30pt and -20pt of legend] {};
\node (4)[right=8pt of 3]{};
\draw [-, dotted] (3.east) -- (4);
\node [right =10pt of 3, font=\small]{thread\_wait};

\node (5)[below left = -40pt and -20pt of legend] {};
\node (6)[right =8pt of 5]{};
\draw [->] (5.east) -- (6);
\node [right =10pt of 5, font=\small]{edge};

\node [draw, solid, minimum width = 10pt, minimum height = 3pt, below left = -50pt and -30pt of legend, label={[font=\small]right:node}]{};

%draw interactive Debugging part
\umlactor[left = 100pt of InteractiveDebugger, scale=2]{user};
\node (user)[left = 100pt of InteractiveDebugger]{};
%\node (query1)[above left = -80pt and -10pt of InteractiveDebugger]{};
%\node (query2)[above left = -100pt and -10pt of InteractiveDebugger]{};
\node (query3)[below left = -100pt and -10pt of InteractiveDebugger]{};
\node (query4)[below left = -80pt and -10pt of InteractiveDebugger]{};
%\draw [->, thick, out=160, in=60] (query1.north west) to node{which node?} (user.north);
%\draw [->, thick, out=350, in=230] (user.north) to node{node 5} (query2.north west);
\draw [->, thick, out=150, in=30] (query3.north west) to node {\includegraphics[width=60pt]{./graph_snippet.png}} (user.south);
\draw [->, thick, out=310, in=230] (user.south) to node[above]{e0} (query4.north);

\node (result) [draw, dotted, minimum height = 100pt, minimum width = 400pt,pattern= north west lines, pattern color = black!30, below = 30pt of InteractiveDebugger]{};
\node (blockrenderer) [draw, solid, minimum height = 10pt, fill=blue!20, below left = -40pt and -200pt of result, label=below:render thread blocks on semaphore]{msg sema\_wait};
\node (blockbrowser) [draw, solid, minimum height = 10pt, below left = -80pt and -200pt of result, label=below:browser thread blocks on condition variable]{msg cv\_wait};

%\node (text1)[below = 50pt of = blockrenderer] {renderer thread \\blocks on semaphore};
%\node (text1)[below = 50pt of = blockbrowser] {browser thread \\blocks on conditional variable};

\draw [vecarrow] (InteractiveDebugger.south) -- (result);
\end{tikzpicture}
}
\end{center}
%\end{document}

    %\includegraphics[width=\columnwidth]{figures/overview_tikz.png}
    \caption{\xxx Work Flow}
    \label{fig:argus-overview}
\end{figure}

In this section, we describe the steps a user takes to investigate a performance
anomaly with \xxx. Figure~\ref{fig:argus-overview} shows \xxx's work flow with
an example of a user investigating a performance problem in Chromium. The
system wide tracing tool, which collects data from \xxx instrumented library
and kernel, generates logs. They are transformed into an graph in \xxx's graph
construction component. Central to our system is our \emph{event graph}, a
generalized control-flow graph which includes inter-thread and inter-process
dependencies. Diagnosis and inferences are performed within this graph, in a
semi-automated fashion: \xxx performs searches to trace logical events as they
flow through the system, and it judicially queries the user for guidance.
%The generated graph is used by the interactive debugger for causal path
%slicing and diagnosis. \xxx supports interactive search in path slicing, by
%providng information to the user and asking for decision, in case of multiple
%predecessors in a vertex.
As shown in Figure~\ref{fig:argus-overview}, the debugger asks user to choose
one edge in a subgraph. After this step, \xxx performs diagnosis algorithm and
reports the root cause vertices. In the example, the root cause is two vertices
which form a circular wait across multiple threads.
%
%In this section, we describe the steps a user takes to investigate a performance
%anomaly with \xxx. Figure~\ref{fig:argus-overview} shows \xxx's work flow, which
%consists of two phases. A user runs command ``\vv{\xxx start}'' to enter the
%system-wide tracing phase, within which \xxx logs events as listed in Table
%~\ref{table:event_types} (\S\ref{subsec:eventgraph}). Whenever a user detects
%a performance issue such as a spinning cursor, she runs ``\vv{\xxx debug}'' to
%enter the diagnosis phase.
%
%%In the cases we consider, the flow of information across threads and processes
%%is essential to discovering the system state that leads to a performance bug.
%%\xxx recovers UI actions from logged data rather than being told the actions
%%that a user performs, because not all of them may be relevant to the true bug.
%%
%Central to our system is our \emph{event graph}, a generalized control-flow
%graph which includes inter-thread and inter-process dependencies. 
%Diagnosis
%and inferences are performed within this graph, in a semi-automated fashion:
%\xxx performs searches to trace logical events as they
%flow through the system, and it judicially queries the user for guidance.
% The event graph is described in further detail in \S\ref{subsec:eventgraph}.

Next, we describe how \xxx assists the user to diagnose a performance issue.

%fwd ref section on chromium, but first, how search event graph

\subsection{Diagnosis with Graph}\label{subsec:debug}

\begin{algorithm}
\caption{Diagnosis algorithm.}
\label{alg:alg-diagnosis}
\begin{algorithmic}[1]
\Require{g - EventGraph; \algspinningnode - the vertex in the UI thread when the spinning cursor occurs}
\Ensure{\algrootcausenodes-collecting root casuse vertices for user inspect}
\Statex
\Function{Diagnose}{g, \algspinningnode} \label{alg:diagnose-line}%\funclabel{alg:diagnose} \label{alg:diagnose-line}
  \Switch {\algspinningnode.block\_type}
    \Case {LongRunning} \label{a1:longrunning_begin}
		\State {slice $\gets$ InteractiveSlice(\algspinningnode)}
		\State\Return {vertex contains UI event} \label{a1:longrunning_end}
	\EndCase
	\Case {RepeatedYield} \label{a1:repeatedyield_begin}
		\If {DataFlagEvent $\notin$ $\{$event types in \algspinningnode$\}$}
			\State {Require users to annotate data flag}
			\State {abort()}
		\EndIf
		\LineComment {Fall through}\label{a1:repeatedyield_end}
	\EndCase
	\Case {LongWait} \label{a1:longwait_begin}
		\State {\algsimilarnode $\gets$ vertex has similar event sequence to \algspinningnode}
		%\State {baseline\_vertex $\gets$ ask user to pick from \algsimilarnodes} 
		\State {baseline\_path $\gets$ InteractiveSlice(\algsimilarnode)} 	
		\For {each t in $\{$threads in baseline\_path$\}$}
			\State {$vertex_t$ $\gets$ vertex in t before \algspinningnode gets spinning}
			\If {$vertex_t$ $\in$ $\{$LongRunning, RepeatedYield, LongWait$\}$}
				%\State{is\_related\_to\_bug $\gets$ ask user if $node_t$ related to bug}
				%\If {is\_related\_to\_bug $\not=$ True}
				%	\State{Continue}
				%\EndIf
				\State {\algrootcausenodes.append($vertex_t$)}
				\State {\algrootcausenodes.append(Diagnose(g, $vertex_t$)}
			\EndIf
			\LineComment{if t is normal running, disgnose the next thread}
		\EndFor \label{a1:longwait_end}
	\EndCase
  \EndSwitch
  \State\Return{\algrootcausenodes}
\EndFunction
%%\end{algorithmic}
%%\end{algorithm}

%\begin{algorithm}
%\caption{InteractiveSlicing algorithm.}
%\label{alg:alg-interactiveslicing}
%\begin{algorithmic}[1]
%\Require{g - EventGraph; node - the node user wants to slice from}
%\Ensure{slice - causual path for node}
\Statex
\Function{InteractiveSlicing}{g, vertex}\label{alg:interslice-line}
%\funclabel{alg:interactiveslicing}
%\label{alg:interactiveslicingline}
\Loop
	\State{path\_slice.append(vertex)}
	\If {vertex has 1 incoming edge}
		\State{vertex $\gets$ predecessor vertex}
	\ElsIf {vertex has multiple incoming edges}
		\State{vertex $\gets$ ask user to pick from predecessors}
	\ElsIf {vertex had weak edges}
		\State{vertex $\gets$ ask user to pick from predecessors}
	\Else \LineComment{The first vertex of current thread}
		\State\Return {path\_slice}
	\EndIf
	\If {vertex is invalid}
	\LineComment{user chooses to stop traversal with invalid input}
		\State\Return {path\_slice}
	\EndIf
\EndLoop
\EndFunction
\end{algorithmic}
\end{algorithm}


Consider a common performance bug on macOS, the \emph{spinning cursor},
which indicates the current application's main thread has not processed
any UI events for over two seconds.  To initialize debugging a spinning
cursor, \xxx first constructs an event graph from the system-wide event
log recorded.  It then queries the event graph to find the ongoing event
in the application's main thread concurrent to the display of the spinning
cursor.  Given the event graph and the \spinningnode, \xxx runs
Algorithm~\ref{alg:alg-diagnosis} to interactively pinpoint the root
cause.

Specifically, upon examining what the main thread is actually doing, there
are three potential cases.

\begin{itemize}

	\item \textbf{LongRunning} (lines~\ref{a1:longrunning_begin}
		-~\ref{a1:longrunning_end}). The main thread is busy performing lengthy CPU
		operations. This case is the simplest, and \xxx traverses the event graph
		backwards to find a slice originating from the offending UI event to the
		long running CPU operations. This slice is particularly useful for further
		diagnosing the bug. As shown in Function \vv{InteractiveSlicing}, \xxx may
		encounter vertices with multiple incoming edges or weak edges that may not
		reflect causality when traversing the graph. It queries the user to resolve
		them.

	\item \textbf{RepeatedYield} (lines~\ref{a1:repeatedyield_begin}
		-~\ref{a1:repeatedyield_end}). The main thread is in a yield loop, which
		is highly indicative it is waiting on a data flag (\eg, ``while(!done)
		thread\_switch();''). If \xxx cannot find any record of data flags in the
		\spinningnode, it terminates debugging by prompting the user to identify data
		flags and re-trace the application. Here we assume that the performance issue
		reproduces with a reasonable probability because, fortunately, a one-off issue
		that never reproduces is not as annoying as one that occurs frequently. If
		\xxx finds the data flag the \spinningnode is waiting for, it falls through to
		the next case.

	\item \textbf{LongWait} (lines~\ref{a1:longwait_begin}
		-~\ref{a1:longwait_end}). The main thread is in a lengthy blocking wait and
		the wake-up has been missing. \xxx handles this case by finding a baseline
		scenario where the wake-up indeed arrives, and then figures out which wake-up
		edge is missing in the spinning scenario along the expected wake-up path.
		Specifically, \xxx first finds a \similarnode to the spinning one based solely
		on the semantic events such as system calls in each vertex. It then traverses
		backwards from the \similarnode to find the baseline wake-up path. For each
		thread in the wake-up path, it examines the vertex in the thread right before
		the \spinningnode waits. If this vertex is also abnormal, \xxx appends it
		to the path of \rootcausenodes, and applies Function Diagnose recursively
		diagnose ``the culprit of the culprit.'' For each such vertex, it queries the
		user to determine whether to proceed or stop because based on our experience
		the user needs to inspect only a few vertices to find the root cause.

\end{itemize}
\noindent
Based on our results and experience, the first case is the most common, but the
second and third represent more severe bugs. Long-running CPU operations tend to
be more straightforward to diagnose with existing tools such as \spindump except
they do not connect CPU operations back to UI events. Repeated yielding or
long waiting cases involve multiple threads and processes, and are extremely
hard to understand and fix even for the application's original developers.
Therefore, issues remain unaddressed for years and significantly impact the
user experience. Algorithm~\ref{alg:alg-diagnosis} is semi-automated but can
integrate user input to leverage hypotheses or expert knowledge
as to why a hang may occur. Our results show that user inputs, albeit few, are
crucial in this process (\S\ref{sec:casestudy}).

\subsection{Chromium Spinning Cursor Example}

\begin{figure}[tb]
	\footnotesize
    \centering
	%%\documentclass{article}
%%\usepackage{tikzpeople}
%%\usepackage{tikz}
%%\usetikzlibrary{shapes, shapes.misc}
%%\usetikzlibrary{arrows, arrows.meta, decorations.markings}
%%\usetikzlibrary{patterns}

%%\begin{document}
\begin{center}
\resizebox{\columnwidth}{!}{%
\begin{tikzpicture}[>=latex]
\tikzstyle{every node}=[font=\Large]
\tikzstyle{app} = [draw, very thick, minimum height=40em, minimum width=20em, fill=white, rectangle, font={\sffamily\bfseries}]
\tikzstyle{actionpoint} = [minimum width = 1em, fill = white]
\tikzstyle{point} = [thick, draw=red, cross out, inner sep = 0pt, minimum width = 0.5em, minimum height =0.5em]
%\tikzset{
%%ext/.pic={
%%\path [fill=white] (-0.2,0)to[bend left](0,0.1)to[bend right](0.2,0.2)to(0.2,0)to[bend left](0,-0.1)to[bend right](-0.2,-0.2)--cycle;
%%\draw (-0.2,0)to[bend left](0,0.1)to[bend right](0.2,0.2) (0.2,0)to[bend left](0,-0.1)to[bend right](-0.2,-0.2);
%%},
%point/.style={
%    thick,
%    draw=red,
%    cross out,
%    inner sep=0pt,
%    minimum width=4pt,
%    minimum height=4pt,
%}}

%draw applications
\node [app, name = browser]{};
\node (browsertag)[minimum width = 20em, above = -2em of browser] {Chromium Browser};
\node (browserend)[minimum width = 20em, below = 38em of browsertag] {};
\node (bt1begin)[minimum width = 8em, below left = 3em and -10em of browsertag]{main thread};
\node (bt1end)[minimum width = 8em, below left = 38em and -10em of browsertag]{};
\node (bt2begin)[minimum width = 8em, below right = 3em and -10em of browsertag]{worker thread};
\node (bt2end)[minimum width = 8em, below right = 38em and -10em of browsertag]{};
\draw [solid] (bt1begin) -- (bt1end);
\draw [solid] (bt2begin) -- (bt2end);
\node (1)[actionpoint, below of = bt1begin, node distance = 8em]{1};
\node (2)[actionpoint, below of = bt2begin, node distance = 9em]{2};
\node (9)[actionpoint, below of = 2, node distance = 16em] {9};
\node (10)[actionpoint, below of = 1, minimum width = 6em, minimum height = 2em, node distance = 18em, rotate=270]{timed wait};
%\node (11)[actionpoint, below of = 10, node distance = 4em]{};

\node [app, name = renderer, right of = browser, node distance = 25em]{};
\node (renderertag)[minimum width = 20em, above = -2em of renderer] {Chromium Renderer};
\node (rendererend)[minimum width = 20em, below = 38em of renderertag]{};
\node (bt3begin)[minimum width = 8em, below left = 3em and -10em of renderertag]{worker thread};
\node (bt3end)[minimum width = 8em, below left = 38em and -10em of renderertag]{};
\node (bt4begin)[minimum width = 8em, below right = 3em and -10em of renderertag]{main thread};
\node (bt4end)[minimum width = 8em, below right = 38em and -10em of renderertag]{};
\draw [solid] (bt3begin) -- (bt3end);
\draw [solid] (bt4begin) -- (bt4end);
\node (3)[actionpoint, below of = bt3begin, node distance = 10em]{3};
\node (4)[actionpoint, below of = bt4begin, node distance = 11em]{4};
\node (7)[actionpoint, below of = 4, node distance = 11em, rotate=270] {sema};
\node (8)[actionpoint, below of = 3, node distance = 13em]{8};

\node [app, name = fontd, right of = renderer, node distance = 25em]{};
\node (fontdtag)[minimum width = 20em, above = -2em of fontd] {fontd};
\node (fontdend)[minimum width = 20em, below = 38em of fontdtag]{};
\node (bt5begin)[minimum width = 20em, below = 3em of fontdtag]{worker thread};
\node (bt5end)[minimum width = 20em, below = 38em of fontdtag]{};
\draw [solid] (bt5begin) -- (bt5end);
\node (5)[actionpoint, below of = bt5begin, node distance = 14em] {5};
\node (6)[actionpoint, below of = 5, node distance = 3em] {6};

\draw [->] (1) -- (2);
\draw [->] (2) -- (3);
\draw [->] (3) -- (4);
\draw [->] (4) -- (5);
\coordinate[name = notarrive7, point, right of = 7, node distance = 1em];
\draw [->] (6) -- (notarrive7);
\coordinate[name = before7, above of = 7, node distance = 1em]{};
\draw [->] (before7) -- (8);
\draw [->] (8) -- (9);
\coordinate[name = notarrive10, point, right of = 10, node distance = 1em]{};
\draw [->] (9) -- (notarrive10);

\node (findfirstrect)[minimum width = 40em, minimum height = 4em, right of = 1, node distance = 13em, rotate = 355] {TextInputClientMsg\_FirstRectForCharacterRange};
\node (javascript)[minimum width = 4em, minimum height = 4em, above left = 0em and 2em of 7, rotate = 15] {Run\_Javascript};
\end{tikzpicture}
}
\end{center}
%%\end{document}

	\parbox{\columnwidth}{\caption{Chromium Example}
		\textit{\vv{Path 1} shows a message flow from chromium to renderer process, to get the
			rect element position with message ID \vv{FirstRectForCharacterRange}. \vv{Path 2}
			shows a JavaScript processing event, dequeued from the renderer's main event
		queue. It blocks on the semaphore from browser for the rendering.}
    \label{fig:chromium-case-study}
	}
\end{figure}
%   resolve symbol, save to log
%     search for set\_spinning
%     if found,
%       find main thread node at the time of set\_spinning
%     find fontd
%     manually check nodes in each thread immediately after the nodes in the slice (normal abnormal boundary)
%  output is a node, and a HTML dump of node and immediate predecessors and successors
% systems preference          
%  spinning node in UI thread is not waiting
%  so we look for messages, and find the diff

One of the authors experienced first-hand the aforementioned performance issue
in Chromium, an open-source browser engine that powers Google Chrome and,
starting recently, Microsoft Edge~\cite{chromiumurl}. She tried to type in the
Chromium search box a Chinese word using SCIM, the default Chinese Input Method
Editor that ships with macOS. The browser appeared frozen and the spinning
cursor occurs for a few seconds. Afterwards everything went back to normal.
This issue is reproducible and always ruins her experience, but is quite
challenging to diagnose because two applications Chromium and SCIM and many
daemons ran and exchanged messages. It was reported by other users for
other non-English input methods, too.

% issue, the user followed Figure 1.Figure~\ref{fig:argus-overview}

To diagnose this issue with \xxx, the author follows the steps in
Figure~\ref{fig:argus-overview}. She started system-wide tracing, and then
reproduced the spinning cursor with a Chinese search string while the page was
loading. After the very first few characters as normal cases, the browser got
blocked with the rest input as spinning cases. The entire session took roughly
five minutes.
%  spinning node
She then ran \xxx to construct the event graph. The graph was highly complex,
with 2,749,628 vertexes and 3,606,657 edges, almost fully connected. It spanned
across 17 applications; 109 daemons including \vv{fontd}, \vv{mdworker},
\vv{nsurlsessiond} and helper tools by applications; 126 processes; 679
threads, and 829,287 IPC messages. Given the scale of the graph and the diverse
communication patterns, it would be extremely challenging for prior automated
causal tracing tools~\cite{aguilera2003performance, zhang2013panappticon,
attariyan2012x, cohen2004correlating} because they handle a fairly limited
set of patterns. Tools that require manual schema~\cite{barham2004using,
reynolds2006pip}, would be prohibitive because developers would have to provide
schema for all involved applications and daemons.
% Perfect case for \xxx

Next she ran interactive debugger in \xxx to find the \spinningnode in the main
thread of the browser process. \xxx returned a \vv{wait} event on condition
variable with timeout that blocked the main thread for a few seconds. Thus \xxx
compares the \spinningnode to a similar one in normal case where the \vv{wait}
was signaled quickly. \xxx reported three, and confirmed with the user which one
she wanted.

% Figure 2....

\xxx then found the normal-case wake-up path which connects five threads.
The browser main thread was signaled by a browser worker thread, which
received IPC from a worker thread of \vv{renderer} where the rendering view
and WebKit code run. The worker thread is woken up by the \vv{renderer} main
thread, which in turn woken by fontd, the font service daemon. \xxx further
compared the wake-up path with the spinning case, as shown in \vv{path 1}
in Figure~\ref{fig:chromium-case-study}, and returned the \vv{wait} event
on semaphore in the \vv{renderer} main thread, the culprit that delayed
waking up the browser main thread over 4 seconds. What caused the wait in the
\vv{renderer} main thread though? She thus continued diagnosis and recursively
applied \xxx to the wait in \vv{renderer}, and got the wake-up path. The culprit
that delayed the semaphore was the timeouts in the browser's main thread, as the
blocking edge shown in \vv{path 2} in Figure ~\ref{fig:chromium-case-study}. At
this point, a circular wait formed.

To understand what exactly happens in the situation, she inspected the full
call stacks by \xxx scripts, taking the reported vertices from the renderer and
the browser as input. Inspection reveals that the \vv{renderer} requested the
browser's help to render JavaScript and waited for reply with semaphore. The
browser was waiting for the \vv{renderer} to return the string bounding box and
the \vv{renderer} was waiting for the browser to help render JavaScript. This
circular wait was broken by a timeout in the browser main thread (the \vv{wait}
on \vv{Cf} timeout was 1,500 ms). While the system was able to make progress, the
next key press caused the spinning cursor to display for another 1,500 ms. The
timeout essentially converted a deadlock into a livelock.

Finally, we verified our finding within chromium source code. Shortening the
timeout interval in the main browser thread proportionally shortens the waiting
of the main render thread on processing the JavaScript. Skipping certain
JavaScript's processing in the renderer thread cuts down the success rate of
spinning case reproducing. \xxx reports enough information for developers to fix
issue once and for all.
% enough info for developer to fix issue once and for all

\subsection{Limitations}

\xxx is designed to support interactive debugging of performance issues. It
sometimes requires the user to reproduce a performance issue so \xxx can capture
more fine-grained event traces such as accesses to data flags. Fortunately, a
performance issue that almost never reproduces is probably not as annoying as
one that occurs frequently.

We implemented \xxx in the closed-source macOS which presents a harsh test
for \xxx, but we have not ported \xxx to other operating systems yet. It is
possible that the ideas and techniques do not generalize to other operating
systems. However, modern operating systems share many similarities, and inspire
each others' designs, so we are hopeful that the ideas in \xxx are generally
applicable. Similarly, the applications and performance issues used in our
evaluation may be non-representative, though we strive to cover a diverse set of
common applications ranging from browsers to text editors.
