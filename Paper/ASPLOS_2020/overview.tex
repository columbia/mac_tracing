\section{Overview}\label{sec:overview}

\subsection{Argus Work Flow}
\begin{figure}[tb]
    \centering
	%\documentclass{article}
%\usepackage{tikzpeople}
%\usepackage{tikz}
%\usepackage{graphicx}
%\usetikzlibrary{shapes, shapes.misc}
%\usetikzlibrary{arrows, arrows.meta, decorations.markings}
%\usetikzlibrary{patterns}
%\usetikzlibrary{positioning}


\makeatletter
% copied from the manual
\pgfdeclareshape{document}{%
    \inheritsavedanchors[from=rectangle] % this is nearly a rectangle
    \inheritanchorborder[from=rectangle]
    \inheritanchor[from=rectangle]{center}
    \inheritanchor[from=rectangle]{north}
    \inheritanchor[from=rectangle]{south}
    \inheritanchor[from=rectangle]{west}
    \inheritanchor[from=rectangle]{east}
	\inheritanchor[from=rectangle]{south west}
	\inheritanchor[from=rectangle]{north west}
	\inheritanchor[from=rectangle]{south east}
	\inheritanchor[from=rectangle]{north east}
    % ... and possibly more
    \backgroundpath{% this is new
        % store lower right in xa/ya and upper right in xb/yb
        \southwest \pgf@xa=\pgf@x \pgf@ya=\pgf@y
        \northeast \pgf@xb=\pgf@x \pgf@yb=\pgf@y
        \def\hangout{5pt}
        % compute corner of "flipped page"
        \pgf@xc=\pgf@xb \advance\pgf@xc by-\hangout % this should be a parameter
        \pgf@yc=\pgf@yb \advance\pgf@yc by-\hangout
        % construct main path
        \pgfpathmoveto{\pgfpoint{\pgf@xa}{\pgf@ya}}
        \pgfpathlineto{\pgfpoint{\pgf@xa}{\pgf@yb}}
        \pgfpathlineto{\pgfpoint{\pgf@xc}{\pgf@yb}}
        \pgfpathlineto{\pgfpoint{\pgf@xb}{\pgf@yc}}
        \pgfpathlineto{\pgfpoint{\pgf@xb}{\pgf@ya}}
        \pgfpathclose
        % add little corner
        \pgfpathmoveto{\pgfpoint{\pgf@xc}{\pgf@yb}}
        \pgfpathlineto{\pgfpoint{\pgf@xc}{\pgf@yc}}
        \pgfpathlineto{\pgfpoint{\pgf@xb}{\pgf@yc}}
        \pgfpathlineto{\pgfpoint{\pgf@xc}{\pgf@yc}}
        \pgfpathclose
        % add lines
        %\pgfmathsetmacro{\step}{round(abs(\pgf@yb-\pgf@ya)/6)}
        %\pgfmathsetmacro{\lines}{round(abs(\pgf@yb-\pgf@ya)/\step)-1}
        %\foreach \y in {1,...,\lines}{%
        %    \pgfpathmoveto{\pgfpoint{\pgf@xa+\step*1pt+\hangout}{\pgf@yb-\y*\step*1pt}}
        %    \pgfpathlineto{\pgfpoint{\pgf@xc-\step*1pt}{\pgf@yb-\y*\step*1pt}}
        %}%  
    }%
}%
\makeatother

%\begin{document}
\begin{center}
\resizebox{\columnwidth}{!}{%
\begin{tikzpicture}[>=latex]
\tikzstyle{every node}=[font=\Large]
\tikzstyle{apps} = [draw, very thick, minimum height=30pt, minimum width=50pt, fill=white, rectangle, font={\sffamily\bfseries}]
\tikzstyle{systemComp} = [draw, very thick, minimum height=30pt, minimum width=150pt, pattern= north west lines, pattern color = black!20!, rectangle, font={\sffamily\bfseries}]
\tikzstyle{doc} = [draw, thick, shape=document, align=left, color=black, minimum width=150pt, minimum height=40pt, inner sep=2ex]
\tikzstyle{vecarrow} = [->, thick, line width=5pt]
\tikzstyle{gthread} = [label={[font=\small, red]below:#1}]
\tikzstyle{gnode} = [draw, thick, font=\small, align =left, minimum width=18pt, minimum height=36pt]

\tikzset{ext/.pic={
\path [fill=white] (-0.2,0)to[bend left](0,0.1)to[bend right](0.2,0.2)to(0.2,0)to[bend left](0,-0.1)to[bend right](-0.2,-0.2)--cycle;
\draw (-0.2,0)to[bend left](0,0.1)to[bend right](0.2,0.2) (0.2,0)to[bend left](0,-0.1)to[bend right](-0.2,-0.2);
}}

%%draw MacOS components
\node [apps, name = software1] at (0, 0) {\includegraphics[width=20pt]{./chromium_logo.png}};
\node [apps, name = software2, right of=software1, node distance=50pt] {daemon1};
\node [apps, name = software3, right of=software2, node distance=50pt] {daemon2};
\node [systemComp, name = libs, below of = software2, node distance =30pt] {Argus Instrumented Libraries};
\node [systemComp, name = os, below of = libs, node distance =30pt] {Argus Instrumented OS};

%% draw Tracing Event log
\node (TracingEventLog) [minimum height=30pt, minimum width=200pt, right of=software3, node distance=220pt]{};
\node (log) [doc, below of=TracingEventLog, node distance=30pt] {\#timestamp, event\_type, attr1, attr2...\\30.4, sendMessge, tid1, tid2 ..\\31.7, wakeUp, tid0, tid2...\\33.2, wakeUp, tid2, tid3...\\34.2, sendMessage, tid3, tid2...\\34.9, wakeUp, tid2, tid0...\\ ... ... };
\draw [vecarrow] (libs.east)+(0.5, 0) -- (log.west);

%draw Interactive Debugger 
\node (InteractiveDebugger) [systemComp, rounded corners=20pt, minimum height=280pt, minimum width=300pt, below left = 20pt and -160pt of log] {};
\node (debugger)[minimum width=220pt, above = -30pt of InteractiveDebugger]{Interactive Debugger};
\node (graph) [minimum height = 220pt, minimum width = 220pt, below = 10pt of debugger] {};
\node (shape) [doc, minimum height = 220pt, minimum width = 220pt, below = 10pt of debugger]{};
%\node [minimum width = 100pt, above = -20pt of graph, label=right:{graph}]{};
\node(t0s)[gthread = browser, minimum width = 2pt, below left = -200pt and -50pt of graph, font=\small]{tid0};
\node(t0e)[gthread = browser, minimum width = 2pt, below left = -20pt and -44pt of graph]{};
\node(t1s)[gthread = kern\_task, minimum width = 2pt, below left = -200pt and -85pt of graph, font=\small]{tid1};
\node(t1e)[gthread = kern\_task, minimum width = 2pt, below left = -20pt and -79pt of graph]{};
\node(t2s)[gthread = renderer, minimum width = 2pt, below left = -200pt and -120pt of graph, font=\small]{tid2};
\node(t2e)[gthread = renderer, minimum width = 2pt, below left = -20pt and -114pt of graph]{};
\node(t3s)[gthread = renderer, minimum width = 2pt, below left = -200pt and -155pt of graph, font=\small]{tid3};
\node(t3e)[gthread = renderer, minimum width = 2pt, below left = -20pt and -149pt of graph]{};
\node(t4s)[gthread = fontd, minimum width = 2pt, below left = -200pt and -185pt of graph, font=\small]{tid4};
\node(t4e)[gthread = fontd, minimum width = 2pt, below left = -20pt and -179pt of graph]{};

\node(brw0) [gnode, fill = blue!20, below = 20pt of t0s]{msg\\msg};
\node(brw1) [gnode, below = 15pt of brw0]{write\\msg};
\node(brw2) [gnode, below = 20pt of brw1]{read\\write};
\node(k0) [gnode, fill = red!20, below = 25pt of t1s] {msg};
\node(k1) [gnode, below = 50pt of k0] {msg\\wait};
\node(rdw0) [gnode, fill = yellow!20, below = 30pt of t2s]{msg\\read\\msg};
\node(rdw1) [gnode, below = 30pt of rdw0]{msg\\msg};
\node(rdm0) [gnode, below = 40pt of t3s]{read\\msg};
\node(rdm1) [gnode, minimum height = 20pt, below = 20pt of rdm0]{write\\msg};
\node(rdm2) [gnode, thick, fill = blue!20, minimum height = 10pt, below = 18pt of rdm1, font=\small]{msg\\sema\_wait};
\node(fontd0) [gnode, minimum height=20pt, below = 50pt of t4s]{msg\\msg};

\draw[-, solid](t0s.south) -- (brw0);
\draw[-, dotted](brw0) -- (brw1);
\draw[-, dotted](brw1) -- (brw2);
%\draw[-, solid](brw2) --(t0e.south);
\draw[-, solid](t1s.south) -- (k0);
\draw[-, dotted](k0) -- (k1);
\draw[-, solid](k1) -- (t1e.south);
\draw[-, solid](t2s.south) -- (rdw0);
\draw[-, dotted](rdw0) -- (rdw1);
\draw[-, solid](rdw1) -- (t2e.south);
\draw[-, solid](t3s.south) -- (rdm0);
\draw[-, dotted](rdm0) -- (rdm1);
\draw[-, solid](rdm1) --pic{ext}(rdm2);
\draw[-, solid](t4s) -- (fontd0);
\draw[-, dotted](rdm2) -- (t3e.south);
\draw[-, solid](fontd0) --(t4e.south);

\draw[->] (brw0.east) -- node[font=\small, minimum width=5pt, pos=0.1, below]{e0} (rdw0.west);
\draw[->] (k0.east) -- node[font=\small, minimum width=5pt, above]{e1} (rdw0.west);
\draw[->] (rdw0.east) -- (rdm0.west);
\draw[->] (rdm0.east) -- (fontd0.west);
\draw[->] (fontd0.west) -- (rdm1.east);
\draw[->] (rdm1.west) -- (rdw1.east);
\draw[->] (rdw1.west) -- (brw2.east);


\draw [vecarrow] (log.south) -- (InteractiveDebugger);

\node (legend)[minimum height =180pt, minimum width = 80pt, right of = InteractiveDebugger, node distance= 200pt]{};
\node (1)[below left = -20pt and -20pt of legend] {};
\node (2)[right=8pt of 1]{};
\draw [-, solid] (1.east) -- (2);
\node [right= 10pt of 1, font=\small]{thread\_running};

\node (3)[below left = -30pt and -20pt of legend] {};
\node (4)[right=8pt of 3]{};
\draw [-, dotted] (3.east) -- (4);
\node [right =10pt of 3, font=\small]{thread\_wait};

\node (5)[below left = -40pt and -20pt of legend] {};
\node (6)[right =8pt of 5]{};
\draw [->] (5.east) -- (6);
\node [right =10pt of 5, font=\small]{edge};

\node [draw, solid, minimum width = 10pt, minimum height = 3pt, below left = -50pt and -30pt of legend, label={[font=\small]right:node}]{};

%draw interactive Debugging part
\umlactor[left = 100pt of InteractiveDebugger, scale=2]{user};
\node (user)[left = 100pt of InteractiveDebugger]{};
%\node (query1)[above left = -80pt and -10pt of InteractiveDebugger]{};
%\node (query2)[above left = -100pt and -10pt of InteractiveDebugger]{};
\node (query3)[below left = -100pt and -10pt of InteractiveDebugger]{};
\node (query4)[below left = -80pt and -10pt of InteractiveDebugger]{};
%\draw [->, thick, out=160, in=60] (query1.north west) to node{which node?} (user.north);
%\draw [->, thick, out=350, in=230] (user.north) to node{node 5} (query2.north west);
\draw [->, thick, out=150, in=30] (query3.north west) to node {\includegraphics[width=60pt]{./graph_snippet.png}} (user.south);
\draw [->, thick, out=310, in=230] (user.south) to node[above]{e0} (query4.north);

\node (result) [draw, dotted, minimum height = 100pt, minimum width = 400pt,pattern= north west lines, pattern color = black!30, below = 30pt of InteractiveDebugger]{};
\node (blockrenderer) [draw, solid, minimum height = 10pt, fill=blue!20, below left = -40pt and -200pt of result, label=below:render thread blocks on semaphore]{msg sema\_wait};
\node (blockbrowser) [draw, solid, minimum height = 10pt, below left = -80pt and -200pt of result, label=below:browser thread blocks on condition variable]{msg cv\_wait};

%\node (text1)[below = 50pt of = blockrenderer] {renderer thread \\blocks on semaphore};
%\node (text1)[below = 50pt of = blockbrowser] {browser thread \\blocks on conditional variable};

\draw [vecarrow] (InteractiveDebugger.south) -- (result);
\end{tikzpicture}
}
\end{center}
%\end{document}

    %\includegraphics[width=\columnwidth]{figures/overview_tikz.png}
    \caption{\xxx Work Flow}
    \label{fig:argus-overview}
\end{figure}

In this section, we describe the steps a user takes to investigate a performance
anomaly with \xxx. Figure~\ref{fig:argus-overview} shows \xxx's work flow with
an example of a user investigating a performance problem in Chromium. The system
wide tracing tool, which collects data from \xxx instrumented library and
kernel, generates logs. They are transformed into an graph in \xxx's
graph construction component. The generated graph is used by the interactive
debugger for causal path slicing and diagnosis. \xxx supports interactive
search, by providng information to the user and asking for decision, in case of
multiple predecessors in a vertex. As shown in Figure~\ref{fig:argus-overview},
the debugger asks user to choose one edge in a subgraph. After this step,
\xxx performs diagnosis algorithm and reports the root cause vertices. In the
example, the root cause is two vertices which form a circular wait acrossing
multiple threads.

%%debugging base of the interactive debugger.
%- sys-wide tracing => log
%- event graph
%- interacjjkjjktive search "find root cause"
%- root cause is two nodes, across multiple threads


%
%In this section, we describe the steps a user takes to investigate a performance
%anomaly with \xxx. Figure~\ref{fig:argus-overview} shows \xxx's work flow, which
%consists of two phases. A user runs command ``\vv{\xxx start}'' to enter the
%system-wide tracing phase, within which \xxx logs events as listed in Table
%~\ref{table:event_types} (\S\ref{subsec:eventgraph}). Whenever a user detects
%a performance issue such as a spinning cursor, she runs ``\vv{\xxx debug}'' to
%enter the diagnosis phase.
%
%%In the cases we consider, the flow of information across threads and processes
%%is essential to discovering the system state that leads to a performance bug.
%%\xxx recovers UI actions from logged data rather than being told the actions
%%that a user performs, because not all of them may be relevant to the true bug.

Central to our system is our \emph{event graph}, a generalized control-flow
graph which includes inter-thread and inter-process dependencies. Diagnosis
and inferences are performed within this graph, in a semi-automated fashion:
\xxx performs searches to trace logical events as they
flow through the system, and it judicially queries the user for guidance.
% The event graph is described in further detail in \S\ref{subsec:eventgraph}.
Next, we describe how \xxx assists the user to diagnose a performance issue.

%fwd ref section on chromium, but first, how search event graph

\subsection{Diagnosis with Graph}\label{subsec:debug}

\begin{algorithm}
\caption{Diagnosis algorithm.}
\label{alg:alg-diagnosis}
\begin{algorithmic}[1]
\Require{g - EventGraph; \algspinningnode - the vertex in the UI thread when the spinning cursor occurs}
\Ensure{\algrootcausenodes-collecting root casuse vertices for user inspect}
\Statex
\Function{Diagnose}{g, \algspinningnode} \label{alg:diagnose-line}%\funclabel{alg:diagnose} \label{alg:diagnose-line}
  \Switch {\algspinningnode.block\_type}
    \Case {LongRunning} \label{a1:longrunning_begin}
		\State {slice $\gets$ InteractiveSlice(\algspinningnode)}
		\State\Return {vertex contains UI event} \label{a1:longrunning_end}
	\EndCase
	\Case {RepeatedYield} \label{a1:repeatedyield_begin}
		\If {DataFlagEvent $\notin$ $\{$event types in \algspinningnode$\}$}
			\State {Require users to annotate data flag}
			\State {abort()}
		\EndIf
		\LineComment {Fall through}\label{a1:repeatedyield_end}
	\EndCase
	\Case {LongWait} \label{a1:longwait_begin}
		\State {\algsimilarnode $\gets$ vertex has similar event sequence to \algspinningnode}
		%\State {baseline\_vertex $\gets$ ask user to pick from \algsimilarnodes} 
		\State {baseline\_path $\gets$ InteractiveSlice(\algsimilarnode)} 	
		\For {each t in $\{$threads in baseline\_path$\}$}
			\State {$vertex_t$ $\gets$ vertex in t before \algspinningnode gets spinning}
			\If {$vertex_t$ $\in$ $\{$LongRunning, RepeatedYield, LongWait$\}$}
				%\State{is\_related\_to\_bug $\gets$ ask user if $node_t$ related to bug}
				%\If {is\_related\_to\_bug $\not=$ True}
				%	\State{Continue}
				%\EndIf
				\State {\algrootcausenodes.append($vertex_t$)}
				\State {\algrootcausenodes.append(Diagnose(g, $vertex_t$)}
			\EndIf
			\LineComment{if t is normal running, disgnose the next thread}
		\EndFor \label{a1:longwait_end}
	\EndCase
  \EndSwitch
  \State\Return{\algrootcausenodes}
\EndFunction
%%\end{algorithmic}
%%\end{algorithm}

%\begin{algorithm}
%\caption{InteractiveSlicing algorithm.}
%\label{alg:alg-interactiveslicing}
%\begin{algorithmic}[1]
%\Require{g - EventGraph; node - the node user wants to slice from}
%\Ensure{slice - causual path for node}
\Statex
\Function{InteractiveSlicing}{g, vertex}\label{alg:interslice-line}
%\funclabel{alg:interactiveslicing}
%\label{alg:interactiveslicingline}
\Loop
	\State{path\_slice.append(vertex)}
	\If {vertex has 1 incoming edge}
		\State{vertex $\gets$ predecessor vertex}
	\ElsIf {vertex has multiple incoming edges}
		\State{vertex $\gets$ ask user to pick from predecessors}
	\ElsIf {vertex had weak edges}
		\State{vertex $\gets$ ask user to pick from predecessors}
	\Else \LineComment{The first vertex of current thread}
		\State\Return {path\_slice}
	\EndIf
	\If {vertex is invalid}
	\LineComment{user chooses to stop traversal with invalid input}
		\State\Return {path\_slice}
	\EndIf
\EndLoop
\EndFunction
\end{algorithmic}
\end{algorithm}


Consider a common performance bug on macOS, the \emph{spinning cursor},
which indicates the current application's main thread has not processed
any UI events for over two seconds.  To initialize debugging a spinning
cursor, \xxx first constructs an event graph from the system-wide event
log recorded.  It then queries the event graph to find the ongoing event
in the application's main thread concurrent to the display of the spinning
cursor.  Given the event graph and the \spinningnode, \xxx runs
Algorithm~\ref{alg:alg-diagnosis} to interactively pinpoint the root
cause.

Specifically, upon examining what the main thread is actually doing, there
are three potential cases.

\begin{itemize}

  \item \textbf{LongRunning} (lines~XX-XX). The main thread is busy
    performing lengthy CPU operations.  This case is the simplest, and
    \xxx traverses the event graph backwards to find a slice originating
    from the offending UI event to the long running CPU operations.  This
    slice is particularly useful for further diagnosing the bug.  As shown
    in FunctionXXX, \xxx may encounter vertices with multiple incoming edges
    or weak edges that may not reflect causality when traversing the
    graph.  It queries the user to resolve them.

  \item \textbf{RepeatedYield} (lines~XX-XX). The main thread is in a
    yield loop, which is highly indicative it is waiting on a data flag
    (\eg, ``while(!done) thread\_switch();'').  If \xxx cannot find any
    record of data flags in the \spinningnode, it terminates debugging by
    prompting the user to identify data flags and re-trace the
    application.  Here we assume that the performance issue reproduces
    with a reasonable probability because, fortunately, a one-off issue
    that never reproduces is not as annoying as one that occurs
    frequently.  If \xxx finds the data flag the \spinningnode is waiting
    for, it falls through to the next case.

  \item \textbf{LongWait} (lines~XX-XX). The main thread is in a lengthy
    blocking wait and the wake-up has been missing.  \xxx handles this
    case by finding a baseline scenario where the wake-up indeed arrives,
    and then figures out which wake-up edge is missing in the spinning
    scenario along the expected wake-up path.  Specifically, \xxx first
    finds a \similarnode to the spinning one based solely on the
    semantical events such as system calls in each vertex.  It then
    traverses backwards from the \similarnode to find the baseline wake-up
    path.  For each thread in the wake-up path, it examines the vertex in
    the thread right before the \spinningnode waits.  If this vertex is also
    abnormal, \xxx appends it to the path of \rootcausenodes, and applies
    Function DiagnoseXX recursively diagnose ``the culprit of the
    culprit.''  For each such vertex, it queries the user to determine
    whether to proceed or stop because based on our experience the user
    needs to inspect only a few vertices to find the root cause.

\end{itemize}
\vskip 0.1cm
Based on our results and experience, the first case is the most common,
but the second and third represent more severe bugs. Long-running CPU
operations tend to be more straightforward to diagnose with existing tools
such as \spindump except they do not connect the CPU operations back
to UI events.  Repeated yielding or long waiting cases involve multiple
threads and processes, and are extremely hard to understand and fix even
for the application's original developers.  Therefore, issues remain
unaddressed for years and significantly impact the user
experience. Algorithm~\ref{alg:alg-main} is semi-automated but can
integrate user input at each stage to leverage hypotheses or expert
knowledge as to why a hang may occur.  Our results show that user inputs,
albeit few, are crucial in this process (\S\ref{sec:eval}).

%% Algorithm~\ref{alg:alg-diagnosis} shows how \xxx investigates a performance
%% anomaly for users, given a buggy test case and (potentially) a baseline
%% test case for comparison purposes. We assume the user has a program to
%% execute and a buggy test case that can be probabilistically reproduced;
%% the tracing log collected by \xxx would contains both a buggy case and a
%% baseline case at the best, or the user can provide a normally executed test
%% case for comparison purpose, listed in line~\ref{algmain:collectdata}.
%% The tracing tool can run all the time in the background to capture the
%% probabalistic bug. \xxx initializes this debugging phase by constructing
%% an event graph from all logged events up to user-specified duration in
%% line~\ref{algmain:recompute}. The detail of algorithm ComputeGraph is presented
%% in Algorithm ~\ref{alg:graphcomputing}(\S\ref{subsec:graphcomputing}), described
%% later.

%% The exact searches and queries performed on the graph depend on the bug under
%% investigation. Consider a common performance bug on macOS, the \emph{spinning
%% cursor}, which indicates the current application's main thread has not processed
%% any UI events for over two seconds. \xxx queries the event graph to find the
%% ongoing event in the application's main thread concurrent to the display of
%% the spinning cursor shown in line~\ref{algmain:findspinningnode}. Upon examining
%% what the main thread is actually doing, the user may encounter three potential
%% cases. First, the thread may be busy performing lengthy CPU operations (which
%% take longer than two seconds). Second, the thread may be in a blocking wait.
%% Third, the main thread may be in a yield loop, which is highly indicative
%% it is waiting on a data flag (\eg, ``while(!done) thread\_switch();'').
%% Line ~\ref{algmain:1st} to ~\ref{algmain:2and3end} is to track events and
%% synchronization primitives throughout the system to figure out the root cause,
%% in terms of user input events and bug in code, for the three cases.

%% \paragraph{B1}

%% For the first case, shown in line~\ref{algmain:1st}, \xxx will examine
%% the stack trace and find a sequence of events that led to the current CPU
%% processing. If more specific tracing is required, the user can rerun the
%% program with more heavyweight instrumentation enabled for any portion of the
%% code's execution, gathering a precise sequence of calls or even instructions
%% executed.

%% In other cases, \xxx locates the node $N_1$ for the rest
%% cases, either a blocking wait or a yield loop, in our graph
%% when the spinning beachball is triggered, as is shown between
%% line~\ref{algmain:2and3begin} and line~\ref{algmain:2and3end}. Then \xxx applies
%% Algorithm~\ref{alg:alg-findsimilarnode} to locate a node $N_0$ in our graph that
%% corresponds to the baseline version of the hanging node $N_1$.

%% \paragraph{B2}

%% For the second case of block waiting, \xxx takes the assumption there usually
%% exits a chain of blocking thread and the initial blocking thread is the most
%% valuable for debugging. Consequently, \xxx diffs the subgraphs between the
%% blocking version and the baseline version to determine their difference with
%% Algorithm ~\ref{alg:alg-graphdiff}, described later.

%% \paragraph{B3}

%% In the third case, the main thread is in a yield loop, which is
%% highly indicative it is waiting on a data flag (\eg, ``while(!done)
%% thread\_switch();''). To discover a data flag, the user re-runs the application
%% with \xxx to collect instruction traces of the concurrent event in both the
%% normal and spinning cases and detects where the control flow diverges. She
%% then reruns the application with \xxx to collect register values for the basic
%% blocks before the divergence and uncovers the address of the data flag. She
%% then configures \xxx to log accesses to the flag during system-wide tracing.
%% Finally, she can recursively apply \xxx to further diagnose ``the culprit of the
%% culprit''.

%% Based on our results and experience, the first case is the most common, but
%% the second and third represent more severe bugs. Long-running CPU operations
%% tend to be more straightforward to diagnose with existing tools. Blocking
%% or yielding cases involve multiple threads and processes, and are extremely
%% hard to understand and fix even for the application's original developers.
%% Therefore, issues remain unaddressed for years and significantly impact the user
%% experience. Algorithm~\ref{alg:alg-main} is semi-automated but can integrate
%% user input at each stage to leverage hypotheses or expert knowledge as to why a
%% hang may occur.



%% \paragraph{Find Similar Node Algorithm}

%% FindSimilarNode in Algorithm~\ref{alg:alg-findsimilarnode} is meant to find
%% nodes which have the same high level semantics but different execution
%% results compared to the spinning node. For example, in both cases a thread
%% waits on a lock, but one returns successfully and the other not. \xxx
%% defines a subset of events that preserve semantics, which consists of
%% System\_call, Back\_trace, NSApp\_event, Mach\_message, Wait, Dispatch\_enqueue,
%% Dispatch\_invoke, Runloop\_submit, Runloop\_invoke, Share\_flag\_read and
%% Share\_flag\_write. The comparison of events depends on event type and its
%% category in Table~\ref{table:event_types}, described in \S~\ref{subsec:eventgraph}.
%% For connection events, \xxx compares their peers. For semantics Events, \xxx
%% compares the content of the event attributes, for example, the system call
%% number. The wait results of wait event usually differentiate the normal
%% execution and buggy case. \xxx differentiate the cases with wait results by
%% default. Depending on the report detail of the spinning node, user can change
%% the comparing metrics or request \xxx to compare proceeding nodes to narrow down
%% the result set reported.

%% \paragraph{Graph Diff Algorithm}

%% The Algorithm~\ref{alg:alg-graphdiff} is a single step for the semi-automatic
%% debugging phase. It is design for figure out the root cause for
%% \textbf{B2}(\S\ref{subsec:debug}). \xxx assumes that the blocking in the main
%% thread is usually caused by the unresponsiveness of another thread. In the
%% algorithm, with the input of the blocking node in main thread, \xxx carries
%% out a backward path slicing from the node from the baseline case in the graph.
%% The path consists of nodes from multiple threads. \xxx traverses every thread
%% and confines the search within the time interval from the baseline node in the
%% thread and the timestamp when the blocking node appears in main thread. \xxx
%% expects to find a blocking call in some threads and reports such cases. At the
%% same time \xxx also collects the UIEvents into return set.

%% \begin{algorithm}[ht!]
%%     \caption{\xxx Find similar node algorithm.}
%%     \label{alg:alg-findsimilarnode}
%% \begin{algorithmic}[1]
%% %%  Output: Similar node set
%% \Require{EventTypesSet + SpinningNode + Graph}
%% \Ensure{SimilarNodeSet}
%% \Statex
%% \Function{FindSimilarNode}{}
%% \State{$thread$ $\gets$ thread of $Node_{blocking}$}
%% \For {$Node_i$ in $thread$ other than $Node_{blocking}$}
%% %%	\State{$iter_i$ $\gets$ iterator of first event in $Node_i$}
%% %%	\State{$iter_b$ $\gets$ iterator of first event in $Node_{blocking}$}
%% %%	\While {$iter_b$ $\ne$ end \And $iter_i$ $\ne$ end}
%% %%		\While {$iter_i$ $\ne$ end \And $Event$($iter_i$) $\notin$ $EventTypesSet$}
%% %%			\State{$iter_i$++}
%% %%		\EndWhile
%% %%		\While {$iter_b$ $\ne$ end \And $Event$($iter_b$) $\notin$ $EventTypesSet$}
%% %%			\State{$iter_b$++}
%% %%		\EndWhile
%% 	\For{$iter_i$ $\gets$ NextEventInEventTypeSet($Node_i$)
%% 		\And $iter_b$ $\gets$ NextEventInEventTypeSet($Node_{blocking}$)}
%% 		%%\If {$iter_b$ $\neq$ end \And $iter_i$ $\neq$ end}
%% 		\If{CompareEvent($Event$($iter_b$), $Event$($iter_i$)) $==$ False}
%% 			%%\If {$if_eq$ $==$ False}
%% 			\State{Break}
%% 		\EndIf
%% 		%%\EndIf
%% 	\EndFor
%% 	\If {$iter_b$ $==$ end \And $iter_i$ $==$ end}
%% 		\If {wresult of $Node_i$ $\neq$ wresult of $Node_{blocking}$
%% 		\Or timespan of $Node_{blocking}$ - timespan of $Node_i$ $\geq$ threshhold}
%% 			\State{Put $Node_i$ into $SimilarNodeSet$}
%% 		\EndIf
%% 	\EndIf
%% \EndFor
%% \EndFunction
%% \end{algorithmic}
%% \end{algorithm}

%% \begin{algorithm}[ht!]
%%     \caption{\xxx Assisted graph diff algorithm.}
%%     \label{alg:alg-graphdiff}
%% \begin{algorithmic}[1]
%% \Require{SpinningNode + BaselineNode + Graph}
%% \Ensure{Possible root cause of thread blocking}
%% \Statex
%% \Function{AssistedGraphDiff}{}
%% \State{$ResultSet$ $\gets$ $\{\}$}
%% \State{$SlicingPath$ $\gets$ BackwardSlicing on $BaseLineNode$}
%% \State{$T_{spinning}$ $\gets$ timestamp of spinning beachball for $SpinningNode$}
%% \For{$thread$ $\gets$ thread of $Node_i$ in $SlicingPath$}
%% 	\State{$T_{normal}$ $\gets$ timestamp of last event in $Node_i$}
%% 	\State{$Node_{blocking}$ $\gets$ $thread$ blocks during ($T_{normal}$, $T_{spinning}$)}
%% 	\If {$Nodeblocking$  $\neq$ $NULL$}
%% 		\State{put $Node_{blocking}$ into $ResultSet$}
%% 	\EndIf
%% 	\If {$UI Events$ in Main UI thread}
%% 		\State{put $UI Event$ into $ResultSet$}
%% 	\EndIf
%% \EndFor
%% \State{Return $ResultSet$}
%% \EndFunction
%% \end{algorithmic}
%% \end{algorithm}

% \subsection{\xxx Assisting Algorithm}

% We mentioned previously in \S\ref{subsec:debug} that \xxx explores the event
% graph semi-automatically to narrow in on relevant nodes in event graph for
% diagnosis. Algorithms that require no user intervention are described
% in this section.

%% \paragraph{Find Similar Node Algorithm}

%% FindSimilarNode in Algorithm~\ref{alg:alg-findsimilarnode} is meant to find
%% nodes which have the same high level semantics but different execution
%% results compared to the spinning node. For example, in both cases a thread
%% waits on a lock, but one returns successfully and the other not. \xxx
%% defines a subset of events that preserve semantics, which consists of
%% System\_call, Back\_trace, NSApp\_event, Mach\_message, Wait, Dispatch\_enqueue,
%% Dispatch\_invoke, Runloop\_submit, Runloop\_invoke, Share\_flag\_read and
%% Share\_flag\_write. The comparison of events depends on event type and its
%% category in Table~\ref{table:event_types}, described in \S~\ref{subsec:eventgraph}.
%% For connection events, \xxx compares their peers. For semantics Events, \xxx
%% compares the content of the event attributes, for example, the system call
%% number. The wait results of wait event usually differentiate the normal
%% execution and buggy case. \xxx differentiate the cases with wait results by
%% default. Depending on the report detail of the spinning node, user can change
%% the comparing metrics or request \xxx to compare proceeding nodes to narrow down
%% the result set reported.

%% \paragraph{Graph Diff Algorithm}

%% The Algorithm~\ref{alg:alg-graphdiff} is a single step for the semi-automatic
%% debugging phase. It is design for figure out the root cause for
%% \textbf{B2}(\S\ref{subsec:debug}). \xxx assumes that the blocking in the main
%% thread is usually caused by the unresponsiveness of another thread. In the
%% algorithm, with the input of the blocking node in main thread, \xxx carries
%% out a backward path slicing from the node from the baseline case in the graph.
%% The path consists of nodes from multiple threads. \xxx traverses every thread
%% and confines the search within the time interval from the baseline node in the
%% thread and the timestamp when the blocking node appears in main thread. \xxx
%% expects to find a blocking call in some threads and reports such cases. At the
%% same time \xxx also collects the UIEvents into return set.

\subsection{Chromium Spinning Cursor Example}

\begin{figure}[h!]
    \centering
	%%\documentclass{article}
%%\usepackage{tikzpeople}
%%\usepackage{tikz}
%%\usetikzlibrary{shapes, shapes.misc}
%%\usetikzlibrary{arrows, arrows.meta, decorations.markings}
%%\usetikzlibrary{patterns}

%%\begin{document}
\begin{center}
\resizebox{\columnwidth}{!}{%
\begin{tikzpicture}[>=latex]
\tikzstyle{every node}=[font=\Large]
\tikzstyle{app} = [draw, very thick, minimum height=40em, minimum width=20em, fill=white, rectangle, font={\sffamily\bfseries}]
\tikzstyle{actionpoint} = [minimum width = 1em, fill = white]
\tikzstyle{point} = [thick, draw=red, cross out, inner sep = 0pt, minimum width = 0.5em, minimum height =0.5em]
%\tikzset{
%%ext/.pic={
%%\path [fill=white] (-0.2,0)to[bend left](0,0.1)to[bend right](0.2,0.2)to(0.2,0)to[bend left](0,-0.1)to[bend right](-0.2,-0.2)--cycle;
%%\draw (-0.2,0)to[bend left](0,0.1)to[bend right](0.2,0.2) (0.2,0)to[bend left](0,-0.1)to[bend right](-0.2,-0.2);
%%},
%point/.style={
%    thick,
%    draw=red,
%    cross out,
%    inner sep=0pt,
%    minimum width=4pt,
%    minimum height=4pt,
%}}

%draw applications
\node [app, name = browser]{};
\node (browsertag)[minimum width = 20em, above = -2em of browser] {Chromium Browser};
\node (browserend)[minimum width = 20em, below = 38em of browsertag] {};
\node (bt1begin)[minimum width = 8em, below left = 3em and -10em of browsertag]{main thread};
\node (bt1end)[minimum width = 8em, below left = 38em and -10em of browsertag]{};
\node (bt2begin)[minimum width = 8em, below right = 3em and -10em of browsertag]{worker thread};
\node (bt2end)[minimum width = 8em, below right = 38em and -10em of browsertag]{};
\draw [solid] (bt1begin) -- (bt1end);
\draw [solid] (bt2begin) -- (bt2end);
\node (1)[actionpoint, below of = bt1begin, node distance = 8em]{1};
\node (2)[actionpoint, below of = bt2begin, node distance = 9em]{2};
\node (9)[actionpoint, below of = 2, node distance = 16em] {9};
\node (10)[actionpoint, below of = 1, minimum width = 6em, minimum height = 2em, node distance = 18em, rotate=270]{timed wait};
%\node (11)[actionpoint, below of = 10, node distance = 4em]{};

\node [app, name = renderer, right of = browser, node distance = 25em]{};
\node (renderertag)[minimum width = 20em, above = -2em of renderer] {Chromium Renderer};
\node (rendererend)[minimum width = 20em, below = 38em of renderertag]{};
\node (bt3begin)[minimum width = 8em, below left = 3em and -10em of renderertag]{worker thread};
\node (bt3end)[minimum width = 8em, below left = 38em and -10em of renderertag]{};
\node (bt4begin)[minimum width = 8em, below right = 3em and -10em of renderertag]{main thread};
\node (bt4end)[minimum width = 8em, below right = 38em and -10em of renderertag]{};
\draw [solid] (bt3begin) -- (bt3end);
\draw [solid] (bt4begin) -- (bt4end);
\node (3)[actionpoint, below of = bt3begin, node distance = 10em]{3};
\node (4)[actionpoint, below of = bt4begin, node distance = 11em]{4};
\node (7)[actionpoint, below of = 4, node distance = 11em, rotate=270] {sema};
\node (8)[actionpoint, below of = 3, node distance = 13em]{8};

\node [app, name = fontd, right of = renderer, node distance = 25em]{};
\node (fontdtag)[minimum width = 20em, above = -2em of fontd] {fontd};
\node (fontdend)[minimum width = 20em, below = 38em of fontdtag]{};
\node (bt5begin)[minimum width = 20em, below = 3em of fontdtag]{worker thread};
\node (bt5end)[minimum width = 20em, below = 38em of fontdtag]{};
\draw [solid] (bt5begin) -- (bt5end);
\node (5)[actionpoint, below of = bt5begin, node distance = 14em] {5};
\node (6)[actionpoint, below of = 5, node distance = 3em] {6};

\draw [->] (1) -- (2);
\draw [->] (2) -- (3);
\draw [->] (3) -- (4);
\draw [->] (4) -- (5);
\coordinate[name = notarrive7, point, right of = 7, node distance = 1em];
\draw [->] (6) -- (notarrive7);
\coordinate[name = before7, above of = 7, node distance = 1em]{};
\draw [->] (before7) -- (8);
\draw [->] (8) -- (9);
\coordinate[name = notarrive10, point, right of = 10, node distance = 1em]{};
\draw [->] (9) -- (notarrive10);

\node (findfirstrect)[minimum width = 40em, minimum height = 4em, right of = 1, node distance = 13em, rotate = 355] {TextInputClientMsg\_FirstRectForCharacterRange};
\node (javascript)[minimum width = 4em, minimum height = 4em, above left = 0em and 2em of 7, rotate = 15] {Run\_Javascript};
\end{tikzpicture}
}
\end{center}
%%\end{document}

    \caption{Chromium Example}
    \label{fig:chromium-case-study}
\end{figure}
%   resolve symbol, save to log
%     search for set\_spinning
%     if found,
%       find main thread node at the time of set\_spinning
%     find fontd
%     manually check nodes in each thread immediately after the nodes in the slice (normal abnormal boundary)
%  output is a node, and a HTML dump of node and immediate predecessors and successors
% systems preference          
%  spinning node in UI thread is not waiting
%  so we look for messages, and find the diff

One of the authors experienced first-hand the aforementioned performance issue
in Chromium, an open-source browser engine that powers Google Chrome and,
starting recently, Microsoft Edge~\cite{chromiumurl}.  She tried to type in the
Chromium search box a Chinese word using SCIM, the default Chinese Input Method
Editor that ships with MacOS.  The browser appeared frozen and the spinning
cursor occurs for a few seconds.  Afterwards everything went back to normal.
This issue is reproducible and always ruins her experience, but it is quite
challenging to diagnose because two applications Chromium and SCIM and many
daemons ran and exchanged messages.  This issue was reported by other users for
other non-English input methods, too.

To diagnose this issue with \xxx, the author started system-wide tracing, and
then reproduced the spinning cursor with a Chinese search string typed via SCIM
while the page was loading. It produced normal cases for the very first few
characters, and the browser got blocked with the rest input as spinning cases.
The entire session took roughly five minutes.

She then ran \xxx to construct the event graph. The graph had 2,749,628 vertexes
and 3,606,657 edges, almost fully connected. It spans across 17 applications;
109 daemons including \vv{fontd}, \vv{mdworker}, \vv{nsurlsessiond} and
helper tools by applications; 126 processes; 679 threads, and 829,287
IPC messages. Given the scale of the graph and the diverse communication
patterns, it would be extremely challenging for prior automated causal tracing
tools~\cite{aguilera2003performance, zhang2013panappticon, attariyan2012x,
cohen2004correlating} because they handle a fairly limited set of patterns.
Tools that require manual schema~\cite{barham2004using, reynolds2006pip}, would
be prohibitive because developers would have to provide schema for all involved
applications and daemons.

Next she ran \xxx to find the \spinningnode in the main thread of the browser
process. \xxx returned a \vv{Wait} event on condition variable with timeout that
blocked the main thread for a few seconds. Thus \xxx compares the \spinningnode
to a similar one in normal case where the \vv{Wait} was signaled quickly with
Algorithm~\ref{alg:alg-findsimilarnode}. \xxx reported three, and confirmed with
the user which one she wanted.

\xxx then found the normal-case wake-up path which connects five threads. The
browser main thread was signaled by a browser worker thread, which received IPC
from a worker thread of \vv{renderer} where the rendering view and WebKit code
run. The worker thread is woken up by the \vv{renderer} main thread, which in
turn woken by fontd, the font service daemon. \xxx further compared the wake-up
path with the spinning case with the Algorithm ~\ref{alg:alg-graphdiff}, and
returned the \vv{wait} event on semaphore in the \vv{renderer} main thread, the
culprit that delayed waking up the browser main thread over 4 seconds.

What caused the wait in the \vv{renderer} main thread though? She thus continued
diagnosis and recursively applied \xxx to the wait in \vv{renderer}, and got
the wake-up path. The culprit that delayed the semaphore was the timeouts in
the browser's main thread. At this point, a circular wait formed. To understand
what exactly happens in the situation, she inspected the full call stacks by
\xxx scripts, taking the reported vertice from the renderer and the browser as
input. Inspection reveals that the \vv{renderer} requested the browser's help to
render Javascript and waited for reply with semaphore. The browser was waiting
for the \vv{renderer} to return the string bounding box and the \vv{renderer}
was waiting for the browser to help render Javascript. This circular wait was
broken by a timeout in the browser main thread (the \vv{wait} on cv timeout was
1,500 ms). While the system was able to make progress, the next key press caused
the spinning cursor to display for another 1,500 ms. The timeout essentially
converted a deadlock into a livelock.

The finding was verified with chromium source code. Shortening the timeout interval
in the main browser thread proportionally shortens the waiting of the main
render thread on processing the javascript. Skipping certain javascripts
processing in the renderer thread cuts down the success rate of spinning case
reproducing.

%%Next she ran \xxx to find the event in the main thread of the browser process.
%%\xxx returned a \vv{cv\_timed\_wait} event %(Figure~\ref{fig:chromium-trace})
%%that blocked the main thread for a few seconds. Inspection of the
%%lightweight call stack revealed that this wait happened within a call to
%%\vv{TextInputClientMac::GetFirstRectForRange}. Without knowing the application's
%%semantics, she could not understand this method. Thus she ran \xxx to compare
%%the spinning case to a normal case. \xxx searched in the main thread of the
%%browser process for vertexes similar to this wait waiting vertexes similar to
%%this wait, found three, and confirmed with the user which one she wanted.
%%
%%\begin{figure*}[p]
%%    \centering
%%	%%\documentclass{article}
%%\usepackage{tikzpeople}
%%\usepackage{tikz}
%%\usetikzlibrary{shapes, shapes.misc}
%%\usetikzlibrary{arrows, arrows.meta, decorations.markings}
%%\usetikzlibrary{patterns}

%%\begin{document}
\begin{center}
\resizebox{\columnwidth}{!}{%
\begin{tikzpicture}[>=latex]
\tikzstyle{every node}=[font=\Large]
\tikzstyle{app} = [draw, very thick, minimum height=40em, minimum width=20em, fill=white, rectangle, font={\sffamily\bfseries}]
\tikzstyle{actionpoint} = [minimum width = 1em, fill = white]
\tikzstyle{point} = [thick, draw=red, cross out, inner sep = 0pt, minimum width = 0.5em, minimum height =0.5em]
%\tikzset{
%%ext/.pic={
%%\path [fill=white] (-0.2,0)to[bend left](0,0.1)to[bend right](0.2,0.2)to(0.2,0)to[bend left](0,-0.1)to[bend right](-0.2,-0.2)--cycle;
%%\draw (-0.2,0)to[bend left](0,0.1)to[bend right](0.2,0.2) (0.2,0)to[bend left](0,-0.1)to[bend right](-0.2,-0.2);
%%},
%point/.style={
%    thick,
%    draw=red,
%    cross out,
%    inner sep=0pt,
%    minimum width=4pt,
%    minimum height=4pt,
%}}

%draw applications
\node [app, name = browser]{};
\node (browsertag)[minimum width = 20em, above = -2em of browser] {Chromium Browser};
\node (browserend)[minimum width = 20em, below = 38em of browsertag] {};
\node (bt1begin)[minimum width = 8em, below left = 3em and -10em of browsertag]{main thread};
\node (bt1end)[minimum width = 8em, below left = 38em and -10em of browsertag]{};
\node (bt2begin)[minimum width = 8em, below right = 3em and -10em of browsertag]{worker thread};
\node (bt2end)[minimum width = 8em, below right = 38em and -10em of browsertag]{};
\draw [solid] (bt1begin) -- (bt1end);
\draw [solid] (bt2begin) -- (bt2end);
\node (1)[actionpoint, below of = bt1begin, node distance = 8em]{1};
\node (2)[actionpoint, below of = bt2begin, node distance = 9em]{2};
\node (9)[actionpoint, below of = 2, node distance = 16em] {9};
\node (10)[actionpoint, below of = 1, minimum width = 6em, minimum height = 2em, node distance = 18em, rotate=270]{timed wait};
%\node (11)[actionpoint, below of = 10, node distance = 4em]{};

\node [app, name = renderer, right of = browser, node distance = 25em]{};
\node (renderertag)[minimum width = 20em, above = -2em of renderer] {Chromium Renderer};
\node (rendererend)[minimum width = 20em, below = 38em of renderertag]{};
\node (bt3begin)[minimum width = 8em, below left = 3em and -10em of renderertag]{worker thread};
\node (bt3end)[minimum width = 8em, below left = 38em and -10em of renderertag]{};
\node (bt4begin)[minimum width = 8em, below right = 3em and -10em of renderertag]{main thread};
\node (bt4end)[minimum width = 8em, below right = 38em and -10em of renderertag]{};
\draw [solid] (bt3begin) -- (bt3end);
\draw [solid] (bt4begin) -- (bt4end);
\node (3)[actionpoint, below of = bt3begin, node distance = 10em]{3};
\node (4)[actionpoint, below of = bt4begin, node distance = 11em]{4};
\node (7)[actionpoint, below of = 4, node distance = 11em, rotate=270] {sema};
\node (8)[actionpoint, below of = 3, node distance = 13em]{8};

\node [app, name = fontd, right of = renderer, node distance = 25em]{};
\node (fontdtag)[minimum width = 20em, above = -2em of fontd] {fontd};
\node (fontdend)[minimum width = 20em, below = 38em of fontdtag]{};
\node (bt5begin)[minimum width = 20em, below = 3em of fontdtag]{worker thread};
\node (bt5end)[minimum width = 20em, below = 38em of fontdtag]{};
\draw [solid] (bt5begin) -- (bt5end);
\node (5)[actionpoint, below of = bt5begin, node distance = 14em] {5};
\node (6)[actionpoint, below of = 5, node distance = 3em] {6};

\draw [->] (1) -- (2);
\draw [->] (2) -- (3);
\draw [->] (3) -- (4);
\draw [->] (4) -- (5);
\coordinate[name = notarrive7, point, right of = 7, node distance = 1em];
\draw [->] (6) -- (notarrive7);
\coordinate[name = before7, above of = 7, node distance = 1em]{};
\draw [->] (before7) -- (8);
\draw [->] (8) -- (9);
\coordinate[name = notarrive10, point, right of = 10, node distance = 1em]{};
\draw [->] (9) -- (notarrive10);

\node (findfirstrect)[minimum width = 40em, minimum height = 4em, right of = 1, node distance = 13em, rotate = 355] {TextInputClientMsg\_FirstRectForCharacterRange};
\node (javascript)[minimum width = 4em, minimum height = 4em, above left = 0em and 2em of 7, rotate = 15] {Run\_Javascript};
\end{tikzpicture}
}
\end{center}
%%\end{document}
 
%%    \caption{Chromium case study.}
%%    \label{fig:chromium-trace}
%%\end{figure*}
%%%need to adjust based on the new figure
%%\xxx then found the normal-case wake-up path shown in the figure, which
%%connects five threads.  The browser main thread was signaled by a browser
%%worker thread as shown in step \textcircled{1} of backward slicing in Figure
%%\ref{fig:chromium-trace}, which in turn \vv{read\_file} in step \textcircled{2}
%%for IPC from a worker thread of \vv{renderer}, the daemon for rendering screens.
%%The \vv{renderer} worker thread is woken up by the \vv{renderer} main thread to
%%\vv{read\_file} \textcircled{3}, which in turn \vv{recv\_msg} \textcircled{4}
%%from \vv{fontd}, the font service daemon.  From this path, we could guess that
%%\vv{GetFirstRectForRange} was for the browser to understand the bounding box of
%%the search string.  \xxx further compared the wake-up path with the spinning
%%case, and returned the \vv{wait\_semaphore} event in the \vv{renderer} main
%%thread, the culprit that delayed waking up the browser main thread over 4
%%seconds.

%%What caused the wait in the \vv{renderer} main thread though?  She thus
%%continued diagnosis and recursively applied \xxx to the wait in \vv{renderer},
%%and got the wake-up path shown in the figure for this wait.  Inspection reveal
%%that the \vv{renderer} requested the browser's help to render Javascript and was
%%waiting for a reply.  At this point, a circular wait formed because the browser
%%was waiting for the \vv{renderer} to return the string bounding box and the
%%\vv{renderer} was waiting for the browser to help render Javascript.  This
%%circular wait was broken by a timeout in the browser main thread (the
%%\vv{cv\_timed\_wait} timeout was 1,500 ms).  While the system was able to make
%%progress, the next key press caused the spinning cursor to display for another
%%1,500 ms.  The timeout essentially converted a deadlock into a livelock.

\subsection{Limitations}
\xxx is designed to support interactive debugging of performance
issues. It sometimes requires the user to reproduce a performance issue so
\xxx can capture more fine-grained event traces such as accesses to data
flags.  Fortunately, a performance issue that almost never reproduces is
probably not as annoying as one that occurs frequently.

We implemented \xxx in the closed-source macOS which presents a harsh test
for \xxx, but we have not ported \xxx to other operating systems yet. It is
possible that the ideas and techniques do not generalize to other operating
systems. However, modern operating systems share many similarities, and good
ideas tend to flow both ways, so we are hopeful that the ideas in \xxx are
generally applicable. Similarly, the applications and performance issues used in
our evaluation may be non-representative, though we strive to cover a
diverse set of common applications ranging from browsers to text editors.
