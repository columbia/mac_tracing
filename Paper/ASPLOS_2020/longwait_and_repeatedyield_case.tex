\subsection{Long Wait and Repeated Yield}
In this section, we discuss the cases where the \spinningnode is blocking
on wait event or yielding loop, corresponding to Long Wait and
Repeated Yield.

\paragraph{2-SystemPref}

System Preferences provides a central location in macOS to customize
system settings, including configuring additional monitors. A tool called
\vv{DisableMonitor}~\cite{disablemonitor} provides full functionality including
the ability to enable/disable monitors online. We blocked on the spinning cursor
while disabling an external monitor and rearranging windows in \vv{Display}
panel.

The log collected with \xxx contains 1) a baseline scenario where the displays
are rearranged with the enabled external monitor, and 2) a spinning scenario in
which we disable the external monitor with \vv{DisableMonitor} and rearrange
the displays. The \spinningnode in the main thread is dominated by system
calls, \vv{mach\_msg} and \vv{thread\_switch}, which falls into the category of
 Repeated Yield. We discovered two missing data flags with \vv{lldb},
``\vv{\_gCGWillReconfigureSeen}'' and ``\vv{\_gCGDidReconfigureSeen}'', which
signify the configuration status and break the thread-yield loop. \xxx
learns from the baseline scenario that the main thread is responsible to set
both of them after receiving specific datagrams from WindowServer. Conversely,
the setting of ``\vv{\_gCGDidReconfigureSeen}'' is missing in the spinning case,
where the main thread yields repeatedly to send messages to WindowServer for
such datagram.

In conclusion, we discovered that the bug is inherent in the design of the
CoreGraphics library, and would have to be fixed by Apple. We verified this
diagnosis by creating a dynamic binary patch with lldb to fix the deadlock. The
patched library makes DisableMonitor work correctly, while preserving correct
behavior for other applications.

\paragraph{3-SequelPro}

Sequel Pro~\cite{SequelPro} is a fast, easy-to-use Mac database management
application for working with MySQL databases. It allows user to connect to
database with a standard way, socket or ssh.

We experienced the non-responsiveness of Sequel Pro while its network connection
got lost and it tried re-connections. The tracing data collected by \xxx
contains 1) a quick network connection during login, and 2) Sequel Pro lost
connection for a while. Although \xxx identified the \spinningnode and
corresponding (baseline) \similarnode with ease, it cannot get the correct
causal path in the baseline scenario without user interaction. The backward
slicing on vertex has multiple incoming edges, including one from a kernel
thread, which means that operations are likely to be batched together and
inseparable by heuristics. Our interactively search is extremely helpful
in this step, greatly reducing the noise in the path. Close examination of
the \spinningnode based on the causal path tells us that the main thread is
waiting for the kernel thread, which in turn waits for the ssh thread. Existing
debugging tools like \vv{lldb} and \vv{spindump} cannot determine the
root cause, because both of them diagnose with only call stacks, missing the
dependency across processes.

\paragraph{4-Installer}

Installer~\cite{Installer} is an application included in macOS that extracts and
installs files out of \vv{.pkg} packages. When \vv{Installer} pops up a window
for privileged permission during the installation of \vv{jdk-7u80-macosx-x64},
moving the cursor out of the popup window triggers a spinning cursor.

As we put in the password before the round of triggering the spinning cursor,
\xxx successfully records the baseline scenario. Examining the \spinningnode and
its \similarnode, \xxx figures out the daemon \vv{authd} blocks on semaphore
while the main thread is waiting for \vv{authd}. Further checking on \vv{authd},
\xxx reveals it is the \vv{SecurityAgent} that processes user input and wakes
up \vv{authd} in baseline scenario. In conclusion, moving the mouse out of the
authentication window causes the missing edge from \vv{SecurityAgent} to
\vv{authd}, which in turn blocks \vv{Installer}.

We also discovered a communication pattern in \vv{Installer} underpinning the
crucial of interactive debugging. It involves four vertices in four threads,
vertex $Vertex_{main}$ in the main thread, and $Vertex_1$ to $Vertex_3$ in
three worker threads. First, the main thread wakes up three worker threads.
Then one worker thread is scheduled to run. At its end, another worker thread,
which waits on mutex lock, is woken in $Vertex_2$, which in turn wakes up the
next worker thread in $Vertex_3$. While \xxx is slicing backward, $Vertex_3$
has two incoming edges: one is from $Vertex_{main}$, and the other one is from
$Vertex_2$. Since users can peek the edges before making decision, they are
likely to figure out that the three worker threads contend with mutex lock, and
all of them are successors of $Vertex_{main}$.

