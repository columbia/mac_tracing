\subsection{\xxx Algorithms} 
The main algrithem of \xxx is shown in figure ~\ref{fig:alg-main}.
\begin{figure}[H]
\footnotesize\begin{verbatim}
Main Algorithm:
    Input: Program to run + buggy test case
        + (optional) similar baseline test case
    Output: Sequence of actions or UI events which
        trigger the performance problem, and trace
        of execution across event handlers

1. Run program under \xxx, trigger baseline case
    (if possible) and then buggy test case
2. Set heuristics = { default heuristics }
3. Compute graph using current heuristics with
    Algorithm ComputeControlFlowGraph
4.a. If the buggy case is busy executing code (livelock):
    run backward traversal of edges until UI event found
4.b. Else, given a hanging node N_1, use Algorithm
    FindSimilarNode to obtain an equivalent
    in the baseline test case,
    N_0 -> FindSimilarNode(N_1)
    5.a. Compare nodes N_0 and N_1, and automatically
        diff the two cases, moving through history
        semi-automatically with user input
	    until useful UI events are found
        [Algorithm AssistedGraphDiff]
    5.b. If new heuristics were added, go to step 3.
6. Return set of UI events found
\end{verbatim}
    \caption{Main \xxx algorithm.}
    \label{fig:alg-main}
\end{figure}

ComputeControlGraph takes the heuristics set and the parsed trace
events as input and produce a event graph as is described in
Section \S\ref{subsec:eventgraph}. The algorithm is shown in
Figure~\ref{fig:alg-graphcomputing}. The graph is subject to improve by rerun
the computing process with new heuristics.

FindSimilarNode in Figure~\ref{fig:alg-findsimilarnode} is meant to find nodes
which have the same high level semantics but different execution results.
For example, both of them are waiting on locks, but one returns successfully and the other not.
\xxx assumes that high level statements are likely executed repeatedly in a process life cycle.
In the algorithm, \xxx defines a subset of
events that peserve semantics, which contains mach\_msg event, wait event,
dispatch queue events, runloop events, breakpointer events, user input events
and system call events.
The comparison of events depends on their types.
\xxx finds nodes that have different wait results from the spinning node by default.
Depending on the report detail of the spinning node, user can
change the comparing metrics.

\begin{figure}[H]
\footnotesize\begin{verbatim}
Algorithm FindSimilarNode:
  Input: Checking Events + CurrentNode + Graph
  Output: Similar node set
1. Get the thread id of CurrentNode
2. Iterate the node from the same thread N_i, 
   compare them to CurrentNode N_c:
   while event{iter_c} in N_c is not checking event
   and iter_c not reaches end
   	 iter_c++
   while event{iter_i} in N_i is checking event
   and iter_i not reaches end
     iter_i++
   if iter_c and iter_i do not reach end:
      Compare Event{iter_c} and Event{iter_i}:
        2.a wait events compares the wait resource
        2.b connection events compare their peer
        2.c system call events and user input events
            compare their syscallname and eventname
   if iter_c and iter_i both reach end:
      put N_i into the result set
3. filter the result set if:
   3.a if the wait events in the end of node
       have the same wait result
   3.b if the difference of node time span
       is within threshold
4. return the result set
\end{verbatim}
    \caption{\xxx Find similar node algorithm.}
    \label{fig:alg-findsimilarnode}
\end{figure}

\begin{figure}[H]
\footnotesize\begin{verbatim}
Algorithm AssistedGraphDiff:
  Input: Backward slicing path from the baseline Node
          + Spinning Node + Graph
  Output: Possible root cause of thread blocking
  1. get timestamp of Spinning Node as t_spinning
  2. For every thread, node in the backware slicing path
     get timestamp of node in the baseline as t_normal
     check if thread block during (t_normal. t_spinning)
     if true, put the blocking node in the result set
  3. return the result set
\end{verbatim}
    \caption{\xxx Assisted graph diff algorithm.}
    \label{fig:alg-graphdiff}
\end{figure}
