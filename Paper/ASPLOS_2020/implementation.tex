\section{Implementation}\label{sec:implementation}
We now discuss how we collect tracing events from both kernel and libraries.

\subsection{Event Tracing}
Current MacOS systems support a system-wide tracing infrastructure built by
Apple XXX~\ref{linktotracetool}. By default, the infrastructure temporarily stores
events in memory and flushes them to screen or disk when an internal buffer is
filled. We extended this infrastructure to support larger-scale tests and avoid
filling up the disk with a file-backed ring buffer. Subject to configurarion,
it allows at most 2GB of data per log, which corresponds to approximately
18,560,187 events (about 5 minute with normal operations).

The default tracing points in MacOS provide too limited information to
apply causual tracing. As a result, we both patch source code of kernel
XXX~\ref{linkofxnusourcecode} and binary instrument libraries to gether
more tracing data. In \xxx, we patched the kernel with 1193 lines of
code, and instrumented the libraries including: libsystem\_kernel.dylib,
libdispatch.dylib, libpthread.dylib, CoreFoundation, CoreGraphics, HIToolbox,
AppKit and QuartzCore, with our binary instrumentation.

\subsection{Instrumentation}
Most libraries as well as many of the applications used day-to-day are
closed-source in MacOS. To add tracing points to such code, techniques such as
library preloading to override individual functions are not applicable on MacOS,
as libraries use two-level executable namespaces~\cite{}. Hence, we implemented
a binary instrumentation mechanism that allows developers to add tracing at
any location in a binary image.

Like Detour~\cite{hunt1999detours}, we use static analysis to decide which
instrumentation to perform, and then enact this instrumentation at runtime.
Firstly, users find a location of interest in the image related to a specific
event by searching a sequence of instructions. Then the users replace a call
instruction to invokes a trampoline target function, in which we overwrite the
victimed instructions and produce tracing data with API from Apple. All of the
trampoline functions are grouped into a new image, as well as an initialization
function which carries out the drop-in replacement. Then command tools from
\xxx helps to configure the image with the following steps: (1)re-export all
symbols from the original image so that the original code can be called Like an
shared library; (2)replace the original image with the new one by renaming them
to ensure the modifications are properly loaded; (3)invoke the initialization
function externally through \texttt{dispatch\_once} during the loading.

%%One potential issue is that we use 5-byte call instructions with 32-bit
%%displacements to jump from the original library to our new one.  This design
%%requires that the libraries be loaded within +/- 2GB of each other in the
%%64-bit process address space.  However, since we list each original library as
%%a dependency of our new libraries, the system loader will map each new and
%%original library in sequence.  In practice, the libraries ended up very close
%%to one another and we did not see the need to implement a more general
%%long-jump mechanism.

\subsection{User Interaction}\label{subsec:tcp}

\para{Tracing Custom Primitives}
%%XXX give a simple command line example of how a user can ask \xxx to trace a
%%data flag
%%XXX say what we do in watch point exception handler (record instruction so
%%can determine read or write, and reg values)
As described in (\S\ref{subsec:userinteraction}), under-connection due
to the missing share data dependency requires users'interaction. \xxx
provides a command line tool which sets the watch point registers to record
share\_flag\_write and share\_flag\_read events in ad-hoc manner. This is one
way users can easily collect the tracing events, and the more tech-savvy users
can also instrument the binary. The tool takes the process id, path to image
where the variable is defined and the symbol of the variable as input. We
show the simple example how a user ask \xxx to trace \_gOutMsgPending in the
following command.

\begin{lstlisting}
./bp_watch pidofWindowServer Path/to/CoreGraphics\
	_gOutMsgPending
\end{lstlisting}

\xxx hooks the watch point break handler in CoreFoundation to make sure that it is
loaded correctly into the address space of our target application. The handler
invokes the event tracing API from Apple to record the value of the shared
variable and the operation type: read or write.
\\
\para{Capturing Instructions for Diagnosis}
%XXX Talk about what data we gather using lldb, the debugger in the LLVM
%compiler tool chain.
Since the output of \xxx is the nodes and user input events suspected to cause
the busy spinning, more detail may required to verify and fix the bug. After
the offline analysis on the graph, \xxx provides tools for user to exact
the backtrace events from the output and generate a script for conditional
debugging.

The debugging scripts go through the instructions of apps and frameworks step
by step to capture the parameters tained by user inputs. At each beginning of
a function call, the script records a full callstack for it. Considering the
overhead and usefullness, it steps over and only record the return value of APIs
from libraries with a filename extension .dylib.

The supplementary information are subject to the users review to pinpoint the
root cause of spinning beachball on MacOS.

\subsection{Limitations}
\xxx is designed to support interactive debugging of performance issues. To
incrementally obtain more fine-grained event traces, it needs to rerun an
application to reproduce a performance issue. Thus, if the issue is difficult to
reproduce, we have to rely on the log collected by the lightweight system-wide
tracing for debugging, and lose the benefits of interactivity. Fortunately, a
performance issue that almost never reproduces is probably not as annoying as
one that occurs frequently.

We implemented \xxx in the closed-source MacOS which presents a harsh test
for \xxx, but we have not ported \xxx to other operating systems yet. It is
possible that the ideas and techniques do not generalize to other operating
systems. However, modern operating systems share many similarities, and good
ideas tend to flow both ways, so we are hopeful that the ideas in \xxx are
generally applicable. Similarly, the applications and performance issues used in
our evaluation may be non-representative.
