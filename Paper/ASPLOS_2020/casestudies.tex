\section{Case Studies}\label{sec:casestudy}

In this section, we demonstrate how \xxx helps to diagnose \nbug spinning-cursor
cases in \napps popular applications. Table~\ref{table:bugs-desc} describes
these spinning-cursor cases.

We studied how WindowsServer and NSEvent thread per application coordinate to
determine when a spinning cursor needs to be displayed because it is fundamental
in understanding all spinning cursor cases yet not straightforward. Two data
flags ``\vv{is\_mainthread\_spinning}'' and ``\vv{dispatch\_to\_mainthread}''
are involved. We refer to this case XXXBug-ID.

%TableXXX describes these spinning-cursor cases.
%Should have a table for issue description
%Bug ID |  Application  |  Bug Description
\begin{table}[ht]
\footnotesize
\centering
  \begin{tabularx}{\columnwidth}{l|c|c}
    \hline
    \textbf{Bug ID} & \textbf{Application} & \textbf{Bug description}\\
    \hline
	 0 & WindowServer & display spinning cursor\\
    \hline
	 1 & \begin{tabular}{@{}c@{}} 
	 System Preferences\\
	 %:system settings\\
	 %customization app.
	 \end{tabular}
	 & \begin{tabular}{@{}c@{}}
	 Disabling an online\\
	 external monitor and\\
	 rearranging windows\\
	 cause the freeze.
	 \end{tabular}
	 \\
     \hline
	 2 & \begin{tabular}{@{}c@{}} 
	 Sequel Pro
	 \end{tabular}
	 & \begin{tabular}{@{}c@{}}
	 Lost connection freezes\\
	 the whole application.
	 \end{tabular}
	 \\
     \hline
	 3 & \begin{tabular}{@{}c@{}} 
	 TeXStudio
	 %: LaTeX editor
	 \end{tabular}
	 & \begin{tabular}{@{}c@{}}
	 Modification on bib file\\
	 from vim causes the main\\
	 window spinning.
	 \end{tabular}
	 \\
     \hline
	 4 & \begin{tabular}{@{}c@{}} 
	 Installer
	 \end{tabular}
	 & \begin{tabular}{@{}c@{}}
	 Move cursor out of the\\
	 authentication window \\
	 causes spinning.
	 \end{tabular}
	 \\
     \hline
	 5 & \begin{tabular}{@{}c@{}} 
	 Notes\\
	 %note-taking \\
	 %app from Apple 
	 \end{tabular}
	 & \begin{tabular}{@{}c@{}}
	 launching Notes with \\
	 a saved relatively\\
	 long note.\\
	 \end{tabular}
	 \\
     \hline
	 6 & \begin{tabular}{@{}c@{}}
	 TextEdit
	 %:rich text\\
	 %editing app from Apple
	 \end{tabular}
	 & \begin{tabular}{@{}c@{}}
	 Select, copy, paste\\%delete, insert and save
	 on text over 30M\\
	 cause freeze.
	 \end{tabular}
	 \\
     \hline
	 7 & \begin{tabular}{@{}c@{}}
	 Microsoft Words
	 \end{tabular}
	 & \begin{tabular}{@{}c@{}}
	 copy and paste on\\
	 document over 400\\
	 pages cause hang.
	 \end{tabular}
	 \\
     \hline
	 8 & Sublime Text
	 & \begin{tabular}{@{}c@{}}
	 Copy and paste in file\\
	 over 49000 lines.
	 \end{tabular}
	\\
    \hline
	 9 & TextMate 
	 & \begin{tabular}{@{}c@{}}
	 Paste text over 4000\\
	 lines causes spinning.
	 \end{tabular}
	\\
    \hline
	 10 & CotEditor
	 & \begin{tabular}{@{}c@{}}
	 Paste in file over 4000\\
	 lines causes spinning.
	 \end{tabular}
	\\
	 \hline
  \end{tabularx}
  \caption{Bug Descriptions}
  \label{table:bugs-desc}
\end{table}


%Start a new paragraph
%"TableXXX shows the results.  [Here we should show the big table, and talk about the high-level bits.]"

We compared \xxx with traditional causual tracing methods~\cite{} on edges and
vertices \xxx mitigated. and studied the ratios of over-connection vertices
avoided with message peers hueristically, and the under-connection edgs added by
both data flags and wait event inside a block. User interactions are critical in
the path slicing due to the inherent incorrect in the graph. Our results shows
the user interaction does not overwhelm the debugging process while it makes
the debugging much more efficient. Table~\ref{table:results} demonstrates our
results with the critical path sliced with \xxx, which is much shorter than the
ones generated automatically by choosing the nearest edge hueristically.


\begin{table*}[ht]
\footnotesize
\centering
  \begin{tabularx}{\textwidth}{l|cccccc}
 	   & rate of vertixes   & rate of vertixes  &           & length of           & length of    & \# of\\
       & filted by          & preserved by      & \# of     & baseline/spinning& baseline/spinning&user\\
Bug ID & message hueristics & wait hueristics& data flag & path slicing        & length of  &interaction \\
\hline
\hline
 1-SystemPreferences&&&&&&\\
 2-SequelPro & 0.3\% & 0.21\% & 3 & 5 &  & 2  \\
 3-TeXStudio & 4.8\% & 1.7\% & 3 & 6 &  & 3 \\
 4-Installer &&&&&& \\
 5-Notes &&&&&& \\
 6-TextEdit(copy)&2.9\% &13.6\% & 3 & 21 & & 5\\
 7-MSWord(copy)&6.7\%&1.0\%& 3 &67& & 22\\
 8-SublimeText(copy)&4.0\%& 0.9\%&3 & 3 & & 1\\
 9-TextMate(paste) & 4.2\% & 4.2\% &3 & 23& & 0\\
 10-CotEditor(paste) & 4.8\%& 5.3\%& 3 & 4 & & 1\\
\hline
  \end{tabularx}
  \caption{Graph Comparison}
  \label{table:results}
\end{table*}

%Last paragraph:
In the remaining of this section, we present the case studies by category in (\S\ref{XXX}). 

\subsection{Long Running}

In this section, we discuss the cases where the spinning node is busy on the
CPU. Most of the text editing apps fall into this bug category. We studied
TeXstudio, TextEdit, Microsoft Word, Sublime Text, Text Mate and CotEditor to
reveal their root causes.

\paragraph{TeXStudio}
\subsection{TeXstudio}

TeXstudio is an integrated writing environment for creating LaTeX documents. In
their github we noticed some one reported while he was modifying his bib file,
the TeXstudio hanged for minutes. Although the issue was closed by author for
incomplete information for reproduction, we reproduce it in our machine. With a
relatively large bib file from a LaTex project opened in a tab, once we touch
the bib file through other editors, vim for exmaple, we would have a spinning
beachball on the window.

While the beachball is spinning over the application window, the main thread
is busy with \textit{QDocumentPrivate::indexOF(QDocumentLineHandle const*,
int)}. Slicing the path from the busy node, \xxx tells the busy processing is
invoked by the callback from daemon fseventsd. The advantage of \xxx over other
debugging tools is it helps to narrow down the root cause with the
path. If the user reported the bug with \xxx' report, it should have provided
the author more hints to reproduce the bug successfully.

Another operation triggering spinning beachball in TeXstudio is pasting
a large amount of context in any file. The spinning node is busy with
\textit{QEditor::insertFromMimeData(QMimeData const*)}, which always invokes
\textit{match(QNFAMatchContext*, QChar const*, int, QNFAMatchNotifier)}. Path
slicing by \xxx attributes it to the user's keybord inputs of cmd+v. The
problematic code resides where developers copy data from the pasteboard.


\paragraph{Other Editing Apps}

Select, copy, paste, delete, insert and save are common operations for text
editing. However, these operations on a large acontext usually trigger spinning
cursors. Depending on their implementations, Softwares like XXX successfully
avoid hangs on XXX operation. \xxx can helps the developer to figure out the
more efficient way to implement those operations. We briefly list the reports
from the spinning node and its corresponding user input from the path slicing
here. The path usually involves several daemons, Figure~\ref{XXX} illustrates
the spinning case in XXX.

\begin{itemize}
  \item \textbf{Notes}:
	  While launching the Notes, it is busy with
	  \vv{NSDetectScrollDevicesThenInvokeOnMainQueue}.

  \item \textbf{TextEdit}:
	  \xxx reports CPU busy on the storage of characters \vv{[NSBigMutableString\
	  getCharacters:range:]} and \vv{NSFastFillAllGlyphHolesForCharacterRange} for
	  selection.

	  When copy spins, \xxx reports
	  \vv{get\_vImage\_converter} and \vv{get\_full\_conversion\_code\_fragment}
	  occupy the main thread to finish \vv{[NSTextView(NSPasteboard)\
	  \_writeRTFDInRanges:toPasteboard:]}.
	  
	  Paste operation causes the main thread overwhelmed by \vv{\_RTFGetToken}
	  and \vv{platform\_memmove}. Both of them are called by the handler
	  \vv{-[NSTextView(NSPasteboard)\ \_readSelectionFromPasteboard:types:]}

  \item \textbf{Microsoft Words}:
	  Both copy and paste in Microsoft Words are overwhelmed by system calls
	  \vv{lseek}, \vv{fstat64}, \vv{fcntl} on the same file descriptor.
	  In copy operation, \vv{\_\_CFStringCreateImmutableFunnel3} and
	  \vv{platform\_memmove} are invoked for \vv{-[NSPasteboard\
	  \_setData:forType:index:usesPboardTypes:]}, while the paste operation is
	  dominated by \vv{platform\_memmove} and \vv{}.

  \item \textbf{Sublime Text}:
	  Copy in SublimeText is busy encoding characters with \vv{\_\_CFToUTF8Len They}
	  are called from handler \vv{px\_copy\_to\_clipboard}.

	  Paste is dominated by \vv{decode\_utf8}, \vv{convert\_utf8\_to\_utf32}, string
	  \vv{append} and \vv{platform\_memmove} from \vv{px\_copy\_from\_clipboard}.

	  Delete on Sublime Text causes its main thread is busy processing
	  \vv{TokenStorage::substr}.

  \item \textbf{TextMate}:
	  TextMate only spins on the paste operation in our experiments, the main
	  thread is busy with \vv{-[OakTextView paste:]} which iterates through
	  paragraphs, fetches characters and processed with \vv{CFAttributedStringSet} and
	  \vv{TASCIIEncoder::Encode}.

  \item \textbf{CotEditor}:
	  Similar to TextMate, CotEditor does not spinning on copy. In paste and hitting
	  a return key cause the main thread busy on processing \vv{-[NSBigMutableString
	  characterAtIndex:]} and \vv{CFStorageGetValueAtIndex} in the call stacks.
\end{itemize}

\subsection{Repeated Yield and Long Wait}
In this section, we discuss the cases where the spinning node is blocking on
wait event or yielding loop, correponding to \textbf{Long Wait} and \textbf{Repeated Yield}.
\paragraph{SequelPro}
Sequel Pro~\cite{SequelPro} is a fast, easy-to-use Mac database management
application for working with MySQL databases. It allows user to connect to
database with a standard way, socket or ssh.

We experienced the non-responsiveness of Sequel Pro while its network connection
got lost and it tried re-connections. The tracing data collected by \xxx
contains 1) a quick network connection during login, and 2) Sequel Pro lost
connection for a while. Although \xxx identified the \spinningnode and
corresponding (baseline) \similarnode with ease, it can hardly get the correct
causal path in the baseline scenario without a user's interaction. The backward
slicing on vertex, which has multiple incomming edges, including one from
kernel thread, makes it hard to rely on heuristics. Our interactively search is
extremely helpful in the step, greatly reducing the noise in the path. Close
examine of the \spinningnode based on the causual path tells that the main
thread is waiting for the kernel thread, which in turn waits for the ssh thread.

%\xxx helps to filter out overconnetions by dividing XXX combined batching node
%with hueristics, and recovers XXX missing edges due to the shared flag. In the
%diagnosis process, we get XXX similar node, XXX times of queries on backward
%slicing , XXX vertices for inspect.

Existing debugging tools like \vv{lldb} and \vv{spindump} could hardly figure
out the root cause, in that both of them diagnose with only call stacks, missing
the dependency across process boundaries.

\paragraph{Installer}
When Installer pops up a window for privileged permission during the
installation of \texttt{jdk-7u80-macosx-x64}, moving the cursor out of the popup
window triggers spinning cursor. Examing the \spinningnode reported by \xxx,
it is easy to figure out the main thread blocks in \textit{[IFRunnerProxy\
requestKeyForRights:askUser:]}. The function sends a synchronous message to
daemon \textit{authd}. The root cause is the authentication of user's previledge
synchronously in the main thread, instead of the handler for moving cursor.
%with \textit{xpc\_connection\_send\_messag\_with\_reply\_sync}.


\paragraph{SystemPreferences}
\subsection{System Preferences spin}
System Preferences is the application in MacOS for users to modify various settings.
Displays in the Panel allows user to rearrange the position of displays, but it does not support the disabling of a montor online.
DisableMonitor is an application used to complete the function, easily disabling/enabling  monitors withou unplug them.
Surprisingly, the operation of DisableMonitor exposed a performance bug in System Preferences.
If we disable an external monitor and arrange them afterward, the window System Preferences freezes for seconds.
We enabled Argus on the background and collect the data by normally arranging the displays and reapeating the spinning sequence.

It is not hard to tell the spinning node in the UI thread with our tool.
However, to find the normal node corresponding to the spinning node is not straightforward as in the previous case.
The execution segment includes two types of events: ``\v{mach\_msg}'' and ``\v{thread\_switch}''.
Both of them are used to waiting for the data available ping.
The semantics of the execution segment is not descriptive enough
to identify the normal nodes in the same operation stage.

In the case we detected the intensive timeout in the node,
our search algorithm identify the corresponding normal node by searching the similarity of their proceding nodes.
We find out the path of the normal path as shown in the Figure \ref{fig:path slicing for system preference}.
Our lightweight callstacks were used to verify the correctness of the findings.

The normal node showed it proceeded to \textit{displayReconfigured} after received the message with id 29675.
The spinning node fell into the ``\v{thread\_switch}'' after receiving the message with the same id,
and end up to send message for the available datagram ping with \textit{CGSSnarfAndDispatchDatagrams} again.
The proceding nodes before them sent message to WindowServer for \textit{activeDisplayNotificationHandler}.

With the result, we initiate the concrete debugging by filling the debugging script with the APIs reported.
We set the method \textit{activeDisplayNotificationHandler} as a beakpoint where the script begins debugging.
\textit{displayReconfigured} and \textit{CGSSnarfAndDispatchDatagrams} are recorded to indicate the end of debugging for the normal case and spinning case respectively.

Our debugging scripts ran within the confined range for both the noraml and the spinning execution.
The logs are generated as shown in Figure \ref{fig:step debug log for system preferences}.
By diff the two logs, it is easy for the user to notice the differen braches in \textit{display\_notify\_proc},
which is resulted from its prameter standing for the datagram type.

We make use of the diassembly tool, and reveals the story in the background.
Datagrams from the WindowServer makes applications to handle notifications.
The datagram causes difference are used to finish display reconfiguration for System Preference.
However, in the spinning case the reconfiguration got initiated but not completed.
The main thread leveraged thread\_switch to wait for the flollowing datagram and resulted in a freeze.
As a conclusion, the handler display\_notify\_proc is not appropriately implemented.

