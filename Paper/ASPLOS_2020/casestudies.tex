\section{Case Studies}\label{sec:casestudy}

In this section, we demonstrate how \xxx helps to diagnose \nbug spinning-cursor
cases in \napps popular applications. Table~\ref{table:bugs-desc} describes
these spinning-cursor cases, which are detailed in the sections that follow.
We compared \xxx with traditional causual tracing methods~\cite{XXXX} on edges
and vertices \xxx mitigated, and studied the ratios of vertices divided by
with message peers heuristically to reduce over-connection, and edges added
by both data flags and wait event inside a block to avoid under-connection.
User interactions are critical in the path slicing due to the inherent
inaccurate in the graph. Our results shows the user interaction does not
overwhelm the debugging process while makes the debugging much more efficient.
Table~\ref{table:results} demonstrates our results with the length of path
sliced with \xxx, which is much shorter than the one generated automatically by
heuristically choosing the most recent edges if multiple predecessors.

The spinning wait cursor is a painful sight for Mac users, signifying that the application is non-responsive.
It usually remains for munites, leaving the users at a loss .                    

Argus shreds light on the design of the spinning wait cursor with the backward path slicing.
We begin the path from the node where the spindump, a hang reporting tool, is launched,  
considering the spindump shares the same triggering condition.
As is shown in Figure \ref{fig:spincursor}, the spindump is launched by the message from WindowServer, after the WindowServe received message from the NSEvent thread of targeted Application.

We further added call stacks for the messages and revealed two shared variables are critical in the desgin, "is\_mainthread\_spinning" and "dispatch\_to\_mainthread".
The NSEvent thread of the targeted App fetches CoreGraphics events from WindowServer, converts and creates NSEvents for the main thread.
If the mainthread is not spinning by checking "is\_mainthread\_spinning", "dispath\_to\_mainthread" is set and a timer is armed.
If the main thread processes the next event before the timer fires, nothing happens and the timer gets reset.
Otherwise, NSEvent thread sends a message to WindowServer from the timer,
and WindowServer notifies the CoreGraphics to draw spinning cursor over the application window.


%Bug 0-Apple is ......

%The spinning cursor is created when the main thread stops responding to events for two seconds.
%Every application has an NSEvent thread, which coordinates with WindowsServer to display a spinning cursor when necessary. Two data
%flags ``\vv{is\_mainthread\_spinning}'' and ``\vv{dispatch\_to\_mainthread}''
%are involved.
%TableXXX describes these spinning-cursor cases.
%Should have a table for issue description
%Bug ID |  Application  |  Bug Description
\begin{table}[t]
\footnotesize
\centering
  \begin{tabularx}{\columnwidth}{l|cl}
    \hline
    \textbf{Bug ID} & \textbf{Application} & \textbf{Bug description}\\
    \hline
	\hline
	 0-Chormium & Chormium & \begin{tabular}{@{}l@{}}
	 Typing non-english in search box\\
	 causes webpage freeze.
	 \end{tabular}
	 \\
     \hline
	 1-SystemPref & \begin{tabular}{@{}l@{}} 
	 System Preferences\\
	 \end{tabular}
	 & \begin{tabular}{@{}l@{}}
	 Disabling an online external mo-\\
	 nitor and rearranging windows\\
	 causes System Preferences freeze.\\
	 \end{tabular}
	 \\
     \hline
	 2-SequelPro& \begin{tabular}{@{}l@{}} 
	 Sequel Pro
	 \end{tabular}
	 & \begin{tabular}{@{}l@{}}
	 Lost connection freezes the APP.
	 \end{tabular}
	 \\
     \hline
	 3-TexStudio & \begin{tabular}{@{}l@{}} 
	 TeXStudio
	 %: LaTeX editor
	 \end{tabular}
	 & \begin{tabular}{@{}l@{}}
	 Modification on bib file with vim\\
	 causes its main window hang.
	 \end{tabular}
	 \\
     \hline
	 4-Installer & \begin{tabular}{@{}l@{}} 
	 Installer
	 \end{tabular}
	 & \begin{tabular}{@{}l@{}}
	 Moving cursor out of an authenti-\\
	 cation window causes freeze.
	 \end{tabular}
	 \\
     \hline
	 5-Notes& \begin{tabular}{@{}l@{}} 
	 Notes\\
	 \end{tabular}
	 & \begin{tabular}{@{}l@{}}
	 Launching Notes where stores a\\
	 long note before causes freeze.
	 \end{tabular}
	 \\
     \hline
	 6-TextEdit & \begin{tabular}{@{}l@{}}
	 TextEdit
	 \end{tabular}
	 & \begin{tabular}{@{}l@{}}
	 Copying text over 30M causes\\
	 freeze.
	 \end{tabular}
	 \\
     \hline
	 7-MSWord & \begin{tabular}{@{}l@{}}
	 Microsoft Words
	 \end{tabular}
	 & \begin{tabular}{@{}l@{}}
	 Copying a document over 400 pa-\\
	 ges causes hang.
	 \end{tabular}
	 \\
     \hline
	 8-SlText & Sublime Text
	 & \begin{tabular}{@{}l@{}}
	 Copying in a file over 49000 lines\\
	 causes freeze.
	 \end{tabular}
	\\
    \hline
	 9-TextMate & TextMate 
	 & \begin{tabular}{@{}l@{}}
	 Pating text over 4000 lines causes\\
	 freeze.
	 \end{tabular}
	\\
    \hline
	 10-CotEditor & CotEditor
	 & \begin{tabular}{@{}l@{}}
	 Pasting in file with 4000 lines co-\\
	 ntext causes freeze.\\
	 \end{tabular}
	\\
	 \hline
  \end{tabularx}
 	\parbox{\columnwidth}{\caption{Bug Descriptions. We assign each bug in Column \textbf{Bug ID} to ease discussion}
  	\label{table:bugs-desc}
  	}
\end{table}



%Start a new paragraph
%"TableXXX shows the results.  [Here we should show the big table, and talk about the high-level bits.]"


\begin{table*}[ht]
\footnotesize
\centering
  \begin{tabularx}{\textwidth}{l|cccccc}
 	   & rate of vertices     & rate of edges added &          & length of \xxx      & \# of        & length of auto\\
       & divided by           & by share flag and   & \# of    & baseline/spinning   & user         & baseline/spinning\\
Bug ID & \xxx msg heuristics  & \xxx wait hueristics& data flag & path slicing        & interaction  & path slicing \\
\hline
\hline
 1-SystemPref&  &  & 5 &   &   & \\
 2-SequelPro &  &  & 3 & 5 & 2 & \\
 3-TeXStudio &  &  & 3 & 6 & 3 & \\
 4-Installer &  &  &   &   &   & \\
 5-Notes     &  &  &   &   &   & \\
 6-TextEdit(copy)   &  &  & 3 & 21 & 5  & 21\\
 7-MSWord(copy)     &  &  & 3 & 67 & 12 & 136\\
 8-SlText(copy)     &  &  & 3 & 3  & 1  & \\
 9-TextMate(paste)  &  &  & 3 & 23 & 0  & \\
 10-CotEditor(paste)&  &  & 3 & 4  & 1  & \\
\hline
  \end{tabularx}
  \caption{Graph Comparison}
  \label{table:results}
\end{table*}

%Last paragraph:
%In the remaining of this section, we present the case studies by category in (\S\ref{XXX}). 

\subsection{Long Running}

In this section, we discuss the cases where the \spinningnode is busy on the
CPU. Most of the text editing apps fall into this bug category. We studied
TeXstudio, TextEdit, Microsoft Word, Sublime Text, Text Mate and CotEditor to
reveal their root causes.

\paragraph{TeXStudio}
\subsection{TeXstudio}

TeXstudio is an integrated writing environment for creating LaTeX documents. In
their github we noticed some one reported while he was modifying his bib file,
the TeXstudio hanged for minutes. Although the issue was closed by author for
incomplete information for reproduction, we reproduce it in our machine. With a
relatively large bib file from a LaTex project opened in a tab, once we touch
the bib file through other editors, vim for exmaple, we would have a spinning
beachball on the window.

While the beachball is spinning over the application window, the main thread
is busy with \textit{QDocumentPrivate::indexOF(QDocumentLineHandle const*,
int)}. Slicing the path from the busy node, \xxx tells the busy processing is
invoked by the callback from daemon fseventsd. The advantage of \xxx over other
debugging tools is it helps to narrow down the root cause with the
path. If the user reported the bug with \xxx' report, it should have provided
the author more hints to reproduce the bug successfully.

Another operation triggering spinning beachball in TeXstudio is pasting
a large amount of context in any file. The spinning node is busy with
\textit{QEditor::insertFromMimeData(QMimeData const*)}, which always invokes
\textit{match(QNFAMatchContext*, QChar const*, int, QNFAMatchNotifier)}. Path
slicing by \xxx attributes it to the user's keybord inputs of cmd+v. The
problematic code resides where developers copy data from the pasteboard.


\paragraph{Other Editing Apps}

Select, copy, paste, delete, insert and save are common operations for text
editing. However, these operations on a large context usually trigger spinning
cursors. Depending on their implementations, CotEditor and TextMate successfully
avoid hangs on copy and selection operation. \xxx can helps the developer to
figure out the more efficient way to implement event handlers. We briefly list
the reports from \spinningnode, including the event handler and most costly
functions. We also list corresponding user input event from the path slicing in
Table~\ref{table:texteditapps}.

%The path usually involves several
%daemons, Figure~\ref{fig:diagramXXX} illustrates the spinning case of XXX to
%represent cases in this category.

\paragraph{7-MSWord}

Microsoft Word is a large and complex piece of software. \xxx can analyze the event graph, but
it identifies multiple possible root causes:
the length of path interactively sliced from the \spinningnode
is 67, while the automatic slicing generates a path of 136 vertices.
%We rely on user interactions to help speed up the path slicing.

We compared the path and find that the earliest difference exists in the
predecessor of the third vertex in backward paths.
In the vertex, user can learn from the callstack that Microsoft Word launches a
service NSServiceControllerCopyServiceDictionarie after being woken by another
MSWord thread; this thread then sends a message to \vv{launchd} to register the
new service and waits for a reply message. With the most recent edge heuristics
in automatic slicing, \xxx chose \vv{launchd} as its precedessor, but the user
can more precisely identify that the execution segment is on behalf of the first
thread. We rely on user interaction in this case to find the true root cause,
since \xxx has identified multiple possibilities.

\begin{table}[H]
\footnotesize
\centering
  \begin{tabularx}{\columnwidth}{l|l|l}
                  &                     &\\%&\textbf{root cause}\\%& \textbf{involved daemons}\\
  \textbf{BUG-ID} & \textbf{costly API} &UI\\%&\textbf{UI}\\%& \textbf{involved daemons}\\
  \hline
  \hline
  5-Notes         & \begin{tabular}{@{}l@{}}
  					\vv{1)NSDetectScrollDevices}\\
					\vv{\xspace ThenInvokeOnMainQueue}\\
					\end{tabular}
				  &-
				  \\
  \hline
  6-TexEdit       & \begin{tabular}{@{}l@{}}
  					\vv{1)[NSTextView(NSPasteboard) \_write}\\
					\vv{\xspace RTFDInRanges:toPasteboard:]}\\
					\vv{2)get\_vImage\_converter}\\
  					\vv{3)get\_full\_conversion\_code\_fragment}\\
					\end{tabular}
				  & \vv{key c}
				  \\
  \hline
  7-MSWord        & \begin{tabular}{@{}l@{}}
					\vv{1)-[NSPasteboard setData:}\\
					\vv{\xspace forType:index:usesPboardTypes:]}\\
 					\vv{2)\_CFStringCreateImmutableFunnel3}\\
  					\vv{3)platform\_memmove}\\
					\vv{4)lseek}, \vv{5)fstat64}, \vv{6)fcntl}\\
					\end{tabular}
				  & \vv{key c}
				  \\
  \hline

  8-SlText   & \begin{tabular}{@{}l@{}} 
					\vv{1)px\_copy\_to\_clipboard}\\
  					\vv{2)\_\_CFToUTF8Len}\\
  					\end{tabular}
				  & \vv{key c}
				  \\
  \hline
  9-TextMate      & \begin{tabular}{@{}l@{}}
  					\vv{1)-[OakTextView paste:]}\\
					\vv{2)CFAttributedStringSet}\\
					\vv{3)TASCIIEncoder::Encode}\\
  					\end{tabular}
				  & \vv{key v}
				  \\
  \hline
  10-CotEditor    & \begin{tabular}{@{}l@{}}
  					\vv{1)CFStorageGetValueAtIndex}\\
					\vv{2)-[NSBigMutableString}\\
					\vv{\xspace characterAtIndex:]}\\
  					\end{tabular}
				  & \vv{key v}
				  \\
  \hline
  \end{tabularx}
  \caption{Root cause of spinning cursor in editing Apps}
  \label{table:texteditapps}
\end{table}


\subsection{Long Wait and Repeated Yield}
In this section, we discuss the cases where the \spinningnode is blocking
on wait event or yielding loop, corresponding to \textbf{Long Wait} and
\textbf{Repeated Yield}.

%\paragraph{Spinning cursor}
%The spinning wait cursor is a painful sight for Mac users, signifying that the application is non-responsive.
It usually remains for munites, leaving the users at a loss .                    

Argus shreds light on the design of the spinning wait cursor with the backward path slicing.
We begin the path from the node where the spindump, a hang reporting tool, is launched,  
considering the spindump shares the same triggering condition.
As is shown in Figure \ref{fig:spincursor}, the spindump is launched by the message from WindowServer, after the WindowServe received message from the NSEvent thread of targeted Application.

We further added call stacks for the messages and revealed two shared variables are critical in the desgin, "is\_mainthread\_spinning" and "dispatch\_to\_mainthread".
The NSEvent thread of the targeted App fetches CoreGraphics events from WindowServer, converts and creates NSEvents for the main thread.
If the mainthread is not spinning by checking "is\_mainthread\_spinning", "dispath\_to\_mainthread" is set and a timer is armed.
If the main thread processes the next event before the timer fires, nothing happens and the timer gets reset.
Otherwise, NSEvent thread sends a message to WindowServer from the timer,
and WindowServer notifies the CoreGraphics to draw spinning cursor over the application window.

\paragraph{SequelPro}
Sequel Pro~\cite{SequelPro} is a fast, easy-to-use Mac database management
application for working with MySQL databases. It allows user to connect to
database with a standard way, socket or ssh.

We experienced the non-responsiveness of Sequel Pro while its network connection
got lost and it tried re-connections. The tracing data collected by \xxx
contains 1) a quick network connection during login, and 2) Sequel Pro lost
connection for a while. Although \xxx identified the \spinningnode and
corresponding (baseline) \similarnode with ease, it can hardly get the correct
causal path in the baseline scenario without a user's interaction. The backward
slicing on vertex, which has multiple incomming edges, including one from
kernel thread, makes it hard to rely on heuristics. Our interactively search is
extremely helpful in the step, greatly reducing the noise in the path. Close
examine of the \spinningnode based on the causual path tells that the main
thread is waiting for the kernel thread, which in turn waits for the ssh thread.

%\xxx helps to filter out overconnetions by dividing XXX combined batching node
%with hueristics, and recovers XXX missing edges due to the shared flag. In the
%diagnosis process, we get XXX similar node, XXX times of queries on backward
%slicing , XXX vertices for inspect.

Existing debugging tools like \vv{lldb} and \vv{spindump} could hardly figure
out the root cause, in that both of them diagnose with only call stacks, missing
the dependency across process boundaries.

\paragraph{Installer}
When Installer pops up a window for privileged permission during the
installation of \texttt{jdk-7u80-macosx-x64}, moving the cursor out of the popup
window triggers spinning cursor. Examing the \spinningnode reported by \xxx,
it is easy to figure out the main thread blocks in \textit{[IFRunnerProxy\
requestKeyForRights:askUser:]}. The function sends a synchronous message to
daemon \textit{authd}. The root cause is the authentication of user's previledge
synchronously in the main thread, instead of the handler for moving cursor.
%with \textit{xpc\_connection\_send\_messag\_with\_reply\_sync}.


\paragraph{SystemPreferences}
\subsection{System Preferences spin}
System Preferences is the application in MacOS for users to modify various settings.
Displays in the Panel allows user to rearrange the position of displays, but it does not support the disabling of a montor online.
DisableMonitor is an application used to complete the function, easily disabling/enabling  monitors withou unplug them.
Surprisingly, the operation of DisableMonitor exposed a performance bug in System Preferences.
If we disable an external monitor and arrange them afterward, the window System Preferences freezes for seconds.
We enabled Argus on the background and collect the data by normally arranging the displays and reapeating the spinning sequence.

It is not hard to tell the spinning node in the UI thread with our tool.
However, to find the normal node corresponding to the spinning node is not straightforward as in the previous case.
The execution segment includes two types of events: ``\v{mach\_msg}'' and ``\v{thread\_switch}''.
Both of them are used to waiting for the data available ping.
The semantics of the execution segment is not descriptive enough
to identify the normal nodes in the same operation stage.

In the case we detected the intensive timeout in the node,
our search algorithm identify the corresponding normal node by searching the similarity of their proceding nodes.
We find out the path of the normal path as shown in the Figure \ref{fig:path slicing for system preference}.
Our lightweight callstacks were used to verify the correctness of the findings.

The normal node showed it proceeded to \textit{displayReconfigured} after received the message with id 29675.
The spinning node fell into the ``\v{thread\_switch}'' after receiving the message with the same id,
and end up to send message for the available datagram ping with \textit{CGSSnarfAndDispatchDatagrams} again.
The proceding nodes before them sent message to WindowServer for \textit{activeDisplayNotificationHandler}.

With the result, we initiate the concrete debugging by filling the debugging script with the APIs reported.
We set the method \textit{activeDisplayNotificationHandler} as a beakpoint where the script begins debugging.
\textit{displayReconfigured} and \textit{CGSSnarfAndDispatchDatagrams} are recorded to indicate the end of debugging for the normal case and spinning case respectively.

Our debugging scripts ran within the confined range for both the noraml and the spinning execution.
The logs are generated as shown in Figure \ref{fig:step debug log for system preferences}.
By diff the two logs, it is easy for the user to notice the differen braches in \textit{display\_notify\_proc},
which is resulted from its prameter standing for the datagram type.

We make use of the diassembly tool, and reveals the story in the background.
Datagrams from the WindowServer makes applications to handle notifications.
The datagram causes difference are used to finish display reconfiguration for System Preference.
However, in the spinning case the reconfiguration got initiated but not completed.
The main thread leveraged thread\_switch to wait for the flollowing datagram and resulted in a freeze.
As a conclusion, the handler display\_notify\_proc is not appropriately implemented.

