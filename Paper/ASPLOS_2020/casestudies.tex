\section{Case Studies}\label{sec:casestudy}

In this section, we demonstrate how \xxx helps to diagnose \nbug spinning-cursor
cases in \napps popular applications. Table~\ref{table:bugs-desc} describes
these spinning-cursor cases.

We studied how WindowsServer and NSEvent thread per application coordinate to
determine when a spinning cursor needs to be displayed because it is fundamental
in understanding all spinning cursor cases yet not straightforward. Two data
flags ``\vv{is\_mainthread\_spinning}'' and ``\vv{dispatch\_to\_mainthread}''
are involved. We refer to this case XXXBug-ID.

%TableXXX describes these spinning-cursor cases.
%Should have a table for issue description
%Bug ID |  Application  |  Bug Description
\begin{table}[ht]
\footnotesize
\centering
  \begin{tabularx}{\columnwidth}{l|c|c}
    \hline
    \textbf{Bug ID} & \textbf{Application} & \textbf{Bug description}\\
    \hline
	 0 & WindowServer & display spinning cursor\\
    \hline
	 1 & \begin{tabular}{@{}c@{}} 
	 System Preferences\\
	 %:system settings\\
	 %customization app.
	 \end{tabular}
	 & \begin{tabular}{@{}c@{}}
	 Disabling an online\\
	 external monitor and\\
	 rearranging windows\\
	 cause the freeze.
	 \end{tabular}
	 \\
     \hline
	 2 & \begin{tabular}{@{}c@{}} 
	 Sequel Pro
	 \end{tabular}
	 & \begin{tabular}{@{}c@{}}
	 Lost connection freezes\\
	 the whole application.
	 \end{tabular}
	 \\
     \hline
	 3 & \begin{tabular}{@{}c@{}} 
	 TeXStudio
	 %: LaTeX editor
	 \end{tabular}
	 & \begin{tabular}{@{}c@{}}
	 Modification on bib file\\
	 from vim causes the main\\
	 window spinning.
	 \end{tabular}
	 \\
     \hline
	 4 & \begin{tabular}{@{}c@{}} 
	 Installer
	 \end{tabular}
	 & \begin{tabular}{@{}c@{}}
	 Move cursor out of the\\
	 authentication window \\
	 causes spinning.
	 \end{tabular}
	 \\
     \hline
	 5 & \begin{tabular}{@{}c@{}} 
	 Notes\\
	 %note-taking \\
	 %app from Apple 
	 \end{tabular}
	 & \begin{tabular}{@{}c@{}}
	 launching Notes with \\
	 a saved relatively\\
	 long note.\\
	 \end{tabular}
	 \\
     \hline
	 6 & \begin{tabular}{@{}c@{}}
	 TextEdit
	 %:rich text\\
	 %editing app from Apple
	 \end{tabular}
	 & \begin{tabular}{@{}c@{}}
	 Select, copy, paste\\%delete, insert and save
	 on text over 30M\\
	 cause freeze.
	 \end{tabular}
	 \\
     \hline
	 7 & \begin{tabular}{@{}c@{}}
	 Microsoft Words
	 \end{tabular}
	 & \begin{tabular}{@{}c@{}}
	 copy and paste on\\
	 document over 400\\
	 pages cause hang.
	 \end{tabular}
	 \\
     \hline
	 8 & Sublime Text
	 & \begin{tabular}{@{}c@{}}
	 Copy and paste in file\\
	 over 49000 lines.
	 \end{tabular}
	\\
    \hline
	 9 & TextMate 
	 & \begin{tabular}{@{}c@{}}
	 Paste text over 4000\\
	 lines causes spinning.
	 \end{tabular}
	\\
    \hline
	 10 & CotEditor
	 & \begin{tabular}{@{}c@{}}
	 Paste in file over 4000\\
	 lines causes spinning.
	 \end{tabular}
	\\
	 \hline
  \end{tabularx}
  \caption{Bug Descriptions}
  \label{table:bugs-desc}
\end{table}


%Start a new paragraph
%"TableXXX shows the results.  [Here we should show the big table, and talk about the high-level bits.]"

We compared \xxx with traditional causual tracing methods~\cite{} on edges and
vertices \xxx mitigated. and studied the ratios of over-connection vertices
avoided with message peers hueristically, and the under-connection edgs added by
both data flags and wait event inside a block. User interactions are critical in
the path slicing due to the inherent incorrect in the graph. Our results shows
the user interaction does not overwhelm the debugging process while it makes
the debugging much more efficient. Table~\ref{table:results} demonstrates our
results with the critical path sliced with \xxx, which is much shorter than the
ones generated automatically by choosing the nearest edge hueristically.


\begin{table*}[ht]
\footnotesize
\centering
  \begin{tabularx}{\textwidth}{l|cccccc}
 	   & rate of vertixes   & rate of vertixes  &           & length of           & length of    & \# of\\
       & filted by          & preserved by      & \# of     & baseline/spinning& baseline/spinning&user\\
Bug ID & message hueristics & wait hueristics& data flag & path slicing        & length of  &interaction \\
\hline
\hline
 1-SystemPreferences&&&&&&\\
 2-SequelPro & 0.3\% & 0.21\% & 3 & 5 &  & 2  \\
 3-TeXStudio & 4.8\% & 1.7\% & 3 & 6 &  & 3 \\
 4-Installer &&&&&& \\
 5-Notes &&&&&& \\
 6-TextEdit(copy)&2.9\% &13.6\% & 3 & 21 & & 5\\
 7-MSWord(copy)&6.7\%&1.0\%& 3 &67& & 22\\
 8-SublimeText(copy)&4.0\%& 0.9\%&3 & 3 & & 1\\
 9-TextMate(paste) & 4.2\% & 4.2\% &3 & 23& & 0\\
 10-CotEditor(paste) & 4.8\%& 5.3\%& 3 & 4 & & 1\\
\hline
  \end{tabularx}
  \caption{Graph Comparison}
  \label{table:results}
\end{table*}

%Last paragraph:
In the remaining of this section, we present the case studies by category in (\S\ref{XXX}). 

\subsection{Long Running}

In this section, we discuss the cases where the spinning node is busy on the
CPU. Most of the text editing apps fall into this bug category. We studied
TeXstudio, TextEdit, Microsoft Word, Sublime Text, Text Mate and CotEditor to
reveal their root causes.

\paragraph{TeXStudio}
TeXstudio~\cite{texstudio} is an integrated writing environment for creating
LaTeX documents from github. We noticed a user reported spinning cursor when he
modified his bib file. Alghough the issue was closed by author for incomplete
information to reproduce, we reproduced the bug with a large bib file opened in
a tab. Each time we touched the file through other editor, vim for example, the
application window shows a spinning cursor.

\xxx figures out the \spinningnode is busy processing. Slicing the path from
the vertex, \xxx reaches the daemon ``\vv{fseventd}'' and figures out the busy
processing is invoked by the callback function from the daemon. The advantage of
\xxx over other debugging tools is it helps to narrow down the root cause with
the path. If the user reported the infomation with \xxx, it should have provided
the author more hints to reproduce the bug successfully.

%%Another operation triggering spinning cursor in TeXstudio is pasting
%%a large amount of context in any file. The \spinningnode is busy with
%%\textit{QEditor::insertFromMimeData(QMimeData const*)}, which always invokes
%%\textit{match(QNFAMatchContext*, QChar const*, int, QNFAMatchNotifier)}. Path
%%slicing by \xxx attributes it to the user's keyboard inputs of cmd+v. The
%%problematic code resides where developers copy data from the pasteboard.


\paragraph{Other Editing Apps}

Select, copy, paste, delete, insert and save are common operations for text
editing. However, these operations on a large acontext usually trigger spinning
cursors. Depending on their implementations, Softwares like XXX successfully
avoid hangs on XXX operation. \xxx can helps the developer to figure out the
more efficient way to implement those operations. We briefly list the reports
from the spinning node and its corresponding user input from the path slicing
here. The path usually involves several daemons, Figure~\ref{XXX} illustrates
the spinning case in XXX.

\begin{itemize}
  \item \textbf{Notes}:
	  While launching the Notes, it is busy with
	  \vv{NSDetectScrollDevicesThenInvokeOnMainQueue}.

  \item \textbf{TextEdit}:
	  \xxx reports CPU busy on the storage of characters \vv{[NSBigMutableString\
	  getCharacters:range:]} and \vv{NSFastFillAllGlyphHolesForCharacterRange} for
	  selection.

	  When copy spins, \xxx reports
	  \vv{get\_vImage\_converter} and \vv{get\_full\_conversion\_code\_fragment}
	  occupy the main thread to finish \vv{[NSTextView(NSPasteboard)\
	  \_writeRTFDInRanges:toPasteboard:]}.
	  
	  Paste operation causes the main thread overwhelmed by \vv{\_RTFGetToken}
	  and \vv{platform\_memmove}. Both of them are called by the handler
	  \vv{-[NSTextView(NSPasteboard)\ \_readSelectionFromPasteboard:types:]}

  \item \textbf{Microsoft Words}:
	  Both copy and paste in Microsoft Words are overwhelmed by system calls
	  \vv{lseek}, \vv{fstat64}, \vv{fcntl} on the same file descriptor.
	  In copy operation, \vv{\_\_CFStringCreateImmutableFunnel3} and
	  \vv{platform\_memmove} are invoked for \vv{-[NSPasteboard\
	  \_setData:forType:index:usesPboardTypes:]}, while the paste operation is
	  dominated by \vv{platform\_memmove} and \vv{}.

  \item \textbf{Sublime Text}:
	  Copy in SublimeText is busy encoding characters with \vv{\_\_CFToUTF8Len They}
	  are called from handler \vv{px\_copy\_to\_clipboard}.

	  Paste is dominated by \vv{decode\_utf8}, \vv{convert\_utf8\_to\_utf32}, string
	  \vv{append} and \vv{platform\_memmove} from \vv{px\_copy\_from\_clipboard}.

	  Delete on Sublime Text causes its main thread is busy processing
	  \vv{TokenStorage::substr}.

  \item \textbf{TextMate}:
	  TextMate only spins on the paste operation in our experiments, the main
	  thread is busy with \vv{-[OakTextView paste:]} which iterates through
	  paragraphs, fetches characters and processed with \vv{CFAttributedStringSet} and
	  \vv{TASCIIEncoder::Encode}.

  \item \textbf{CotEditor}:
	  Similar to TextMate, CotEditor does not spinning on copy. In paste and hitting
	  a return key cause the main thread busy on processing \vv{-[NSBigMutableString
	  characterAtIndex:]} and \vv{CFStorageGetValueAtIndex} in the call stacks.
\end{itemize}

\subsection{Repeated Yield and Long Wait}
In this section, we discuss the cases where the spinning node is blocking on
wait event or yielding loop, correponding to \textbf{Long Wait} and \textbf{Repeated Yield}.
\paragraph{SequelPro}
Sequel Pro~\cite{SequelPro} is a fast, easy-to-use Mac database management
application for working with MySQL databases. It allows user to connect to
database with a standard way, socket or ssh.

We experienced the non-responsiveness of Sequel Pro while its network connection
got lost and it tried re-connections. The tracing data collected by \xxx
contains 1) a quick network connection during login, and 2) Sequel Pro lost
connection for a while. Although \xxx identified the \spinningnode and
corresponding (baseline) \similarnode with ease, it cannot get the correct
causal path in the baseline scenario without user interaction. The backward
slicing on vertex has multiple incoming edges, including one from a
kernel thread, which means that operations are likely to be batched together and inseparable by heuristics. Our interactively search is
extremely helpful in the step, greatly reducing the noise in the path. Close
examination of the \spinningnode based on the causual path tells us that the main
thread is waiting for the kernel thread, which in turn waits for the ssh thread.

%\xxx helps to filter out overconnetions by dividing XXX combined batching node
%with hueristics, and recovers XXX missing edges due to the shared flag. In the
%diagnosis process, we get XXX similar node, XXX times of queries on backward
%slicing , XXX vertices for inspect.

Existing debugging tools like \vv{lldb} and \vv{spindump} could hardly figure
out the root cause, in that both of them diagnose with only call stacks, missing
the dependency across process boundaries.

\paragraph{Installer}
Installer is an application included in macOS that extracts and installs
files out of \vv{.pkg} packages. When \vv{Installer} pops up a window for
privileged permission during the installation of \vv{jdk-7u80-macosx-x64},
moving the cursor out of the popup window triggers spinning cursor.

As we put in the password directly before the round of triggering the
spinning cusor, \xxx successfully records the baseline scenario. Examing
the \spinningnode and its \similarnode, \xxx figures out the daemon
\vv{authd} blocks on semaphore. The blocking synchronous authentication
of user's previledge in the main thread (user stops input) is the root
casue, instead of the cursor moving handler. The result was verified with
\v{lldb}. The main thread sends a synchromous message via \vv{[IFRunnerProxy\
requestKeyForRights:askUser:]} to authd.


We also discovered a communication pattern in \vv{Installer} underpinning the
crucial of interactive debugging. It involves four vertices in four threads,
vertex $Vertex_{main}$ in the main thread, and $Vertex_1$ to $Vertex_3$ in
three worker threads. First, the main thread wakes up three worker threads.
Then one worker thread is scheduled to run. At its end, another worker thread,
which waits on mutex lock, is woken in $Vertex_2$, which in turn wakes up the
next worker thread in $Vertex_3$. While \xxx is slicing backward, $Vertex_3$
has two incoming edges: one is from $Vertex_{main}$, and the other one is from
$Vertex_2$. Since users can peek the edges before making decision, they are
likely to figure out that the three worker threads contend with mutex lock, and
all of them are the heirs of $Vertex_{main}$.

\paragraph{SystemPreferences}
\subsubsection{System Preferences}

System Preferences provides a central location in macOS to customize system
settings. The ``Display'' pane allows users to configure additional monitors,
mirroring or extending their work space. However, the software does not support
disabling monitors that are online (the user must physically unplug monitors).
There is a tool called DisableMonitor~\cite{disablemonitor}, distributed via
GitHub, which addresses this functionality. Surprisingly, we find performance
bug in System Preferences when we disable an external monitor, and rearrange
the windows in the Display panel afterward. The System Preferences windows
freezes for seconds in this situation.

To diagnose this issue, we run the System Preferences app with \xxx. We
rearrange the displays with two active monitors, and repeat the process with
one of them deactivated. \xxx collects 132MB data, and constructs dependency
graph system-wide in the period, which contains 428,785 vertexes and 320,554
edges.

To diagnose the root cause, \xxx first finds out when the NSEvent thread in
System Preferences notifies WindowServer to draw the spinning cursor and marks
it as \textit{t}. The node in the main UI thread causing the non-responsiveness
must overlap the time interval \textit{(t - 2s, t)}. It is not hard for
\xxx to tell the spinning node in the main UI thread, either busy processing or
blocking.

The spinning node in the main UI thread is dominated by \textit{mach\_msg} and
\textit{thread\_switch}, both of which are meant to wait for available data
ping from WindowServer. Noticing the timeouts of \text{thread\_switch}, \xxx
classifies this case into the third category (\S\ref{subsec:debug}) and heads
to find a comparable normal node in the same thread. As its semantics are not
descriptive enough to identify its comparable node, \xxx extends the comparison
with its proceeding nodes.

In the normal case, System Preferences gets rid of the thread\_switch
quickly after receiving messages from WindowServer. It proceeds
to \textit{displayReconfigured}. On the other hand, the spinning
node ends up sending message for the available datagram ping with
\textit{CGSSnarfAndDispatchDatagrams}. By checking the lightweight call stacks,
\xxx figures out the messages from WindowServer in the previous nodes are
responses to data available pings from \textit{activeDisplayNotificationHandler}.

Given the information, we launched the interactive debugging by 
feeding the debugging script with those APIs mentioned above. We set the method
\textit{activeDisplayNotificationHandler} as a breakpoint where the script begins
debugging. \textit{displayReconfigured} and \textit{CGSSnarfAndDispatchDatagrams}
are used to indicate the end of debugging for the normal case and spinning case
respectively.

By diffing the two logs, we notice the different
branches in \textit{display\_notify\_proc} called by
\textit{activeDisplayNotificationHandler}. The handler depends on received
datagram and two shared variables ``\vv{\_gCGWillReconfigureSeen}'' and
``\vv{\_gCGDidReconfigureSeen}'' to finish a display configuration. In
both case the first variable is set to indicate the begin of display
configuration. In normal case, it receives a datagram which set the variable
``\vv{\_gCGDidReconfigureSeen}'' in \textit{display\_notify\_proc} and
finish display reconfiguration, while in the spinning case such datagram
is never received. Instead, an alternative datagram just drive it through
\textit{display\_notify\_proc} without setting any variable, which causes the
repeating \textit{thread\_switch} in the handler.

[In conclusion, the bug is... would have to be fixed by... we provide a binary fix...]

