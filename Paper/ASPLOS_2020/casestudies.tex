\section{Case Studies}\label{sec:casestudy}

In this section, we demonstrate how \xxx helps to diagnose \nbug spinning-cursor
cases in \napps popular applications. Table~\ref{table:bugs-desc} describes
these spinning-cursor cases, which are detailed in the sections that follow.
We compared \xxx with traditional causual tracing methods~\cite{XXXX} on edges
and vertices \xxx mitigated, and studied the ratios of vertices divided by
with message peers heuristically to reduce over-connection, and edges added
by both data flags and wait event inside a block to avoid under-connection.
User interactions are critical in the path slicing due to the inherent
inaccurate in the graph. Our results shows the user interaction does not
overwhelm the debugging process while makes the debugging much more efficient.
Table~\ref{table:results} demonstrates our results with the length of path
sliced with \xxx, which is much shorter than the one generated automatically by
heuristically choosing the most recent edges if multiple predecessors.

The spinning wait cursor is a painful sight for Mac users, signifying that the application is non-responsive.
It usually remains for munites, leaving the users at a loss .                    

Argus shreds light on the design of the spinning wait cursor with the backward path slicing.
We begin the path from the node where the spindump, a hang reporting tool, is launched,  
considering the spindump shares the same triggering condition.
%As is shown in Figure \ref{fig:spincursor},
The spindump is launched by the message from WindowServer, after the WindowServe received message from the NSEvent thread of targeted Application.

We further added call stacks for the messages and revealed two shared variables, ``\v{is\_mainthread\_spinning}'' and ``\v{dispatch\_to\_mainthread}'', are critical in the desgin,
The NSEvent thread of the targeted App fetches CoreGraphics events from WindowServer, converts and creates NSEvents for the main thread.
If the mainthread is not spinning with the cleared ``\v{is\_mainthread\_spinning}'', ``\v{dispath\_to\_mainthread}'' is set and a timer is armed.
If the main thread processes the next event before the timer fires, nothing happens and the timer gets re-armed.
Otherwise, NSEvent thread sends a message to WindowServer from the timer,
and WindowServer notifies the CoreGraphics to draw spinning cursor over the application window.



%TableXXX describes these spinning-cursor cases.
%Should have a table for issue description
%Bug ID |  Application  |  Bug Description
\begin{table}[ht]
\footnotesize
\centering
  \begin{tabularx}{\columnwidth}{l|cl}
    \hline
    \textbf{Bug ID} & \textbf{Application} & \textbf{Bug description}\\
    \hline
	\hline
	 0-Chormium & Chormium & \begin{tabular}{@{}l@{}}
	 Typing non-english in\\
	 search box on yahoo\\
	 webpage quickly triggers\\
	 spinning cursor\\
	 \end{tabular}
	 \\
     \hline
	 1-SystemPref & \begin{tabular}{@{}l@{}} 
	 System Preferences\\
	 \end{tabular}
	 & \begin{tabular}{@{}l@{}}
	 Disabling an online\\
	 external monitor and\\
	 rearranging windows\\
	 cause the freeze.
	 \end{tabular}
	 \\
     \hline
	 2-SequelPro& \begin{tabular}{@{}l@{}} 
	 Sequel Pro
	 \end{tabular}
	 & \begin{tabular}{@{}l@{}}
	 Lost connection freezes\\
	 the whole application.
	 \end{tabular}
	 \\
     \hline
	 3-TexStudio & \begin{tabular}{@{}l@{}} 
	 TeXStudio
	 %: LaTeX editor
	 \end{tabular}
	 & \begin{tabular}{@{}l@{}}
	 Modification on bib file\\
	 from vim causes the main\\
	 window spinning.
	 \end{tabular}
	 \\
     \hline
	 4-Installer & \begin{tabular}{@{}l@{}} 
	 Installer
	 \end{tabular}
	 & \begin{tabular}{@{}l@{}}
	 Move cursor out of the\\
	 authentication window \\
	 causes spinning.
	 \end{tabular}
	 \\
     \hline
	 5-Notes& \begin{tabular}{@{}l@{}} 
	 Notes\\
	 \end{tabular}
	 & \begin{tabular}{@{}l@{}}
	 launching Notes with \\
	 a saved relatively\\
	 long note.\\
	 \end{tabular}
	 \\
     \hline
	 6-TextEdit & \begin{tabular}{@{}l@{}}
	 TextEdit
	 \end{tabular}
	 & \begin{tabular}{@{}l@{}}
	 copy on text over 30M\\
	 cause freeze.
	 \end{tabular}
	 \\
     \hline
	 7-MSWord & \begin{tabular}{@{}l@{}}
	 Microsoft Words
	 \end{tabular}
	 & \begin{tabular}{@{}l@{}}
	 copy a whole document\\
	 over 400\\
	 pages cause hang.
	 \end{tabular}
	 \\
     \hline
	 8-SlText & Sublime Text
	 & \begin{tabular}{@{}l@{}}
	 Copy and paste in file\\
	 over 49000 lines.
	 \end{tabular}
	\\
    \hline
	 9-TextMate & TextMate 
	 & \begin{tabular}{@{}l@{}}
	 Paste text over 4000\\
	 lines causes spinning.
	 \end{tabular}
	\\
    \hline
	 10-CotEditor & CotEditor
	 & \begin{tabular}{@{}l@{}}
	 Paste in file over 4000\\
	 lines causes spinning.
	 \end{tabular}
	\\
	 \hline
  \end{tabularx}
  \caption{Bug Descriptions. We assign each bug in Column \textbf{Bug ID} to ease discussion}
  \label{table:bugs-desc}
\end{table}


%Bug 0-Apple is ...... The spinning cursor is created when the main thread
%stops responding to events for two seconds. Every application has an
%NSEvent thread, which coordinates with WindowsServer to display a spinning
%cursor when necessary. Two data flags ``\vv{is\_mainthread\_spinning}'' and
%``\vv{dispatch\_to\_mainthread}'' are involved.
%Start a new paragraph "TableXXX shows the results. [Here we should show the big
%table, and talk about the high-level bits.]"


\begin{table*}[ht]
\footnotesize
\centering
  \begin{tabularx}{\textwidth}{l|cccccc}
 	   & rate of connections & rate of connections &          & length of \xxx      & \# of        & length of auto\\
       & filted by        & added by share flag and & \# of    & baseline/spinning   & user         & baseline/spinning\\
Bug ID & \xxx msg heuristics  & \xxx wait hueristics  & data flag & path slicing        & interaction  & path slicing \\
\hline
\hline
SystemPreferences &  &  & 5 & &  & \\
SequelPro & 0.0049 & 0.0035 & 3 & 5 & 2 & 264\\
TeXStudio & 0.0243 & 0.0058 & 3 & 6 & 3 & \\
installer & 0.0439 & 0.0283 & 3 & 6 & 2 & 36\\
notes & 0.0297 & 0.1153 & 3 & &  & 42\\
TextEdit(copy) & 0.0797 & 0.0072 & 3 & 21 & 3 & 21\\
MSWord(copy) & 0.0672 & 0.0104 & 3 & 67 & 22 & 136\\
SublimeText(copy) & 0.0407 & 0.0092 & 3 & 3 & 1 & \\
TextMate(paste) & 0.0215 & 0.0218 & 3 & 23 & 0 & \\
CotEditor & 0.0481 & 0.0532 & 3 & 4 & 1 & \\

\hline
  \end{tabularx}
  \caption{Graph Comparison}
  \label{table:results}
\end{table*}

%Last paragraph:
%In the remaining of this section, we present the case studies by category in (\S\ref{XXX}). 
\subsubsection{Long Running}

In this section, we discuss the cases where the \spinningnode is busy on the
CPU. Most text editing apps fall into this bug category. We studied bugs on
TeXstudio, TextEdit, Microsoft Word, SublimeText, TextMate and CotEditor,
to reveal the root causes.

\para{5-TeXStudio}

\vv{TeXstudio}~\cite{TeXStudio} is an integrated writing environment for
creating LaTeX documents. We noticed a user reported spinning cursor on the
modification of bibliography (bib) file. Although the issue was closed by the
developer for insufficient information, we reproduced it with our bib file
around 500 items in a \vv{TeXStudio} tab. When we touch the file in
other editors like vim, a spinning cursor appears in \vv{TeXStudio}'s window.

\xxx recognizes the \spinningnode belongs to the category of LongRunning. The
causal path sliced from the \spinningnode by \xxx reveals the long-running
function is a callback from daemon \vv{fseventd}, and the long processing segment
is busy to CalculateGrowingBlockSize, even without modifications to the file.
The advantage of \xxx over other debugging tools is it narrows down the root
cause with the inter-processes execution path.

\para{6-TextEdit}

\vv{TextEdit} is a simple word processing and text editing tool from Apple,
which often hangs on editing large files.  When \xxx is used to diagnose
this issue, the heuristics of choosing the most recent edge is powerful enough 
to get the causal path.

The event graph reveals a communication pattern where a kernel thread is woken
from I/O by another kernel thread; the woken kernel thread processed
a timer callback function armed by \vv{TextEdit} and finally woke a \vv{TextEdit}
thread. In the pattern, the kernel thread has two incoming edges. One is from
another kernel thread's IO completion, and the other is from \vv{TextEdit}'s timer
creation. It not hard to reveal the high level semantics. \vv{TextEdit} arms a timer
for IO work, and the kernel thread gets the notification for the completion of IO
and processes the timer callback. 
Although the most recent incoming edge to the vertex reflects the purpose of
the execution segment in this case, it is not general enough to for all vertex
patterns.

\para{7-MSWord}

\vv{Microsoft Word} is a large and complex piece of software. \xxx can analyze
the event graph, but it identifies multiple possible root causes: the length of
path interactively sliced from the \spinningnode is 67, while the slicing with
heuristics of choosing the most recent edge generates a path of 136 vertices.

We compared the paths and find they diverge from the third vertex backward
from \spinningnode. In the vertex, a \vv{Microsoft Word} thread is
woken by another \vv{Microsoft Word} thread, and launches a service
\vv{NSServiceControllerCopyServiceDictionarie}. The woken thread sends a message
to \vv{launchd} to register the new service and waits for a reply message. In
this case, a user can accurately identify that the execution segments is on
behalf of \vv{Microsoft Word}, instead of \vv{launchd}. where the most recent
incoming edge comes from. Heuristics is likely to identify all possibilities
without priority. We rely on user interaction in this case to find the true root
cause. We identified \vv{Microsoft Word} spends most time on file accesses and
the copy command triggers those accesses.


\para{Other Editing Apps}

Select, copy, paste, delete, insert and save are common operations for text
editing. However, these operations on a large context usually trigger spinning
cursors. In Table~\ref{table:texteditapps}, we list the root causes reported by
\xxx, including the most costly functions in the event handler, and the user
input event (derived from path slicing).

\begin{table}[tb]
\vspace{-0.2cm}
\footnotesize
\centering
  \begin{tabularx}{\columnwidth}{l|l|l}
  \hline
                  &                     &\\
  \textbf{BUG-ID} & \textbf{costly API} &UI\\
  \hline
  \hline
  8-Notes         & \begin{tabular}{@{}l@{}}
  					\vv{1)NSDetectScrollDevicesThe}\\
					\vv{\xspace -nInvokeOnMainQueue}\\
					\end{tabular}
   		          & \begin{tabular}{@{}l@{}}
				  	\vv{system defined event}
					\end{tabular}
				  \\
  \hline
  9-SlText   & \begin{tabular}{@{}l@{}} 
					\vv{1)px\_copy\_to\_clipboard}\\
  					\vv{2)\_\_CFToUTF8Len}\\
  					\end{tabular}
				  & \vv{key c}
				  \\
  \hline
  10-TextMate      & \begin{tabular}{@{}l@{}}
  					\vv{1)-[OakTextView paste:]}\\
					\vv{2)CFAttributedStringSet}\\
					\vv{3)TASCIIEncoder::Encode}\\
  					\end{tabular}
				  & \vv{key v}
				  \\
  \hline
  11-CotEditor    & \begin{tabular}{@{}l@{}}
  					\vv{1)CFStorageGetValueAtIndex}\\
					\vv{2)-[NSBigMutableString}\\
					\vv{\xspace characterAtIndex:]}\\
  					\end{tabular}
   		          & \begin{tabular}{@{}l@{}}
				  	\vv{key Return}
  					\end{tabular}

				  \\
  \hline
  \end{tabularx}
  \caption{Root cause of spinning cursor in editing Apps.}
  \label{table:texteditapps}
  \vspace{-0.2cm}
\end{table}


\subsubsection{Long Wait and Repeated Yield}

In this section, we discuss the cases where the \spinningnode is blocking on
wait event or yielding loop, corresponding to LongWait and RepeatedYield.
These cases are mostly can be verified by themselve, as the root causes are
manifested by comparing to normal scenarios.

\paragraph{2-SystemPref}

\vv{System Preferences} provides a central location in macOS to customize system
settings, \eg additional monitors configuration.
\vv{DisableMonitor}~\cite{disablemonitor} provides more functionality, \eg
enable/disable monitors online. We caught the spinning cursor while we disable
an external monitor and rearrange windows in \vv{Display} panel.

The log collected by \xxx contains 2 cases: 1) a baseline scenario where the
displays are rearranged with the enabled external monitor, and 2) a spinning
scenario as we described above. The \spinningnode in the main thread is dominated
by system calls, \vv{mach\_msg} and \vv{thread\_switch}, which falls into the
category of Repeated Yield. We discovered two missing data flags,
``\vv{\_gCGWillReconfigureSeen}'' and ``\vv{\_gCGDidReconfigureSeen}'', which
signify the configuration status and break the thread-yield loop. 
\xxx's diagnosis result reveals that the main thread in \vv{System Preferences} sets
them after receiving specific datagrams from \vv{WindowServer}. Conversely, the
setting of ``\vv{\_gCGDidReconfigureSeen}'' is missing in the spinning case.
The main thread thus repeatedly sent messages to \vv{WindowServer} for datagram.

In conclusion, we discovered that the bug is inherent in the design of the
\vv{CoreGraphics} framework, and would have to be fixed by Apple. We verified this
diagnosis by creating a dynamic binary patch to fix the deadlock. The
patched library makes \vv{DisableMonitor} work correctly, while preserving correct
behavior for other applications.

\paragraph{3-SequelPro}

\vv{Sequel Pro}~\cite{SequelPro} is a fast, easy-to-use Mac database management
application for \vv{MySQL}. It allows user to connect to database with socket or ssh.
We experienced the non-responsiveness of Sequel Pro when it lost network
connection and tried reconnections.

The tracing log contains two cases: 1) a quick network connection during login,
and 2) Sequel Pro lost connection for a while. Although \xxx identified the
\spinningnode and \similarnode with ease, the backward slicing from
\similarnode encountered multiple incoming edges, including one from a kernel
thread, where batching processing from different applications happens.
Interaction is helpful and reduces the noise in the path.  Diagnosis on the
\spinningnode, comparing with the normal causal path, tells that the main thread
is blocking on a kernel thread, which in turn waits for a ssh thread. Thus the
root cause is the blocking for network IO.

\paragraph{4-Installer}

\vv{Installer}~\cite{Installer} is an application that extracts
and installs files out of \vv{.pkg} packages in macOS. When \vv{Installer} pops up a
window for privileged permission during the installation of
\vv{jdk-7u80-macosx-x64}, moving the cursor out of the popup window triggers a
spinning cursor.

\xxx successfully records the baseline scenario with our operations. We first
type in password in the pop-up window and then click the back button to
reproduce the spinning case described above.  Examining the \spinningnode and
its \similarnode, \xxx reveals the daemon \vv{authd} blocks on semaphore while
the main thread is waiting for \vv{authd}. Further checking on \vv{authd}, \xxx
finds out \vv{SecurityAgent} processes user input and wakes up \vv{authd} in
baseline scenario. In conclusion, moving the mouse out of the authentication
window causes the missing edge from \vv{SecurityAgent} to \vv{authd}, which in
turn blocks \vv{Installer}.

We also discovered a communication pattern in \vv{Installer} underpinning the
crucial of interactive debugging. It involves four vertices in four threads,
vertex $Vertex_{main}$ in the main thread, and $Vertex_1$ to $Vertex_3$ in
three worker threads. First, the main thread wakes up three worker threads.
Then one worker thread is scheduled to run. At its end, another worker thread,
which waits on mutex lock, is woken in $Vertex_2$, which in turn wakes up the
next worker thread in $Vertex_3$. While \xxx is slicing backward, $Vertex_3$
has two incoming edges: one is from $Vertex_{main}$, and the other one is from
$Vertex_2$. Since users can peek the edges before making decision, they are
likely to figure out that the three worker threads contend with mutex lock, and
all of them are successors of $Vertex_{main}$.


