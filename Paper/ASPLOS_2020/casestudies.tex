\section{Case Studies}\label{sec:casestudy}

In this section, we demonstrate how \xxx helps to diagnose \nbug spinning-cursor
cases in popular applications. Table~\ref{table:bugs-desc} describes these
spinning-cursor cases, which are detailed in sections that follow. We compared
\xxx with traditional causal tracing methods on edges and vertices mitigated.
The portions of over-connection filtered out and incognito under-connections
disclosed are studied. Even with our technique, the filtered graph remains
too inaccurate for automatic causal tracing. Fortunately, user interactions
compensate the inaccuracy in causal path slicing, while not overwhelming in the
diagnosis process.

Our results in Table~\ref{table:results} show the portion of connections
filtered with heuristics is limited. User interaction is still required for
diagnosis, but in most cases, up to 3 user queries suffice to find root cause
path accurately. Although a complex application like 8-MSWord and 1-Chromium
requires 22 and 13 queries respectively, many of them results from repeated
patterns. They can be easily identified by users, as discussed below. Our
interactively path slicing exclude much noise, and are shorter and easier to
inspect.

Overall, in the case of simple text editing applications, \xxx can identify the
UI event that causes a spinning cursor by merely relying on a few heuristics.
However, these heuristics may make the wrong decision in complicated cases, and
misidentify the relationships between intra/inter-thread events. It is unlikely
that there exists a single graph search method that works in all cases, e.g.
when given the choice between multiple incoming edges, the most recent match is
sometimes correct, but sometimes not. This is why our system relies on expert
knowledge of users to reconstruct a developer's intent and accurately diagnose
performance issues.

\paragraph{How \xxx Detects Spinning Cursors}

When the spinning cursor shows up, a hang reporting tool, \spindump usually
kicks in automatically to sample callstacks for debugging. To figure out the
\spinningnode in the main thread, we turn to the event graph, and slice path
backward from the launch of \spindump.
The path shows that \spindump is launched after receiving a message from
\vv{WindowServer}, which received a message from the \vv{NSEvent} thread of
the freezing app. The call stack attached to the messages further reveals the
\vv{NSEvent} thread fetches \vv{CoreGraphics} events from
\vv{WindowServer}, converts and creates NSApp events for the main 
thread. If the main thread fails to process an NSApp event before the timer
fires, \vv{NSEvent} thread sends a message to \vv{WindowServer} via the API
``\vv{CGSConnectionSetSpinning}'' from the timer handler, and \vv{WindowServer}
notifies the \vv{CoreGraphics} to draw a spinning cursor over the application
window. To avoid sending more events to an already-spinning main thread, 
the NSApp thread uses two data flags to track the main thread status, 
\vv{is\_mainthread\_spinning} and \vv{dispatch\_to\_mainthread}.
These flags are also modified by the main thread to update its status.
\xxx leverages this API to identify the \spinningnode in the main thread.



%TableXXX describes these spinning-cursor cases.
%Should have a table for issue description
%Bug ID |  Application  |  Bug Description
\begin{table}[t]
\footnotesize
\centering
  \begin{tabularx}{\columnwidth}{l|cl}
    \hline
    \textbf{Bug ID} & \textbf{Application} & \textbf{Bug description}\\
    \hline
	\hline
	 0-Chormium & Chormium & \begin{tabular}{@{}l@{}}
	 Typing non-english in search box\\
	 causes webpage freeze.
	 \end{tabular}
	 \\
     \hline
	 1-SystemPref & \begin{tabular}{@{}l@{}} 
	 System Preferences\\
	 \end{tabular}
	 & \begin{tabular}{@{}l@{}}
	 Disabling an online external mo-\\
	 nitor and rearranging windows\\
	 causes System Preferences freeze.\\
	 \end{tabular}
	 \\
     \hline
	 2-SequelPro& \begin{tabular}{@{}l@{}} 
	 Sequel Pro
	 \end{tabular}
	 & \begin{tabular}{@{}l@{}}
	 Lost connection freezes the APP.
	 \end{tabular}
	 \\
     \hline
	 3-TexStudio & \begin{tabular}{@{}l@{}} 
	 TeXStudio
	 %: LaTeX editor
	 \end{tabular}
	 & \begin{tabular}{@{}l@{}}
	 Modification on bib file with vim\\
	 causes its main window hang.
	 \end{tabular}
	 \\
     \hline
	 4-Installer & \begin{tabular}{@{}l@{}} 
	 Installer
	 \end{tabular}
	 & \begin{tabular}{@{}l@{}}
	 Moving cursor out of an authenti-\\
	 cation window causes freeze.
	 \end{tabular}
	 \\
     \hline
	 5-Notes& \begin{tabular}{@{}l@{}} 
	 Notes\\
	 \end{tabular}
	 & \begin{tabular}{@{}l@{}}
	 Launching Notes where stores a\\
	 long note before causes freeze.
	 \end{tabular}
	 \\
     \hline
	 6-TextEdit & \begin{tabular}{@{}l@{}}
	 TextEdit
	 \end{tabular}
	 & \begin{tabular}{@{}l@{}}
	 Copying text over 30M causes\\
	 freeze.
	 \end{tabular}
	 \\
     \hline
	 7-MSWord & \begin{tabular}{@{}l@{}}
	 Microsoft Words
	 \end{tabular}
	 & \begin{tabular}{@{}l@{}}
	 Copying a document over 400 pa-\\
	 ges causes hang.
	 \end{tabular}
	 \\
     \hline
	 8-SlText & Sublime Text
	 & \begin{tabular}{@{}l@{}}
	 Copying in a file over 49000 lines\\
	 causes freeze.
	 \end{tabular}
	\\
    \hline
	 9-TextMate & TextMate 
	 & \begin{tabular}{@{}l@{}}
	 Pating text over 4000 lines causes\\
	 freeze.
	 \end{tabular}
	\\
    \hline
	 10-CotEditor & CotEditor
	 & \begin{tabular}{@{}l@{}}
	 Pasting in file with 4000 lines co-\\
	 ntext causes freeze.\\
	 \end{tabular}
	\\
	 \hline
  \end{tabularx}
 	\parbox{\columnwidth}{\caption{Bug Descriptions. We assign each bug in Column \textbf{Bug ID} to ease discussion}
  	\label{table:bugs-desc}
  	}
\end{table}


%Bug 0-Apple is ...... The spinning cursor is created when the main thread
%stops responding to events for two seconds. Every application has an
%NSEvent thread, which coordinates with WindowsServer to display a spinning
%cursor when necessary. Two data flags ``\vv{is\_mainthread\_spinning}'' and
%``\vv{dispatch\_to\_mainthread}'' are involved.
%Start a new paragraph "TableXXX shows the results. [Here we should show the big
%table, and talk about the high-level bits.]"

\begin{table*}[ht]
\footnotesize
\centering
  \begin{tabularx}{\textwidth}{l|ccccccc}
 	   & \% of & \% of & \# of& \# of & \multicolumn{2}{c}{size of baseline/spinning path}& auto slicing\\
       & connections & connections added  & user provided & user  & \multicolumn{2}{c}{with}  & \/ \\
Bug ID & filtered out & by heuristics & data flag & interactions & interactive slicing & automatic slicing &  interactive slicing\\
\hline
\hline
1-Chromium & 0.02 & 0.02 & 0 & 13 & 32 & 303 & 9.47\\
2-SystemPref & 0.56 & 2.48 & 2 & 1 & 2 & 30 & 15.00\\
3-SequelPro & 0.49 & 0.35 & 0 & 2 & 5 & 264 & 52.80\\
4-Installer & 4.39 & 2.83 & 0 & 2 & 6 & 36  & 6.00\\
5-TeXStudio & 2.43 & 0.58 & 0 & 3 & 6 & 44  & 7.33\\
6-Notes & 2.97 & 11.53 & 0 & 2 & 10 & 42 & 4.20\\
7-TextEdit & 7.97 & 0.72 & 0 & 3 & 21 & 21 & 1.00\\
8-MSWord & 6.72 & 1.04 & 0 & 22 & 67 & 136 & 2.03\\
9-SlText & 4.07 & 0.92 & 0 & 1 & 3 & 3 & 1.00\\
10-TextMate & 2.15 & 2.18 & 0 & 0 & 3 & 3 & 1.00\\
11-CotEditor & 4.81 & 5.32 & 0 & 1 & 4 & 6 & 1.50\\

\hline
  \end{tabularx}

  \parbox{\textwidth}
  {\caption{Graph Comparison} 
	  %{1.\xxx filters out connections by dividing a batch processing vertex into
		%  vertices, and adds connections as edges for data flag or heuristic. The portion
		%  changed is small. 2.\xxx only requires a few user interactions, but it is
		%  critical to reduce the path, so as to reduce user's inspecting efforts in
		%  diagnosis. 3.The last column shows the ratio of path size with user interactions
    %over the automatic slicing.}
    {The first and second column presents the portion of connection adjusted by \xxx compared
     to traditional cuasal tracing. Column 3 is the number of data flag added for diagnosis
     inaddition to the default flag \xxx tracks. The forth and fifth column illustrates
     the vetices in the paths, which are sliced backward with and without user interaction.
     The last column shows that without user interaction, the path sliced by \xxx, even with
     our filtering heuristics, still includes too much vertices for inspection.
    }
  \label{table:results}
  }

\end{table*}
%Last paragraph:
%In the remaining of this section, we present the case studies by category in (\S\ref{XXX}). 
\subsection{Long Wait and Repeated Yield}

In this section, we discuss the cases where the \spinningnode is blocking on
wait event or yielding loop, corresponding to Long Wait and Repeated Yield.

\paragraph{2-SystemPref}

System Preferences provides a central location in macOS to customize system
settings, including configuring additional monitors. A tool called
\vv{DisableMonitor}~\cite{disablemonitor} provides full functionality including
the ability to enable/disable monitors online. We blocked on the spinning
cursor while disabling an external monitor and rearranging windows in
\vv{Display} panel.

The log collected with \xxx contains 1) a baseline scenario where the displays
are rearranged with the enabled external monitor, and 2) a spinning scenario in
which we disable the external monitor with \vv{DisableMonitor} and rearrange
the displays. The \spinningnode in the main thread is dominated by system
calls, \vv{mach\_msg} and \vv{thread\_switch}, which falls into the category of
Repeated Yield. We discovered two missing data flags with \vv{lldb},
``\vv{\_gCGWillReconfigureSeen}'' and ``\vv{\_gCGDidReconfigureSeen}'', which
signify the configuration status and break the thread-yield loop. \xxx learns
from the baseline scenario that the main thread is responsible to set both of
them after receiving specific datagrams from WindowServer. Conversely, the
setting of ``\vv{\_gCGDidReconfigureSeen}'' is missing in the spinning case,
where the main thread yields repeatedly to send messages to WindowServer for
such datagram.

In conclusion, we discovered that the bug is inherent in the design of the
CoreGraphics library, and would have to be fixed by Apple. We verified this
diagnosis by creating a dynamic binary patch with lldb to fix the deadlock. The
patched library makes DisableMonitor work correctly, while preserving correct
behavior for other applications.

\paragraph{3-SequelPro}

Sequel Pro~\cite{SequelPro} is a fast, easy-to-use Mac database management
application for working with MySQL. It allows user to connect to database with
a standard way, socket or ssh.

We experienced the non-responsiveness of Sequel Pro when it lost network
connection and tried reconnections. The tracing data collected by
\xxx contains 1) a quick network connection during login, and 2) Sequel Pro
lost connection for a while. Although \xxx identified the \spinningnode and
corresponding (baseline) \similarnode with ease, it cannot get the correct
causal path in the baseline scenario without user interaction. The backward
slicing on vertex has multiple incoming edges, including one from a kernel
thread, which means that operations are likely to be batched together and
inseparable by heuristics. Our interactively search is extremely helpful in
this step, greatly reducing the noise in the path. Close examination of the
\spinningnode based on the causal path tells us that the main thread is waiting
for the kernel thread, which in turn waits for the ssh thread. Existing
debugging tools like \vv{lldb} and \vv{spindump} cannot determine the root
cause, because both of them diagnose with only call stacks, missing the
dependency across processes.

\paragraph{4-Installer}

Installer~\cite{Installer} is an application included in macOS that extracts
and installs files out of \vv{.pkg} packages. When \vv{Installer} pops up a
window for privileged permission during the installation of
\vv{jdk-7u80-macosx-x64}, moving the cursor out of the popup window triggers a
spinning cursor.

As we put in the password before the round of triggering the spinning cursor,
\xxx successfully records the baseline scenario. Examining the \spinningnode
and its \similarnode, \xxx figures out the daemon \vv{authd} blocks on
semaphore while the main thread is waiting for \vv{authd}. Further checking on
\vv{authd}, \xxx reveals it is the \vv{SecurityAgent} that processes user input
and wakes up \vv{authd} in baseline scenario. In conclusion, moving the mouse
out of the authentication window causes the missing edge from
\vv{SecurityAgent} to \vv{authd}, which in turn blocks \vv{Installer}.

We also discovered a communication pattern in \vv{Installer} underpinning the
crucial of interactive debugging. It involves four vertices in four threads,
vertex $Vertex_{main}$ in the main thread, and $Vertex_1$ to $Vertex_3$ in
three worker threads. First, the main thread wakes up three worker threads.
Then one worker thread is scheduled to run. At its end, another worker thread,
which waits on mutex lock, is woken in $Vertex_2$, which in turn wakes up the
next worker thread in $Vertex_3$. While \xxx is slicing backward, $Vertex_3$
has two incoming edges: one is from $Vertex_{main}$, and the other one is from
$Vertex_2$. Since users can peek the edges before making decision, they are
likely to figure out that the three worker threads contend with mutex lock, and
all of them are successors of $Vertex_{main}$.


\subsection{Long Running}

In this section, we discuss the cases where the \spinningnode is busy on the
CPU. Most of the text editing apps fall into this bug category. We studied
TeXstudio, TextEdit, Microsoft Word, Sublime Text, Text Mate and CotEditor to
reveal their root causes.

%represent cases in this category.

\paragraph{3-TeXStudio}

TeXstudio~\cite{TeXStudio} is an integrated writing environment for creating
LaTeX documents. We noticed a user reported spinning cursor when he
modified his bib file. Although the issue was closed by the developer, due to
insufficient information to reproduce the bug, we reproduced it with a large bib
file opened in a tab. Each time we touched the file through another editor, vim
for example, the application window showed a spinning cursor.

\xxx recognizes the \spinningnode belongs to the category of \textbf{Long
Running}. Slicing causal path from the vertex, \xxx reaches daemon
``\vv{fseventd}'' and figures out that the long-running function is invoked by a callback
function from this daemon. The advantage of \xxx over other debugging tools is
it helps to narrow down the root cause with the path. If the user's bug report
had included details captured with \xxx, it may have provided the developer with
enough information to reproduce the bug successfully.

\paragraph{6-TextEdit}

TexEdit is a simple word processing and text editing tool shipped by Apple, which
often hangs on the editing of large files. 

\xxx reveals the same causal path with heuristics as with user
interaction. We observed a communicating pattern in the vertices where a kernel
thread was woken up from blocking IO by another kernel thread; and it processed
the timer armed by TextEdit and woke up one of its threads. The first incoming
edge is from the second kernel thread, and the second incoming edge is from
TextEdit(from vertex where the timer armed to where it is processed). Users can
make decision on the vertex base on the event sequence, which implies the story:
TextEdit first arms the timer for IO work, then kernel threads work for it, and
finally it processes the timer and wakes up TextEdit when finished.

It is not surprising because the pattern of vetices in the case fits the
hueristics in \xxx, which chooses the most recent incomming edge.
Although the automatic heuristics works for the particular simple tool, it is
not general enough to make decision for all patterns on complex softwares.

\paragraph{7-MSWord}

Microsoft Word is a large and complex piece of software. \xxx can analyze the
event graph, but it identifies multiple possible root causes: the length of path
interactively sliced from the \spinningnode is 67, while the automatic slicing
generates a path of 136 vertices.

%We rely on user interactions to help speed up the path slicing.

We compared the path and find that the earliest difference exists in the
predecessor of the third vertex in backward paths. In the vertex, user
can learn from the callstack that \vv{Microsoft Word} launches a service
\vv{NSServiceControllerCopyServiceDictionarie} after being woken by another
\vv{Microsoft Word} thread; this thread then sends a message to \vv{launchd}
to register the new service and waits for a reply message. With the most
recent edge heuristics in automatic slicing, \xxx chose \vv{launchd} as its
precedessor, but the user can more precisely identify that the execution segment
is on behalf of the first thread. We rely on user interaction in this case
to find the true root cause, since \xxx has identified multiple
possibilities.

\paragraph{Other Editing Apps}

Select, copy, paste, delete, insert and save are common operations for text
editing. However, these operations on a large context usually trigger spinning
cursors. Depending on their implementations, CotEditor and TextMate successfully
avoid hangs on copy and selection operation. \xxx can helps the developer to
figure out the more efficient way to implement event handlers. We briefly list
the reports from \spinningnode, including the event handler and most costly
functions. We also list corresponding user input event from the path slicing in
Table~\ref{table:texteditapps}.

\begin{table}[H]
\vspace{-0.2cm}
\footnotesize
\centering
  \begin{tabularx}{\columnwidth}{l|l|l}
  \hline
  \hline
                  &                     &\\
  \textbf{BUG-ID} & \textbf{costly API} &UI\\
  \hline
  \hline
  5-Notes         & \begin{tabular}{@{}l@{}}
  					\vv{1)NSDetectScrollDevices}\\
					\vv{\xspace ThenInvokeOnMainQueue}\\
					\end{tabular}
   		          & \begin{tabular}{@{}l@{}}
				  	\vv{system}\\
					\vv{define}\\
					\vv{event}
					\end{tabular}
				  \\
  \hline
  6-TexEdit       & \begin{tabular}{@{}l@{}}
  					\vv{1)[NSTextView(NSPasteboard) \_write}\\
					\vv{\xspace RTFDInRanges:toPasteboard:]}\\
					\vv{2)get\_vImage\_converter}\\
  					\vv{3)get\_full\_conversion\_code\_fragment}\\
					\end{tabular}
				  & \vv{key c}
				  \\
  \hline
  7-MSWord        & \begin{tabular}{@{}l@{}}
					\vv{1)-[NSPasteboard setData:}\\
					\vv{\xspace forType:index:usesPboardTypes:]}\\
 					\vv{2)\_CFStringCreateImmutableFunnel3}\\
  					\vv{3)platform\_memmove}\\
					\vv{4)lseek}, \vv{5)fstat64}, \vv{6)fcntl}\\
					\end{tabular}
				  & \vv{key c}
				  \\
  \hline
  8-SlText   & \begin{tabular}{@{}l@{}} 
					\vv{1)px\_copy\_to\_clipboard}\\
  					\vv{2)\_\_CFToUTF8Len}\\
  					\end{tabular}
				  & \vv{key c}
				  \\
  \hline
  9-TextMate      & \begin{tabular}{@{}l@{}}
  					\vv{1)-[OakTextView paste:]}\\
					\vv{2)CFAttributedStringSet}\\
					\vv{3)TASCIIEncoder::Encode}\\
  					\end{tabular}
				  & \vv{key v}
				  \\
  \hline
  10-CotEditor    & \begin{tabular}{@{}l@{}}
  					\vv{1)CFStorageGetValueAtIndex}\\
					\vv{2)-[NSBigMutableString}\\
					\vv{\xspace characterAtIndex:]}\\
  					\end{tabular}
   		          & \begin{tabular}{@{}l@{}}
				  	\vv{key}\\
				  	\vv{Return}
  					\end{tabular}

				  \\
  \hline
  \end{tabularx}
  \caption{Root cause of spinning cursor in editing Apps}
  \label{table:texteditapps}
\vspace{-0.5cm}
\end{table}


